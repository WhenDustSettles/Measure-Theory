\documentclass{article}
\usepackage[utf8]{inputenc}
\usepackage{amsfonts, amsmath, amssymb}
\usepackage[english]{babel}
% \usepackage{boisik}
\usepackage{amsthm}
\usepackage{graphicx}
\usepackage{mathrsfs}
\usepackage{centernot}
\usepackage[margin=0.5in]{geometry}
\usepackage{empheq}
\usepackage{calligra}


%\usepackage{tgbonum}
%\usepackage{cmbright}
%\usepackage{textcomp}
\usepackage{sectsty}
\sectionfont{\fontsize{20}{20}\selectfont}
%\titlefont{\fontsize{20}{20}\selectfont}
%\usepackage{mathpazo} 

\usepackage{hyperref} %Uncomment for Hyperlinked Table of Contents.

\hypersetup{
	colorlinks,
	citecolor=blue,
	filecolor=black,
	linkcolor=blue,
	urlcolor=blue
}

\usepackage[skins,theorems]{tcolorbox}
\tcbset{highlight math style={enhanced,
		colframe=red,colback=white,arc=0pt,boxrule=1pt}}

\newcommand*\widefbox[1]{\fbox{\hspace{2em}#1\hspace{2em}}}

\subsectionfont{\fontsize{15}{15}\selectfont}

\usepackage{xcolor}

\newcommand*{\myfont}{\fontfamily{cmr}\selectfont}
\newcommand{\id}[1]{{\myfont \text{id}_{#1}}}
\newcommand{\notset}[0]{\setminus}



\title{\Huge Measure Theory\\
	\Large Definitions, Propositions, Theorems \& Proofs}
\author{Animesh Renanse}
\date{\today}
\usepackage{amsthm}

\theoremstyle{definition}
\newtheorem{definition}{$\boxed{\star}$ Definition}
\newcommand{\tit}[1]{\textit{#1}}
\newtheorem{theorem}{$\boxed{\boxed{\circledast}}$ Theorem}

\newcommand{\nll}[0]{\newline\newline}
\theoremstyle{remark}
\newtheorem*{remark}{\textbf{Remark}}

\theoremstyle{definition}
\newtheorem{corollary}{$ \to $ Corollary}

\theoremstyle{definition}
\newtheorem{proposition}{$\checkmark$ Proposition}

\theoremstyle{definition}
\newtheorem{lemma}{Lemma}

\usepackage{mathtools}
\newcommand{\bb}[1]{\mathbb{#1}}
\newcommand{\theor}[0]{\boxed{\textbf{\textit{Theorem}}}}
\newcommand{\algo}[0]{\boxed{\textbf{\textit{Algorithm}}}}
\newcommand{\assum}[0]{\boxed{\textbf{\textit{Assumption}}}}
\newcommand{\formula}[0]{\boxed{\textbf{\textit{Formula}}}}
\newcommand{\defin}[0]{\boxed{\textbf{\textit{Definition}}}}
\DeclarePairedDelimiter\abs{\lvert}{\rvert}
\newcommand{\vect}[1]{\bm{#1}}
\newcommand{\mat}[1]{\bm{\mathrm{#1}}}
\newcommand{\norm}[1]{\Vert #1 \Vert}
\newcommand{\innerp}[2]{\langle #1, #2 \rangle}
\newcommand{\expec}[2]{\mathbb{E}_{#1}\left[ #2\right]}
\newcommand{\seq}[1]{\left\{#1\right\}}
\newcommand{\pder}[2]{\frac{\partial #1}{\partial #2}}
\newcommand{\der}[2]{\frac{d #1}{d #2}}
\newcommand{\estif}[1]{\hat{\theta}\left( #1 \right)}
\newcommand{\esti}[0]{\hat{\theta}}



\newcommand{\defeq}[1]{
		\tcbhighmath[boxrule=1pt,arc=0pt,colback=yellow!10,colframe=black]{\begin{aligned}
				#1
		\end{aligned}}
	}
\newcommand{\theoreq}[1]{
		\tcbhighmath[boxrule=2pt,arc=0pt,colback=blue!10!white,colframe=black]{\begin{aligned}
				#1
		\end{aligned}}}
\renewcommand{\qedsymbol}{\ensuremath{\blacksquare}}

%%%%%%%%%%%%%%%%%%%%%%%%%%%%%%%%%%%%%%%Commands
\newcommand{\union}{\cup}
\newcommand{\intrs}{\cap}
\newcommand{\bunion}{\bigcup}
\newcommand{\bintrs}{\bigcap}
\newcommand{\interior}[1]{{#1}^{\circ}}
\newcommand{\closure}[1]{\overline{#1}}
\newcommand{\boundary}[1]{\partial #1}
\newcommand{\lsum}[2]{l(#1,\mathcal{#2})}
\newcommand{\usum}[2]{u(#1,\mathcal{#2})}
\newcommand{\where}{\;\vert\;}
\newcommand{\R}{\mathbb{R}}
\newcommand{\N}{\mathbb{N}}
\newcommand{\Q}{\mathbb{Q}}
\newcommand{\Z}{\mathbb{Z}}
\newcommand{\alg}[1]{\mathscr{#1}}
\newcommand{\bor}[1]{\mathscr{B}(#1)}
\newcommand{\comp}[1]{#1^{\text{c}}}
\newcommand{\m}[1]{\mu\left (#1\right )}
\newcommand{\limit}[2]{\underset{#1}{\lim}\; #2}
\newcommand{\pow}[1]{\mathscr{P}\left (#1\right )}
\newcommand{\om}[1]{\mu^*\left ( #1\right )}
\newcommand{\lom}[1]{\lambda^*\left (#1\right )}
\newcommand{\set}[1]{\mathscr{#1}}
\newcommand{\msigm}[1]{\set{M}_{#1}}
\DeclarePairedDelimiter{\ceil}{\lceil}{\rceil}
\newcommand{\lm}[1]{\lambda\left (#1\right )}

\begin{document}
\begin{titlepage}
	{\scshape\LARGE Indian Institute of Technology, Guwahati \par}
	\vspace{1cm}
	{\scshape\Large Pre-Final Year : $ 6^{th} $ Semester\par}
	\vspace{1.5cm}
	{\Huge\bfseries Measure Theory\par}
	\vspace{2cm}
	{\Large\itshape Prof. Pratyoosh \textsc{Kumar}\par}
	\vfill
	Scribe by\par
	Animesh Renanse
	
	\vfill
	
	% Bottom of the page
	{\large \today\par}
\end{titlepage}
\tableofcontents
\newpage
\section{Introduction}
Measure Theory is a field of mathematics which deals with formalizing the notion of a \emph{measure} of a given quantity. Traditional notion of a \emph{measure} fails in lots of complicated scenario. Hence, we need a rigorous definition of it. The work in this direction was forwarded by Lebesgue and others in the dawn of $ 20^{th} $ century.
\subsection{Few introductory definitions}
These are few of the basic definitions that one might remember from real analysis.
\begin{itemize}
	\item{\textbf{Limit Points} : $ x\in X $ is called a \emph{limit point} of a subset $ S \subseteq X $ if $\forall\; r > 0 $, $ \exists\; a\neq x$ such that $ a\in S\intrs B_{r}(x) $. That is, ball of any size $ r $ around $ x $ contains atleast one point of $ S $.}
	\item{\textbf{Isolated Points} : $ y\in S $ is called an \emph{isolated point} of a subset $ S\subseteq X $ if $ \exists \;r > 0 $ such that $ (B_r(y)\setminus \{y\}) \intrs S = \Phi $. That is, $ B_r(y) $ contains no other point of $ S $ apart from $ y $. \begin{itemize}
			\item {Also note that every point of closure $ \closure{S} $ is either a limit point or an isolated point of $ S $.}
			\item {More specifically, \emph{any subset of $ \R^d $ is closed if and only if it contains all of it's limit points}.}
		\end{itemize} }
	\item{\textbf{Perfect Set} : $ A $ is called a perfect set if $ A = A^\prime $ where $ A^\prime $ is the set of all \emph{limit points} of $ A $. More conveniently, if $ A $ does not contain any isolated points then it is a perfect set. $ \R $ is a perfect set.}
	\item {\textbf{Symmetric Difference} : $ A $ and $ B $ are two sets then symmetric difference is $ A\Delta B = (A\setminus B) \cup (B\setminus A)$.}
	\item {\textbf{Power Set} : Collection of all subsets of a set $ S $, written as $ P(S) $.}
	\item {\textbf{Lower Bound} : 	A \emph{lower bound} of a subset $ S $ of a poset $ (P,\le) $ is an element $ a \in P$ such that $ a\le x $ for all $ x\in S $.}
	\item {\textbf{Infimum} : A lower bound $ p\in P $ is called an \emph{infimum} of $ S $ if for all lower bounds $ y $ of $ S $ in $ P $, $ y\le p $.}
	\item {\textbf{Upper Bound} : An \emph{upper bound} of a subset $ S $ of a poset $  P$ is an element $ b\in P $ such that $ b\ge x $ for all $ x\in S $.}
	\item {\textbf{Supremum} : An upper bound $ u\in P $ is called a \emph{supremum} of $ S $ if for all upper bounds $ z $ of $ S $ in $ P $, $ z \ge u $.}
	\item {\textbf{Lower Sum} : $ l(f,\mathcal{P}) $ is the sum of the minimum functional values at the partition. That is,
		\[\lsum{f}{P} = \sum_{i=0}^{n-1} m_i (a_{i+1} - a_i)\]
		where $ m_i = \inf\{f(x)\;\vert\;x\in [a_{i-1},a_i]\}$.}
	\item {\textbf{Upper Sum} : Similarly,
		\[\usum{f}{P} = \sum_{i=0}^{n-1} M_i (a_{i+1} - a_i)\]
		where $ M_i = \sup\{f(x)\where x\in [a_{i-1},a_i] \} $.\\
		Remember that the function is \emph{Riemann Integrable} if $ l(f,\mathcal{P}) = u(f,\mathcal{P}) $.}
	\item{\textbf{Countable Sets} : Note the following,
\begin{enumerate}
	\item{\emph{Cardinality} : Sets $ X $ and $ Y $ have the same cardinality if there exists a bijection from $ X $ to $ Y $.}
	\item{\emph{Finite Set} : A set is finite if it is empty or it has the same cardinality as $ \{1,2,\dots,n\} $ for some $ n\in \N $.}
	\item{\emph{Countably Infinite} : If the set has the same cardinality as $ \N $.}
	\item {\emph{Enumeration} : An enumeration of a countably infinite set $ X $ is a bijection of $ \N $ onto $ X $. That is, an enumeration is an infinite sequence $ \{x_n\} $ such that each of the $ x_i $'s are in $ X $ and each element of $ X $ is $ x_i $ for some $ i $.}
	\item {\emph{Countable} : A set is countable if it is finite or countably infinite. For example, $ \N $ is countable, $ \Q $ is also countable (!), $ \R \setminus \Q $ (irrationals) is not countable, $ \R $ is not countable.}
\end{enumerate}	
\item {\textbf{Totally Bounded} : A subset $ B\subseteq X $ is \emph{totally bounded} when it can be covered by a finite number of $ r $-balls for all $ r>0 $. That is,
\[\forall \; r > 0,\;\exists N\in \N,\;\exists a_1,\dots,a_N\in X\;\text{such that}\;B\subseteq \bunion_{n=1}^N B_{r}(a_n)\]
}
\item{\textbf{Compact Set} : A set $ K $ is said to be \emph{compact} when given any cover of balls of possibly unequal radii, there is a finite sub-collection of them that still covers the set $ K $. That is,
\[K\subseteq \bunion_{i}B_{r_i}(a_i)\implies \exists \;i_1,\dots,i_N,\;\;K \subseteq \bunion_{n=1}^N B_{r_{i_n}}(a_{i_n})\]
Note that \emph{compact metric spaces are totally bounded (!)}. Also, \emph{compact sets are closed}.
}
} 
\end{itemize}
The problem begins with Riemann Integrable functions when we see that functions like Dirichlet function ($ 1 $ on  irrational and $ 0 $ on rational points) can become \emph{measurable} even when the function is not continuous! This motivates the need of a formal notion of a \emph{measure}.
\subsection{Homework - I}
\begin{enumerate}
	\item{Every open set in $ \mathbb{R} $ can be written as disjoint union of open intervals.
\begin{proof}
	\label{H1-1}
	Let $ G\subseteq \R$ be a open subset. Now by definition of an open subset, we have that for any $ x\in G $, there exists atleast one open subset $ U $ such that $ x\in U \subseteq G $. Now consider the following union of all such open subsets of $ x $,
	\[U_x = \bunion_{x\in U\subseteq G} U\]
	It's now easy to see that $ U_x $ is the largest such subset of $ G $, as any other $ V\subseteq G $ such that $ x\in V $ is by definition contained in $ U_x $. Moreover, $ U_x $ is an interval as it is an arbitrary union of open intervals. Now, define the following relation on $ G $:
	\[y\sim x \;\iff\; y\in U_x\]
	Now we clearly have that $ x\in U_x $ (reflexive); for $ y\sim U_x $ we have $ U\subseteq U_x $ such that $ x,y\in U $, hence $ x\in U_y $ (symmetric); for $ x\in U_y $ and $ y\in U_z $, we have that $ x,y,z\in U_y $, since $ z\in U_y \subseteq G $ so $ U_y\subseteq U_z $, so $ x\in U_z $ (transitive). Hence $ \sim $ is an equivalence relation, hence $ \sim $ partitions the set $ G $. Denote the set of all equivalence classes as $ \mathcal{I} $ so we get 
	\[G = \bunion_{I\in \mathcal{I}} I\]
	such that $I_1\intrs I_2 = \Phi$ for any $ I_1,I_2\in \mathcal{I} $. Now note that for any $ I \in \mathcal{I}$ is open because each $ I $ is generated by the relation $ \sim $ such that $ y\sim x $ iff $ y\in U_x $. Hence for any $ z\in I $, we have $ z\in U_x \subseteq G$ where $ U_x $ is open. Therefore, we have $ G = \union_{I\in \mathcal{I}} I $ for disjoint open intervals in $ \mathcal{I} $.
\end{proof}	
}
	\item{Prove that every non-empty perfect subset of $ \mathbb{R} $ (or $ \mathbb{R}^n $) is uncountable. That is, if $ A= A^\prime $ then $ A $ is uncountable.
\begin{proof}
	Take $ A\subseteq \R $ to be a perfect subset. Since $ A $ it is perfect, therefore, it must contain all of it's limit points or, equivalently, contains no isolated points. Clearly, then, $ A $ cannot be finite, but can only be countably infinite or uncountable. If it is uncountable, then the proof is over. If $ A $ is countably infinite, then we can write $ A $ as the following :
	\[A= \{a_1,a_2,\dots\}.\]
	Construct a ball around $ a_{i_1} $ of any radius $ r_1 >0 $. Since $ A $ is perfect, therefore $ \exists $ $ a_{i_2} \in B_{r_{1}} (a_{i_1}) \intrs A = C_1$. Similarly, for some $ r_2>0 $, we have $ a_{i_3} \in B_{r_2}(a_{i_2}) \intrs B_{r_1}(a_{i_1}) \intrs A = C_2$ such that $ a_{i_1}\notin C_2 $ and so on. In general, we would have the following,
	\[a_{i_{n+1}} \in \left (\bintrs_{j=1}^n B_{r_j}(a_{i_j}) \right ) \intrs A = C_n.\]
	Now, consider $ C = \intrs_n C_n $. Since $ C_{n+1} \subseteq C_n $, therefore $ C\neq \Phi $. But, $ a_i\notin C $ for any $ i \in\N $ as $ a_i \notin C_{i+1} $. Therefore we have a contradiction. Hence $ A $ cannot by countably infinite, it must only be uncountable.
\end{proof}	
}
%\item {A subset of R is compact if, and only if, it is
%	closed and bounded.
%\begin{proof}
%	\emph{Do It!}
%\end{proof}
%}	
\item {In the definition of Lebesgue Outer Measure on $ \R $, one can instead take $ \alg{C}_A $ to be collection of infinite sequences of the any form from $ \{[a_n,b_n]\}, \{(a_n,b_n)\} $ or $ \{(a_n,b_n]\} $.
\begin{proof}
	\emph{Refer Proof of Proposition \ref{P-9}}.
\end{proof}
}
\item {Show the following:
\[\bunion_{n=1}^N E_n = \bunion_{n=1}^N \left (E_n \intrs \comp{\left ( \bunion_{k<n} E_k\right )}\right )\]
\begin{proof}
Take $ x\in \bunion_{n=1}^N E_n $. Then $ \exists \; E_k $ for some $a  $ such that $ x\in E_a $. Now, clearly, $ x\in E_a \subseteq \left ( \bunion_{k<a} E_k\right )^c $, hence $ x\in\left (E_a \intrs \comp{\left ( \bunion_{k<a} E_k\right )}\right )  $. Hence, we have $ \bunion_{n=1}^N E_n \subseteq  \bunion_{n=1}^N \left (E_n \intrs \comp{\left ( \bunion_{k<n} E_k\right )}\right ) $. The converse is easy to see too.
\end{proof}
}

\end{enumerate}
\newpage
\subsection{Measure of an Interval}
In the definition of Riemann Integral, we see that a function is Riemann Integrable if $ \usum{f}{P} = \lsum{f}{P} $. But it depends upon the definition of the \emph{length of an interval}.\\
Note that in the traditional notion of \emph{measure}, the number of points has no contribution in length. Hence the question becomes:
\[\text{\emph{How to measure a set $ A $ in $ \R $?}}\]
For this, we define \emph{Outer Measure}.
\subsubsection{Outer Lebesgue Measure}
The concept of this topic began from the following question : If we consider the cauchy sequence of Riemann Integrable functions on $ \R $, will it converge to a Riemann Integrable function?\\
We follow the following notion of measure (atleast for the first part of the course). We do approximation by approximation.\\\\
Suppose $ A\subseteq \R $. We define $ m^*(A) $ as the \emph{Outer Lebesgue Measure} defined by
\begin{equation*}
	\begin{split}
		m^*(A) = \inf \left \{\sum_{i=1}^\infty l(I_i)\where A\subseteq \bunion_{i}I_i\right \}
	\end{split}
\end{equation*}
where $ I_i = [a_i,b_i) $. Since we are taking infimum of the the collection, hence Note that it is approximating measure from outside. Also note that this is different from the definition of measurable sets.\\
Outer measure is a function $ m^* : P(\R) \to [0,\infty] $. 
%\subsubsection{Properties}
%We have the following properties:
%\begin{itemize}
%	\item{$ m^*(\phi) = 0 $.}
%	\item{$ m^*(\{a\}) = 0 $.}
%	\item{}
%\end{itemize}
%\begin{lemma}
%	$ A = I $ then $ m^*(A) = l(I) $ if $ I $ is a closed interval.
%\end{lemma}
\newpage
\section{Measures}
\subsection{Algebras \& Sigma-Algebras}
\begin{definition}
	(\textbf{Algebra/Field}) Let $ X $ be an arbitrary set. A collection $ \alg{A} \subseteq P(X)$ of subsets of $ X $ is an algebra on $ X $ if:
	\begin{itemize}
		\item {$ X\in \alg{A} $.}
		\item {$ A\in \alg{A} \implies A^c \in \alg{A}$.}
		\item {For each finite sequence $ A_1,A_2,\dots, A_n \in \alg{A}$ implies that
	\[\bunion_{i=1}^n A_i \in \alg{A}\]	
	 }
 \item {For each finite sequence $ A_1,A_2,\dots,A_n\in\alg{A} $ implies that
\[\bintrs_{i=1}^nA_i \in\alg{A}\] 
}
	\end{itemize}
\end{definition}
\hrulefill
\begin{definition}
	(\textbf{$ \sigma $-Algebra/$ \sigma $-Field}) Let $ X $ be an arbitrary set. A collection $ \alg{A} \subseteq P(X)$ of subsets of $ X $ is a $ \sigma $-algebra on $ X $ if:
	\begin{itemize}
		\item {$ X\in \alg{A} $.}
		\item {$ A\in\alg{A} \implies A^c\in \alg{A}$.}
		\item {For each infinite sequence $ \{A_i\} $ such that $ A_i\in \alg{A} $, it implies that
	\[\bunion_{i=1}^\infty A_i \in\alg{A}\]	
	}
\item{For each infinite sequence $ \{A_i\} $ such that $ A_i\in \alg{A} $, it implies that
	\[\bintrs_{i=1}^\infty A_i \in\alg{A}\]} 
	\end{itemize}
\end{definition}
\hrulefill
\begin{proposition}
	\label{P-1}
	Let $ X $ be a set. Then the intersection of an arbitrary non-empty collection of $ \sigma $-algebras on $ X $ is a $ \sigma $-algebra on $ X $.
\end{proposition}
\begin{proof}
	Consider a collection $ \alg{C} $ of $ \sigma $-algebras on $ X $. Denote $ \alg{A} = \bintrs\alg{C} $ as intersection of all $ \sigma $-algebras in $ \alg{C}$. We can now easily see that any subset in $ \alg{A} $ would be present in every $ \sigma $-algebra present in collection $ \alg{C} $, hence, it would obey all properties of a $ \sigma $-algebras. Therefore, $ \alg{A} $ is a $ \sigma $-algebra.
\end{proof}
\begin{corollary}
	Let $ X $ be a set and let $ \alg{F}\subseteq P(X) $ be a family of subsets of $ X $. Then there exists a smallest $ \sigma $-algebra on $ X $ that includes $ \alg{F} $.
\end{corollary}
\begin{proof}
	Consider any given family $ \alg{F} \subseteq P(X)$ and just take intersection of the family $ \alg{C} $ of all $ \sigma $-algebras which contains $ \alg{F} $ to construct this smallest $ \sigma $-algebra. 
\end{proof}
\hrulefill
\begin{definition}
	(\textbf{Generated $ \sigma $-algebra}) The smallest $ \sigma $-algebra on $ X $ containing a given family $ \alg{F}  \subseteq P(X) $ of subsets is called the $ \sigma $-algebra \emph{generated} by $ \alg{F} $, denoted as $ \sigma(\alg{F}) $.
\end{definition}
\hrulefill
\begin{definition}
	(\textbf{Borel $ \sigma $-algebra on $ \R^d $}) It is the $ \sigma $-algebra on $ \R^d $ generated by the collection of all open subsets of $ \R^d $, denoted as $ \alg{B}(\R^d) $.
\end{definition}
\hrulefill
\begin{definition}
	(\textbf{Borel Subsets of $ \R^d $}) Any $ A \subseteq \R^d$ is called a Borel subset of $ \R^d $ if $ A\in \bor{\R^d} $.
\end{definition}
\hrulefill
\newpage
\begin{proposition}
	\label{P-2}
	The Borel $ \sigma $-algebra on $ \R $, $ \bor{\R} $, of Borel subsets of $ \R $ is  generated by each of the following collection of sets:
	\begin{enumerate}
		\item {The collection of all closed subsets of $ \R $.}
		\item {The collection of all subintervals of $ \R $ of the form $ (-\infty,b] $.}
		\item {The collection of all subintervals of $ \R $ of the form $ (a,b] $.}
	\end{enumerate}
\end{proposition}
\begin{proof}
	To show all of these, consider the three $\sigma$-algebras $ \alg{A}_1, \alg{A}_2, \alg{A}_3 $ corresponding to conditions 1,2 \& 3 respectively and try to prove $ \alg{A}_3\subseteq \alg{A}_2\subseteq \alg{A}_1 \subseteq \bor{\R}$ together with $ \bor{\R}\subseteq \alg{A}_3 $. The first three inclusions are trivial to see. For the case that $ \bor{\R}\subseteq \alg{A}_3 $, simply note that any open subset can be made by unions of the sets of form $ (a,b] $ and by Homework-I,1, each open set is union of open subsets.
\end{proof}
\hrulefill
\begin{proposition}
	\label{P-3}
	The $ \sigma $-algebra $ \bor{\R^d} $ of Borel subsets of $ \R^d $ is generated by each of the following collections:
	\begin{enumerate}
		\item{The collection of all closed subsets of $ \R^d $.}
		\item {The collection of all closed half-spaces in $ \R^d $ that have the form $ \{(x_1,\dots,x_d)\where x_i\le b\} $ for some index $ i $ and some $ b\in \R $.}
		\item {The collection of all rectangles in $ \R^d $ that have the form 
	\[\{(x_1,\dots,x_d)\where a_i< x_i\le b_i\;\text{for}\;i=1,\dots,d\}\]	
	}
	\end{enumerate}
\end{proposition}
\begin{proof}
	Almost the same as in Proposition \ref{P-2}. $\alg{A}_1 \subseteq \bor{\R^d}$ trivially by definition. $ \alg{A}_2 \subseteq \alg{A}_1 $ as $ \{(x_1,\dots,x_d)\where x_i\le b\} $ is closed itself. $ \alg{A}_3\subseteq \alg{A}_2 $ by the observation that $ \{(x_1,\dots,x_d)\where a_i< x_i\le b_i\} $ is made by the difference of two subsets of the form $ \{(x_1,\dots,x_d)\where x_i\le b_i\} $ and $ \{(x_1,\dots,x_d)\where x_i> a_i\} $, the latter is the complement of a certain subset in $ \alg{A}_2 $, moreover, $ \{(x_1,\dots,x_d)\where a_i< x_i\le b_i\;\text{for}\;i=1,\dots,d\} $ is then constructed by intersection of $ d $ such subsets. Finally, $ \bor{\R^d} \subseteq \alg{A}_3$ can be seen via the fact that open subsets in $ \R^d $ are made by union of rectangles of type 3 and as such, they are called open subsets. 
\end{proof}
\hrulefill

\emph{Conditions for an algebra to become a $ \sigma $-algebra.}
\begin{proposition}
	\label{P-4}
	Let $ X $ be a set and let $ \alg{A} $ be an algebra on $ X $. Then, $ \alg{A} $ is a $ \sigma $-algebra on $ X $ if \textbf{either}
	\begin{itemize}
		\item {$ \alg{A} $ is closed under the formation of \textbf{unions} of \emph{increasing} sequence of sets, or,}
		\item {$ \alg{A} $ is closed under the formation of \textbf{intersections} of \emph{decreasing} sequence of sets.}
	\end{itemize}
\end{proposition}
\begin{proof}
	Take any countably infinite collection of subsets $ A_1,A_2,\dots \in \alg{A}$ where $ \alg{A} $ is an algebra. Due to the definition of an algebra, we have that $ C_n = \bunion_{i = 1}^n A_i \in \alg{A}$ for any $ n\ge 1\in \Z_+ $. Now note that $ C_1\subseteq C_2 \subseteq \dots  $, that is, the sequence $ \{C_n\} $ forms an increasing sequence of sets. Hence, by the requirement of the question, we have that $ \bunion_{i=1}^\infty C_i \in \alg{A} $. But then we also have that $ \bunion_{i = 1}^\infty A_i \subseteq \bunion_{i=1}^\infty C_i \in \alg{A}$. Hence we have the required condition for part 1. For part 2, we can see that $ \comp{C_1} \supseteq \comp{C_2} \supseteq \dots $ is a decreasing sequence of sets. Then we must have, by the requirement of the question, that $ \bintrs_{i=1}^{\infty} \comp{C_i} = \comp{\left (\bunion_{i=1}^\infty C_i\right )} \in \alg{A}$. But then by definition of algebra, we must have $ \bunion_{i=1}^\infty C_i\in \alg{A} $, which already contains the countably infinite union $ \bunion_{i=1}^\infty A_i $.
\end{proof}
\hrulefill
\newpage
\subsection{Measures}
\begin{definition}
	(\textbf{Countably Additive Function}) Let $ X $ be a set and $ \alg{A} $ be a $ \sigma $-algebra on $ X $. Function $ \mu : \alg{A} \longrightarrow [0,+\infty]$ is said to be \emph{countably additive} \textbf{if} it satisfies:
	\[\mu\left (\bunion_{i=1}^\infty A_i\right ) = \sum_{i=1}^\infty \mu(A_i) \]
	for \textbf{each} infinite sequence $ \{A_i\} $ of \textbf{disjoint} sets in $ \alg{A} $.
\end{definition}
\hrulefill
\begin{definition}
	(\textbf{Measure}) A \emph{measure} on $ \alg{A} $ is a function $ \mu: \alg{A} \to [0,+\infty] $ that is \textbf{countably additive} and satisfies:
	\[\mu(\Phi) = 0.\]
\end{definition}
\begin{remark}
	This is sometimes also referred as \emph{countably additive measure} on $ \alg{A} $.
\end{remark}
\hrulefill
\begin{definition}
	We have following definitions to compactly represent above definitions:
	\begin{enumerate}
		\item {(\textbf{\emph{Measure Space}}) If $ X $ is a set, $ \alg{A} $ is a $ \sigma$-algebra on $ X $ and if $ \mu $ is a measure on $ \alg{A} $, then the triple $ (X,\alg{A}, \mu) $ is called a \emph{measure space}.}
		\item {(\textbf{\emph{Measurable Space}}) If $ X $ is a set and $ \alg{A} $ is a $ \sigma $-algebra on $ X $, then the pair $ (X,\alg{A}) $ is called a \emph{measurable space}.}
	\end{enumerate}
\end{definition}
\hrulefill
\begin{proposition}
	\label{P-5}
	Let $ (X,\alg{A},\mu) $ be a measure space and let $ A, B\in \alg{A} $ such that $ A\subseteq B$. Then,
	\begin{itemize}
		\item {We have $ \mu(A) \le \mu(B) $.}
		\item {Additionally, if $ A $ satisfies that $ \mu(A)< +\infty $, then:
	\[\mu(B-A) = \mu(B) - \mu(A).\]	
	}
	\end{itemize}
\end{proposition}
\begin{proof}
	Note that $ A $ and $ B\intrs \comp{A} $ are disjoint sets in the sigma algebra $ \alg{A} $. Hence we can write, by countably additive property of $ \mu $, that:
	\begin{equation*}
		\begin{split}
			\mu(A\union \left (B\intrs \comp{A}\right )) &= \mu(B)\\
			&=  \mu(A) + \mu(B\intrs \comp{A})
		\end{split}
	\end{equation*}
Since $ \mu(B\intrs \comp{A}) \ge 0$, hence $ \mu(A)\le \mu(B) $. Moreover, if $ \mu(A) <\infty $, then we can additionally write $ \mu(B\intrs \comp{A})  = \mu(B) - \mu(A)$.
\end{proof}
\hrulefill
\begin{definition}
	Let $ \mu $ be a measure on a measurable space $ (X,\alg{A}) $. Then,
	\begin{itemize}
		\item {(\textbf{\emph{Finite Measure}}) If $ \mu(X)< +\infty $.}
		\item {(\textbf{\emph{$ \sigma $-Finite Measure}}) If $ X = \bunion_{i}A_i $ where $ A_i \in\alg{A} $ such that $ \mu(A_i)< +\infty $ for all $ i\in \N $.}
	\end{itemize}
\end{definition}
\begin{remark}
	In other words, a subset $ A\in\alg{A} $ is \emph{$ \sigma $-finite} if it is a union of a countable sequence of sets that are in $ \alg{A} $ and are of finite measure under $ \mu $.
\end{remark}
\hrulefill
\subsubsection{Elementary Properties of Measures}
\begin{proposition}
	\label{P-6}
	Let $ (X,\alg{A}, \mu) $ be a measure space. If $ \{A_k\} $ is an \textbf{arbitrary} sequence of sets that belong to $ \alg{A} $, then,
	\[\mu\left (\bunion_{k=1}^\infty A_k\right) \le \sum_{k=1}^\infty \mu(A_k).\]
\end{proposition}
\begin{proof}
	Denote $ B_1 = A_1 $ and $ B_i = A_i \intrs \comp{\left (\bunion_{k=1}^{i-1} A_k \right)} $. Note that $ B_i $ and $ B_j $ are disjoint for distinct $ i $ and $ j $. Since $ \{A_k\}\in \alg{A} $, therefore $ \{B_i\} \in \alg{A}$. Moreover, $ \bunion_{i=1}^{\infty} B_i = \bunion_{k=1}^\infty A_k $ by construction. We then get,
	\begin{equation*}
		\begin{split}
			\mu\left (\bunion_{k=1}^\infty A_k\right ) = \mu\left (\bunion_{i=1}^\infty B_i\right ) &= \sum_{i=1}^\infty \mu(B_i)\\
			&\le \sum_{i=1}^\infty \mu(A_i)\;\;\;\;\text{($ \because \;B_i \subseteq A_i$ by construction.)}
		\end{split}
	\end{equation*}
Hence proved.
\end{proof}
\hrulefill
\newpage
\begin{proposition}
	\label{P-7}
	Let $ (X,\alg{A},\mu) $ be a measure space.
	\begin{itemize}
		\item {If $ \{A_k\} $ is an \textbf{increasing} sequence of sets in $ \alg{A} $, then
	\[\m{\bunion_{k}A_k} = \limit{k}{\m{A_k}}.\]	
	}
\item {If $ \{A_k\} $ is a \textbf{decreasing} sequence of sets in $ \alg{A} $ \textbf{and} if $ \m{A_n} < + \infty$ holds for some $ n $, then
\[\m{\bintrs_k A_k} = \limit{k}{\m{A_k}}.\]
}
	\end{itemize}
\end{proposition}
\begin{proof}
	Consider $ \{A_i\} \in \alg{A} $ be an increasing sequence of subsets of $ X $. Define $ B_1 = A_1 $ and $ B_i = A_i - A_{i-1} = A_i \intrs \comp{A_{i-1}}$. Clearly, $ B_i $ are disjoint for distinct $ i $. Hence, we can now write,
	\[\m{\bunion_i B_i} =\sum_{i=1}^\infty \m{B_i}.\]
	Moreover, $ \bunion_i B_i = \bunion_k A_k $ and $ B_i = \bunion_{j=1}^i A_j $, therefore we can modify the above as
	\[\m{\bunion_k A_k} = \limit{k\to \infty}{\sum_{i=1}^k \m{\bunion_{j=1}^i A_j}}= \limit{k\to\infty }{\m{\bunion_{i=1}^k \bunion_{j=1}^i A_j}} = \limit{k\to \infty }{\m{A_k}}\]
	Now, consider $ \{A_k\} $ to be a decreasing sequence of subsets in $ \alg{A} $. Take $ n\in \N $ such that $ \m{A_n} < + \infty$. Denote $ B_i = A_n - A_{n+i} $. We have each $ B_i $ to be disjoint by construction for any two $ i $'s and $ \{B_i\} $ is increasing. Moreover, $ \bunion_k B_k = A_n - \bintrs_{i}A_{n+i}$. Hence by first part, we have:
	\begin{equation*}
		\begin{split}
			\m{\bunion_{k}B_k} &= \m{A_n - \bintrs_i A_{n+i}}\\
			&= \m{A_n} - \m{\bintrs_i A_{n+i}}\;\;\;\;\text{(Since $ \{A_k\} $ is decreasing)}\\
			&= \limit{k\to \infty}{\m{B_k}} = \limit{k\to \infty}{\m{A_n - A_{n+k}}}\;\;\text{(Part 1.)}\\
			&= \m{A_n} - \limit{k\to\infty}{\m{A_{n+k}}}
		\end{split}
	\end{equation*} 
Hence, we get the second result (from $ 2^{nd} $ and $ 4^{th} $ equality).
\end{proof}
\hrulefill

\emph{The following partial converse of Proposition \ref{P-7} helps in checking whether finitely additive measure is countably additive.}
\begin{proposition}
	\label{P-8}
	Let $ (X,\alg{A}) $ be a measurable space and let $ \mu$ be a \textbf{finitely additive} measure on $ (X,\alg{A}) $. Then $ \mu $ is a countably additive measure if \textbf{either}
	\begin{itemize}
		\item {The following holds for each \textbf{increasing sequence} $ \{A_k\} $ of sets that belong to $ \alg{A} $:
	\[\limit{k}{\m{A_k}} = \m{\bunion_k A_k}\]	
	or,
	}
\item {The following holds for each \textbf{decreasing sequence} $ \{A_k\} $ that belongs to $ \alg{A} $ which \textbf{satisfies} $ \bintrs_k A_k = \Phi $:
\[\limit{k}\m{A_k} = 0.\]
 }
	\end{itemize}
\end{proposition}
\begin{proof}
	Consider any disjoint sequence of sets $ \{A_i\} $ in $ \alg{A} $. Let the first part be true. We can construct an increasing sequence from $ \{A_i\} $ by considering a new sequence $ \{B_i\} $ where $ B_k = \bunion_{j=1}^k A_j  $. Now, finite additivity of $ \mu $ means that $ \m{B_k}  = \m{\bunion_{j=1}^{k} A_j} = \sum_{j=1}^k \m{A_j}$. Also, by the condition of part 1, we have $ \limit{k\to \infty}{\m{B_k}}  = \m{\bunion_{k} B_k} = \m{\bunion_k A_k}$. Hence proved part 1.\\
	For part 2, we simply have for any disjoint sequence $ \{A_i\} $ in $ \alg{A} $, a decreasing sequence $ \{B_k\} $ given as $ B_k = \bunion_{i=k}^\infty A_i $. Again, we can write,
	\begin{equation*}
		\begin{split}
			\m{\bunion_{i=1}^\infty A_i} &= \m{\left (\bunion_{i=1}^k A_i\right ) \bunion \left (\bunion_{j=k+1}^\infty A_j\right )}\\
			&= \m{\left (\bunion_{i=1}^k A_i\right ) \bunion B_{k+1}}\\
			&= \m{\bunion_{i=1}^k A_i} + \m{B_{k+1}}\;\;\text{(Since these two are disjoint.)}\\
			&= \sum_{i=1}^k \m{A_i} + \m{B_{k+1}}   
		\end{split}
	\end{equation*}
	Taking $ \limit{k\to \infty}{} $ both sides then yields us 
	\[\m{\bunion_{i=1}^\infty A_i} = \sum_{i=1}^\infty \m{A_i} + 0 \]
as $ \limit{k\to \infty}{\m{B_k}} =0$ is given in the assumption of part 2.
\end{proof}
\hrulefill
\newpage
\subsection{Outer Measures}
\begin{definition}
	(\textbf{Outer Measure}) Let $ X $ be a set and let $  \pow{X} $ be the collection of all subsets of $ X $. An \emph{outer measure} on $ X $ is a function $ \mu^* : \pow{X} \longrightarrow [0,+\infty] $ such that:
	\begin{itemize}
		\item {For the empty set $ \Phi $,
	\[\om{\Phi} = 0\]	
	}
		\item {If $ A\subseteq B\subseteq X $, then
	\[\om{A}\le \om{B}.\]	
	}
\item {If $ \{A_n\} $ is an infinite sequence of subsets of $ X $, then
\[\om{\bunion_n A_n} \le \sum_n \om{A_n}\]
}
	\end{itemize}
\end{definition}
\hrulefill
\begin{definition}
	(\textbf{Lebesgue Outer Measure on $ \R $}) For each subset $ A\subseteq \R $, let $ \alg{C}_A $ be the \emph{set of all infinite sequences} $ \{(a_i,b_i)\} $ of \emph{bounded open intervals} such that $ A\subseteq \bunion_i (a_i,b_i) $. That is, 
	\[\alg{C}_A = \{\left \{(a_i,b_i)\}\where A\subseteq \union_i (a_i,b_i)\;\text{and}\;a_i,b_i\in \R\right \}\]
	Then, $ \lambda^* : \pow{\R}\longrightarrow [0,+\infty]$ is the Lebesgue outer measure, defined by:
	\begin{equation}
		\lom{A} = \inf \left \{\left .\sum_{i} (b_i - a_i)\;\right \vert\;\{(a_i,b_i)\}\in \alg{C}_A \right \}
	\end{equation} 
\end{definition}
\hrulefill

\emph{To verify that $ \lambda^* $ is indeed an outer measure.}
\begin{proposition}
	\label{P-9}
	Lebesgue outer measure on $ \R $ is an outer measure and it assigns to each subinterval of $ \R $ it's length.
\end{proposition}
\begin{proof}
	Denote $ \set{C}_A = \{\{(a_i,b_i)\}\where A \subseteq \union_i(a_i,b_i)\}$. To show that $ \lambda^* $ is an outer measure, we first need to show that $ \lom{\Phi} = 0 $. For that, consider the set of all infinite sequences $ \{(a_i,b_i)\} \in \alg{C}_\Phi$, that is (trivially) $\Phi\subseteq  \union_{i}(a_i,b_i) $, such that $ \sum_{i}(b_i - a_i) <\epsilon $ for all $ \epsilon>0 $. Then, if we denote $ \alg{L}_A = \left \{\sum_{i} (b_i - a_i)\where \{(a_i,b_i)\}\in \alg{C}_A\right \} $, then $ \inf\alg{L}_\Phi = 0$ as for any lower bound $ l $ of $ \alg{L}_A $, if $ l>0 $ then $ \exists \; \{(a_i,b_i)\}\in  \set{C}_\Phi$ such that $ \sum_i(b_i - a_i) < l $, hence $ l\le 0 $, or $ \inf\set{L}_\Phi = 0$. \\
	Second, we need to show that if $ A\subseteq B \subseteq X $, then $ \lom{A}\le \lom{B} $. For this, consider $ A\subseteq B $. Clearly, we have that $ \set{C}_B \subseteq \set{C}_A $, therefore $ \set{L}_B \subseteq \set{L}_A $ and hence $ \inf \set{L}_B \ge \inf \set{L}_A $.\\
	Third, we need to show that for any infinite sequence $ \{A_n\} $ of subsets of $ X $,
	\[\lom{\bunion_n A_n}\le \sum_{n}\lom{A_n}\]
	For this, consider the Lebesgue outer measure of $ A_n $, that is, $ \lom{A_n} $. We must have, that for any infinite sequence $ \{(a_{n,i}, b_{n,i})\} \in \set{C}_{A_n} $, that 
	\[\sum_{i=1}^\infty (b_{n,i} - a_{n,i}) \ge \lom{A_n}.\]
	Hence, consider that the difference is upper bounded according to $ n $, that is the sequence $ \{(a_{n,i},b_{n,i})\}\in \set{C}_{A_n} $ is such that,
	\[\sum_{i=1}^\infty (b_{n,i} - a_{n,i}) - \lom{A_n} \le \epsilon/2^n.\]
	 Now, we can cover the entire $ \bunion_i A_i $ by the union of the above intervals, that is, 
	 \[\bunion_i A_i \subseteq \bunion_n \bunion_i (a_{n,i},b_{n,i}).\]
	 Now, we know that
	 \begin{equation*}
	 	\begin{split}
	 		\lom{\bunion_iA_i} &= \inf\set{L}_{\union_iA_i}.
	 	\end{split}
	 \end{equation*}
	 But since
	 \[\sum_{n}\sum_i (b_{n,i}- a_{n,i}) \in \set{L}_{\union_i A_i},\]
	 and
	 \[\sum_n \left (\sum_i(b_{n,i}-a_{n,i}) - \lom{A_n}\right ) \le \sum_n \epsilon/2^n\]
	 which is equal to 
	 \[\sum_n \sum_i (b_{n,i}-a_{n,i}) - \sum_n \lom{A_n} \le \epsilon \times 1\]
	 or, 
	 	 \[\sum_n \sum_i (b_{n,i}-a_{n,i}) \le \sum_n \lom{A_n} +\epsilon \]
	 	 and since $ \lom{\bunion_i A_i} = \inf\set{L}_{\bunion_i A_i} $, therefore,
	 	 \[\lom{\bunion_i A_i} \le \sum_n\sum_i (b_{n,i}-a_{n,i}) \le \sum_n\lom{A_n}\]
Hence proved. \\\\
Now, we need to show that $ \lambda^* $ assigns each subinterval it's length.\\
For this first show that $ \lom{[a,b]} \le b-a $. This is easy to show if we take,
\[[a,b] = \bunion_{i} (a_i,b_i)\]
where $ (a_1,b_1) = (a,b) $, $ (a_i,b_i) = (a-\epsilon/2^i,a) $ for all even $ i $ and $ (a_j,b_j) = (b,b+\epsilon/2^j) $ for all odd $ j $. Now,
\begin{equation*}
	\begin{split}
		\sum_i (b_i-a_i) &= (b-a) + \sum_{i=2,4,\dots} \epsilon/2^i + \sum_{i=3,5,\dots} \epsilon/2^i\\
		&= b-a + \sum_{i=1,2,\dots} \epsilon/2^i\\
		&= b-a + \epsilon
	\end{split}
\end{equation*}
therefore $ \lom{[a,b]} = \inf \set{L}_{[a,b]} \le b-a + \epsilon $ for all $ \epsilon>0 $, hence $ \lom{[a,b]} \le b-a $. \\
Now, to show the converse that $ b-a \le\lom{[a,b]} $, we first note that $ [a,b] $ is compact, so for any infinite cover $ \{(a_i,b_i)\} \in \set{C}_{[a,b]} $, there exists a finite subcover $ \{(a_i,b_i)\}_{i=1}^n $ of $ [a,b] $. Now, since $ \lambda^* $ is an outer measure, therefore,
\[  b-a \le \sum_{i=1}^n \lom{(a_i,b_i)}\le \sum_{i=1}^\infty \lom{(a_i,b_i)} \in \set{L}_{[a,b]} \] 
Therefore, $ b-a $ is a lower bound of $ \set{L}_{[a,b]} $ and hence $ b-a \le \inf \set{L}_{[a,b]} = \lom{[a,b]} $.\\
Hence $ \lom{[a,b]} = b-a $.\\
Now since, one can construct subintervals of the form $ (a,b] $ or $ [a,b) $ from the following manner:
\[(a,b] \subseteq (a,b) \bunion \left (\bunion_{n}[b,b+\epsilon/2^n]\right )\]
from which we get that $ \lom{(a,b]} \le b-a  $ and also,
\[[a,b] \subseteq (a,b] \bunion \left ( \bunion_n [a-\epsilon/2^n ,a] \right )\]
which yields $ b-a \le \lom{(a,b]} $. Similarly for $ (-\infty,b] $ to show that $ \lom{(-\infty,b]} = +\infty $. 
\end{proof}
\hrulefill
\newpage
\subsection{Exercises - II}
\begin{enumerate}
	\item{\begin{proof}
			Note that $ \Q  $ is dense, that is $ \closure{\Q} = \R $. Therefore, $ \closure{\Q\intrs [0,1]} = [0,1] $. Now, since \emph{closure of union is union of closures}\footnote{Proof done in diary : \emph{$ 3^{rd} $ March, 2018} - The date of reference.}, then we get,
			\[\Q\intrs [0,1] \subseteq \closure{\Q\intrs [0,1]} = [0,1]\subseteq \bunion_{n=1}^m \closure{(a_n,b_n]} = \bunion_{n=1}^m [a_n,b_n]\]
			Now since $ \lambda^* $ is an outer measure, therefore,
			\[\lom{[0,1]} = 1 \le \lom{\bunion_{n=1}^m [a_n,b_n]} \le \sum_{n=1}^m l(I_n) \]
			because $ [0,1] $ is a subinterval, closure of unions is union of closures and infinite subadditivity implies finite subadditivity. Hence proved.
		\end{proof}
}
\item {\begin{proof}
		\label{E2-2}
		Since $ \lom{A} = \inf \set{L}_A$ where
		\[\set{L}_A = \left \{\sum_{i} (b_i - a_i)\where \{(a_i,b_i)\} \in \set{C}_A \right \}\]
		and 
		\[\set{C}_A = \{\{(a_i,b_i)\} \where A\subseteq \union_n (a_n,b_n) \}.\]
		Therefore, for any $ \epsilon >0 $, and if we write $ \lom{A} +\epsilon $, then $ \exists$ $ \{(a_i,b_i)\} \in \set{C}_A $ such that $ \lom{A} + \epsilon \ge \sum_i (b_i - a_i) \ge \lom{\bunion_i (a_i,b_i)} $. We can clearly see that $ \bunion_i (a_i,b_i) $ is union of open intervals and since arbitrary union of open intervals is open, therefore, we have the union of the sequence $ \{(a_i,b_i)\} $ as the needed open interval.
	\end{proof}
}
\item {\begin{proof}
		\label{E2-3}
		From the previous question 2, we have that for $ A \subseteq \R $ and for any $ \epsilon>0 $, there exists an open subset $ O\supseteq A $ such that $ \lom{O} \le \lom{A} + \epsilon $. Now, consider the sequence $ \epsilon_n = 1/n $ and the corresponding open subsets $ O_n $ containing $ A $. Clearly, $ A \subseteq \bintrs_n O_n $. Since $ G = \bintrs_n O_n$ is countable intersection of open subsets, therefore $ G $ is a $ G_\delta $ set. Now, note that $ G = \bintrs_n O_n \subseteq O_m $ for any $ m\in \N $. Therefore, $ G \setminus E \subseteq O_m\setminus E $ for any $ m\in \N $. Hence,
		\begin{equation*}
			\begin{split}
				\lom{G\setminus E} &\le \lom{O_m \setminus E}\\
				&= \lom{O_m} - \lom{E}\;\;\;\;\text{(Since $ E\subseteq O_m $)}\\
				&\le \frac{1}{n}
			\end{split}
		\end{equation*} 
 		for all $ n\in \N $. Therefore $ \lom{G\setminus E} = 0 $ which implies that 
 		\[\lom{G} = \lom{E}\]
 		as $ E\subseteq G $. 
%		We have $ A\subseteq \R $. For this, we have 
%		\[\lom{A} = \inf \set{L}_A\]
%		Therefore, for any $ \{(a_i,b_i)\} \in \set{C}_A $, 
%		\[\lom{A} \le \sum_i (b_i-a_i).\]
%		Now, consider,
%		\[B^n = \bunion_i (a_i^n, b_i^n)\;\text{for any }\{(a_i^n,b_i^n)\} \in \set{C}_A\]
%		Now, since $ B^n $ covers $ A $, that is $ A\subseteq B^n$ for all $ n $ so $ B^n $ is open as arbitrary union of open sets is open, therefore, 
%		\[A\subseteq \bintrs_n B^n = C\]
%		where $\bintrs_n B^n = C $ is clearly $ G_\delta $. Now we need to show both have the same outer measure. For this, we first trivially have that $ \set{C}_C \subseteq \set{C}_A $ because $ A\subseteq C $ and any infinite cover of $ C $ would then cover $ A $. But moreover, we also see that any infinite sequence $ \{(a_i^n,b_i^n)\}\in \set{C}_A $ for some $ n $, we have 
%		\[C =\bintrs_nB^n \subseteq B^n = \bunion_i(a_i^n,b_i^n).\]
%		That is, any infinite sequence in $ \set{C}_A $ also covers $ C $. Therefore $ \set{C}_A \subseteq \set{C}_C $. Hence $ \set{C}_A = \set{C}_C $. So that $ \set{L}_A = \set{L}_C $ and hence $ \lom{A} = \lom{C} $.
\end{proof}}
\item {\begin{proof}
		Take any $ B\subseteq \R $. We are given a subset $ A\subseteq \R $ such that $ \lom{A} = 0 $. Note that 
		\[\lom{B}\le \lom{A\union B}\le \lom{A} + \lom{B} = \lom{B}\]
		Hence we already have that $ \lom{B}= \lom{A\union B} $. Now to show that $ \lom{A\union B} = \lom{B- A} $, first note that $ B-A = B\intrs \comp{A} $. Clearly, $ B\intrs \comp{A} \subseteq B $, therefore,
		\[\lom{B\intrs \comp{A}}\le \lom{B} = \lom{A\union B}.\]
		Now, to show the converse to complete the proof, we note that $ A\union B = A \union (B\intrs \comp{A}) $, hence, by countable sub-additivity, we finally have,
		\[\lom{A\union (B\intrs \comp{A})} = \lom{A\union B} = \lom{B} \le \lom{A} + \lom{B\intrs \comp{A}} = \lom{B\intrs \comp{A}}\]
		Hence the proof is complete.
\end{proof}}
\item {\begin{proof}
		Take $ A\subseteq \R $ such that it is countable. Now, $ A $ can either be finite or countably infinite. Let's first take $ A $ to be finite, so that $ \exists n\in \N $ such that
		\[A = \{a_1,a_2\dots,a_n\}.\]
		Now, consider $ \{(a_i,b_i)\} \in \set{C}_A$ such that it covers each point with an exponentially decreasing interval, as follows:
		\[\{I_j\} = \left \{(a_1-\frac{\epsilon}{2},a_1 + \frac{\epsilon}{2} ), \dots, (a_n - \frac{\epsilon}{2},a_n + \frac{\epsilon}{2}), (a_1 -\frac{\epsilon}{2^2},a_1 + \frac{\epsilon}{2^2}),\dots\right \}\]
		In other words,
		\[\{(a_i,b_i)\} = \left \{ \left (a_{i\%n + 1} - \frac{\epsilon}{2^{\ceil{i\% n}}}, a_{i\%n + 1} + \frac{\epsilon}{2^{\ceil{i\% n}}}\right ) \right \}\]
		We clearly have that this sequence $ \{(a_i,b_i)\}\in \set{C}_A $ and since $ \lom{A} = \inf \set{L}_A $, therefore we must have that 
		\[\lom{A} \le \sum_{i} (b_i - a_i) = \sum_i \frac{2\epsilon}{2^{\ceil{i\%n}}}= 4\epsilon = \epsilon^\prime\]
		Now since $ \lambda^* : \pow{\R} \to [0,+\infty] $, therefore, $ \lom{A} = 0 $ as $ \epsilon >0$ is arbitrary. So far, we have proved that finite set has outer measure zero. For countably infinite set, we would have $ A = \{a_1,a_2,\dots\} $. On this, consider the cover $ \{(a_i,b_i)\} = \left \{\left ( a_i - \frac{\epsilon}{2^{i+1}}, a_i + \frac{\epsilon}{2^{i+1}} \right )\right \} $. Clearly, this sequence $ \{(a_i,b_i)\} $ is in $ \set{C}_A $. Moreover,
		\[\lom{A} \le \sum_i (b_i - a_i) = \sum_i \frac{\epsilon}{2^i} = \epsilon\]
		Now since $ \epsilon >0 $ is arbitrary, therefore, $ \lom{A} = 0 $ even for countably infinite set $ A $.\\
		Now since $ \lom{\R} \neq 0 $, therefore $ \R $ is not countable.
\end{proof}}
\item {\begin{proof}
		Take any open interval $ (a,b) $. We can see that 
		\[ (a,b) \subseteq (a,b]  \]
		and
		\[(a,b] \subseteq (a,b) \union \bunion_n (b,b+\epsilon/2^n]\]
		Therefore,
		\begin{equation*}
			\begin{split}
				\lom{(a,b)} &\le \lom{(a,b]}\\
				&= b-a.
			\end{split}
		\end{equation*}
		and
		\begin{equation*}
			\begin{split}
				\lom{(a,b]} =b-a &\le \lom{(a,b) \union \bunion_{n} (b,b+\epsilon/2^n]} \\
				&\le \lom{(a,b)}  \sum_{n} b - (b+\epsilon/2^n)\\
				&= \lom{(a,b)} + \sum_n b-b -\epsilon/2^n\\
				&= \lom{(a,b)} + \sum_n \epsilon/2^n\\
				&= \lom{(a,b)}.
			\end{split}
		\end{equation*}
	Hence, $ l((a,b)) = \lom{(a,b)} = b-a $. Hence, one can take interval $ \{(a_n,b_n)\} $ as the sequence of intervals in the definition of Lebesgue Outer Measure as well, instead of $ \{(a_n,b_n]\} $.
\end{proof}}
\item {\begin{proof}
		Consider $ \set{C}_A = \left \{\{(a_i,b_i)\}\where A \subseteq \bunion_i(a_i,b_i)\right \} $ and $ \set{C}_{\alpha A} = \left \{ \{(c_i,d_i)\}\where \alpha A \subseteq  \bunion_i (c_i,d_i) \right \} $. We will show that 
		\[\text{To Show : }\set{C}_{\alpha A} = \alpha \set{C}_{A} = \left \{\{\alpha(a_i,b_i)\} \where A \subseteq \bunion_i(a_i,b_i)\right \}.\]
		First, take any infinite sequence $ \{(a_i,b_i)\}\in \set{C}_A $. Then $ \{\alpha(a_i,b_i)\} \in \set{C}_{\alpha A} $ because if $ A \subseteq \bunion_i (a_i,b_i) $, then $ \alpha A \subseteq \bunion_i \alpha(a_i,b_i) $ where $ \alpha(a_i,b_i) = (\alpha a_i,\alpha b_i) $ if $ \alpha > 0 $ or $ = (\alpha b_i, \alpha a_i) $ if $ \alpha <0 $. Hence, 
		\[ \alpha \set{C}_A \subseteq \set{C}_{\alpha A}. \]
		To see the converse, first note that the case of $ \alpha = 0 $ is trivial, so we would take $ \alpha \neq 0 $. Now, take $ \{(c_i,d_i)\}\in \set{C}_{\alpha A} $. Hence $ \alpha A \subseteq \bunion_i (c_i,d_i)  $. Now, $ \frac{1}{\alpha} \cdot \alpha A = A \subseteq \bunion_i \frac{1}{\alpha}\left (c_i,d_i\right ) $. Therefore, $ \{\frac{1}{\alpha}(c_i,d_i)\} \in \set{C}_A \implies \{(c_i,d_i)\} \in \alpha \set{C}_A$. Hence $ \set{C}_{\alpha A} \subseteq \alpha \set{C}_A $. Hence $ \set{C}_{\alpha A} = \alpha\set{C}_A $. Hence, we get that $ \set{L}_{\alpha A} = \abs{\alpha}\set{L}_A $ because $ \sum_i l(\alpha I_n) = \sum_{i} \alpha(b_i-a_i)$ if $ \alpha > 0$ or $ = \sum_i \alpha(a_i-b_i) $ if $ \alpha<0 $. Hence we can write $ \alpha\set{L}_A = \{ \sum_i \abs{\alpha}(b_i - a_i)\where \{(a_i,b_i)\} \in \set{C}_A  \} $. Now, $ \inf \set{L}_{\alpha A} = \inf \abs{\alpha} \set{L}_A \implies \lom{\alpha A} = \abs{\alpha} \lom{A}$.
\end{proof}}
\item {\begin{proof}\footnote{This is the Borel-Cantelli Lemma(!)}
		We have that $ B $ is the set of all points of $ \R $ which are present in infinite number of subsets $ A_n $. Now consider $ x\in B $. Then, there exists $ \{n\} $ such that the first element of the sequence $ n_1 $ is such that $ x\in A_{n_1} $, or $ x\in \bunion_{n\ge n_1} A_n $. Then, for any $ x\in B $, we have $ x\in\bintrs_{m=1}^{\infty}  \bunion_{n\ge m} A_n $. Hence,
		\[B\subseteq \bintrs_{m=1}^{\infty}  \bunion_{n\ge m} A_n.\]
		Now, take $ x\in \bintrs_{m=1}^{\infty} \bunion_{n\ge m} A_n $. Then $ x\in \bunion_{n\ge m} A_n $ for all $ m\in \N $. This implies that $ x\in A_{k_m} $ for some $ k_m \ge m $ for all $ m $. This implies that $ x\in A_n $ for infinitely many $ n $. Hence $ x\in B $, so that,
		\[B = \bintrs_{m=1}^{\infty}\bunion_{n\ge m} A_n.\]
		We now show the following:
		\begin{equation}
			\label{E2-8.1}
		\text{To Show : }	\left (\bunion_{n\ge m_0} A_n\right ) \setminus \left (\bintrs_{m=1}^{\infty} \bunion_{n\ge m} A_n\right ) = \bunion_{n\ge m_0} A_n.
		\end{equation}
		for any $ m_0 \in \N $.
%\[\left (\bintrs_{m=1}^{\infty} \bunion_{n\ge m} A_n\right ) = \bunion_n A_n\]
		To show this, we first note that the following is easily deduced:
		\[\left (\bintrs_{m=1}^{\infty} \bunion_{n\ge m} A_n\right ) \subseteq \bunion_n A_n\]
		because for any $ x\in \left (\bintrs_{m=1}^{\infty} \bunion_{n\ge m} A_n\right ) $, $ \exists k \in\N$ such that $ x \in  \bunion_{n\ge k} A_n \subseteq \bunion_{n} A_n$. Now, note the following:
		\begin{equation}
			\label{E2-8.2}
			\begin{split}
				\bunion_{n=1}^{m-1} A_n &\subseteq \comp{\left(\bunion_{n\ge m} A_n\right )}\\
				\bunion_{m=1}^{\infty}\bunion_{n=1}^{m-1} A_n&\subseteq \bunion_{m=1}^{\infty} \comp{\left (\bunion_{n\ge m} A_n\right )}\\
				\bunion_{n} A_n &\subseteq \bunion_{m=1}^{\infty} \comp{\left (\bunion_{n\ge m} A_n\right )}
			\end{split}
		\end{equation} 
	Therefore, if we now expand Eq. \ref{E2-8.1}, we get the following:
	\begin{equation}\label{E2-8.3}
		\begin{split}
			\left (\bunion_{n\ge m_0} A_n\right ) \setminus \left (\bintrs_{m=1}^{\infty} \bunion_{n\ge m} A_n\right ) &= \bunion_{n\ge m_0} A_n \bintrs \comp{\left (\bintrs_{m=1}\bunion_{n\ge m} A_n\right )}\\
			&= \bunion_{n\ge m_0} A_n \bintrs \comp{\left (\bunion_{m=1}^{\infty} \bintrs_{n\ge m}\comp{A_n}\right )}\\
			&= \bunion_{n\ge m_0}A_n \bintrs \left (\bunion_{m=1}^{\infty}\comp{\left (\bunion_{n\ge m}A_n\right )} \right) \\
			&= \bunion_{n\ge m_0}A_n\;\;\;\;\text{(Equation \ref{E2-8.2})}
		\end{split}
	\end{equation}
Therefore, 
\begin{equation*}
	\begin{split}
		\lom{\left (\bunion_{n\ge m_0} A_n\right ) \setminus \left (\bintrs_{m=1}^{\infty} \bunion_{n\ge m} A_n\right )} &= \lom{\bunion_{n\ge m_0} A_n} - \lom{\left (\bintrs_{m=1}^{\infty} \bunion_{n\ge m} A_n\right )}\\
		\lom{\left (\bunion_{n\ge m_0} A_n\right ) \setminus \left (\bintrs_{m=1}^{\infty} \bunion_{n\ge m} A_n\right )} - \lom{\bunion_{n\ge m_0} A_n} &= \lom{\left (\bintrs_{m=1}^{\infty} \bunion_{n\ge m} A_n\right )}\\
		0 &= \lom{\left (\bintrs_{m=1}^{\infty} \bunion_{n\ge m} A_n\right )} = \lom{B} \;\;\;\;\text{(From Eq. \ref{E2-8.3})}
	\end{split}
\end{equation*}
Hence, proved.

 \end{proof}}
\item {\begin{proof} \textbf{This proof is wrong, write the correct solution.} We divide the proof as follows:\\\\
		\textbf{Act 1.} \emph{$ A $ and $ B $ are disjoint.}\\
		Take two subsets $ A,B \subseteq \R $ such that $ d(A,B) >0 $. Now, take any arbitrary pair $ a,b $ such that $ a\in A $ and $ b\in B $. Due to the fact that $ d(A,B) = \inf \{\abs{x-y}\where x\in A, \;y\in B\}$, therefore we have
		\begin{equation*}
			\begin{split}
				\abs{a-b} &\ge \inf\{ \abs{x-y}\where x\in X\;,\;\;y\in Y\}\\
				&>0 \;\;\;\;\;(\text{As per question}).
			\end{split}
		\end{equation*}
	Therefore, for any arbitrary $ a\in A $ and $ b\in B $, we have that $ \abs{a-b} >0 $. This implies that either $ a < b $ or $ a > b $, but NOT $ a = b $ for any pair $ a,b$. This further implies that for any $ a\in A $ there does not exists $ b\in B $ such that $ a=b $. Hence $ A \intrs B  = \Phi$. \\\\
	\textbf{Act 2.} \emph{$ \lambda^* $ for disjoint sets is countably additive.}\\
	Now we show that for any subsets $ A, B\subseteq \R $ satisfying $ A\intrs B =\Phi $, we have
	\[\lom{A\union B} = \lom{A} + \lom{B}.\]
	For this, we first trivially have due to countable sub-additivity of $ \lambda^* $ that $ \lom{A \union B} \le \lom{A} + \lom{B} $. All that remains to be shown is 
	\[\lom{A\union B}\ge \lom{A} + \lom{B}.\]
	For this, first consider the set relation which is trivial to show
	\[A\union B \supseteq A\setminus (A\intrs B) \union B\setminus (A\intrs B) \union (A\intrs B).\]
	as for any $ x\in A\setminus (A\intrs B) \union B\setminus (A\intrs B) \union (A\intrs B) $, either $ x\in A $ or $ x\in B $. Now, due to the fact that $ A\intrs B = \Phi $, we get the following conclusion:
	\begin{equation*}
		\begin{split}
			 B\setminus (A\intrs B) &= B\intrs \comp{(A\intrs B)}\\
			 &= B\intrs (\comp{A} \union \comp{B})\\
			 &= (B\intrs \comp{A}) \union (B\intrs \comp{B})\\
			 &= B \union \Phi \;\;\;\;\;\text{($ B\subseteq \comp{A} $)}\\
			 &= B
		\end{split}
	\end{equation*}
Similarly, $ A\setminus (A\intrs B) = A $. Combining both of these results, we get,
\begin{equation*}
	\begin{split}
		\lom{A\union B} &= \lom{A\setminus (A\intrs B) \union B\setminus (A\intrs B) \union (A\intrs B)}\\
		&\le \lom{A} + \lom{B} + \lom{\Phi}\\
		&= \lom{A} + \lom{B}\;\;\;\;\;\text{($ \lom{\Phi}= 0 $)}
	\end{split}
\end{equation*}
Hence we have that $ \lom{A\union B} = \lom{A} + \lom{B} $ for disjoint sets $ A $ and $B $ of $ \R $. Therefore, for the sets $ A $ and $ B $ satisfying the question's criterion, we have the required result.
 \end{proof}}
\end{enumerate}
\newpage
\subsection{Lebesgue Measurability.}
\begin{definition}
	(\textbf{$ \mu^* $-measurable subset}) Let $ X $ be a set and let $ \mu^* $ be an \emph{outer measure} on $ X $. A subset $ B\subseteq X $ is $ \mu^* $-measurable if: 
	\[\om{A} = \om{A\intrs B} + \om{A\intrs \comp{B}}\]
	holds \textbf{for all} subsets $ A \subseteq X$.
\end{definition}
\hrulefill
\begin{definition}
	(\textbf{Lebesgue Measurable subset of $ \R $}) A subset $ B\subseteq \R $ is called a Lebesgue measurable subset of $ \R $ if \textbf{$ B $ is $ \lambda^* $-measurable.} That is, for \textbf{any} $ A\subseteq \R $, we must have:
	\[\lom{A} = \lom{A\intrs B} + \lom{A\intrs \comp{B}}\]
\end{definition}
\begin{remark}
	Important to note are the following:
	\begin{itemize}
		\item {Due to sub-additivity of $ \mu^* $ and $ A \subseteq (A\intrs B) \union (A\intrs \comp{B}) $, we already have that 
	\[\om{A} \le  \om{A\intrs B} + \om{A\intrs \comp{B}}\]	
for \emph{any} subsets $ A,B\subseteq X $.	
}
\item[$ \bigstar $]{Due to the above fact, \emph{all that remains to be shown to ascertain that $ B\subseteq \R $ is $ \mu^* $-measurable} is to show the following converse:
\[\om{A}\ge \om{A\intrs B} + \om{A\intrs \comp{B}}.\]
\textbf{for all} $ A\subseteq X $.
}
	\end{itemize}
\end{remark}
\hrulefill
\begin{proposition}
	\label{P-10}
	Let $ X $ be a set and let $ \mu^* $ be an outer measure on $ X $. Then each subset $ B\subseteq X $ that satisfies $ \om{B} = 0 $ \textbf{or} that satisfies $ \om{\comp{B}} = 0 $ is $ \mu^* $-measurable.
\end{proposition}
\begin{proof}
	This result actually proves that for subset $ B \subseteq X$ which has zero outer measure under $ \mu^* $, any other subset $ A\subseteq X $ would be such that $ \om{A\intrs B} = 0 $(!) After proving this, and from the remark above, we would just be left to show that if $ \om{B} = 0 $, then $ \om{A} \ge \om{A\intrs B} + \om{A\intrs \comp{B}}$. We show the former here, from which the latter follows naturally.\\\\
	Consider $ B\subseteq X $ such that $ \om{B} = 0 $. It's true that $ A\intrs B \subseteq B $. Now since $ \mu^* $ is an outer measure on $ X $, therefore, we must have $ \om{A\intrs B} \le \om{B} = 0.$ This implies that $ \om{A\intrs B} = 0 $. Now, we would see that the required condition follows naturally from the previous. First, note the following:
	\[A\intrs B \subseteq A\;\text{and}\; A\intrs \comp{B} \subseteq A.\]
	Hence, we can write:
	\[\om{A\intrs B} \le \om{A}\;\text{and}\;\om{A\intrs \comp{B}}\le \om{A}.\]
	 Now if $ \om{B} = 0 $, then $ \om{A\intrs B} = 0 $ and then in the second inequality, we would have:
	 \begin{equation*}
	 	\begin{split}
	 		\om{A\intrs \comp{B}} + \om{A\intrs B} \le \om{A} + 0
	 	\end{split}
	 \end{equation*}
 	Or, if $ \om{\comp{B}}  = 0$, then $ \om{A\intrs \comp{B}}= 0 $ and then in the first inequality, we would have:
 	\begin{equation*}
 		\begin{split}
 			\om{A\intrs B} + \om{A\intrs \comp{B}} \le \om{A} + 0.
 		\end{split}
 	\end{equation*}
 Hence, $ B $ is $ \mu^* $-measurable for any $ B\subseteq X $ which satisfies that either $ \om{B} = 0 $ or $ \om{\comp{B}} = 0 $.
\end{proof}
\hrulefill

\emph{The following theorem is a fundamental fact about outer measures.}
\newpage

\begin{theorem}
	\label{T-1}
	Let $ X $ be a set, let $ \mu^* $ be an outer measure on $ X $ and let $ \set{M}_{\mu^*} $ be the collection of all $ \mu^* $-measurable subsets of $ X $. Then,
	\begin{itemize}
		\item {$ \set{M}_{\mu^*} $ is a $ \sigma $-algebra.}
		\item {The restriction of $ \mu^* $ to $ \set{M}_{\mu^*} $ is a measure on $ \set{M}_{\mu^*} $.}
	\end{itemize}
\end{theorem}
\begin{proof}
	\textbf{Act 1.} \emph{$ \msigm{\mu^*} $ is an algebra.}\\
		First, it is clear that $ X,\Phi\in \msigm{\mu^*} $ from Proposition \ref{P-10}, because $ \om{\Phi} = \om{\comp{X}} = 0 $. Now, if $ B\in \msigm{\mu^*} $, then $ \om{A} = \om{A\intrs B} + \om{A\intrs \comp{B}} \;\forall\;A\subseteq X$. But if we replace $ B $ by $ \comp{B} $ in the above, we would get the same equation, hence $ \comp{B} \in \msigm{\mu^*} $. So $ \msigm{\mu^*} $ is closed under complements. Now, to show closed nature under finite unions, we take any two subsets $ B_1, B_2\in \msigm{\mu^*} $ and show that $ A\union B\in \msigm{\mu^*} $. First we have
		\begin{equation*}
			\begin{split}
				\om{A} &= \om{A\intrs B_1} + \om{A\intrs \comp{B_1}}\\
				&= \om{A\intrs B_2} + \om{A\intrs \comp{B_2}}
			\end{split}
		\end{equation*}
	for any $ A\subseteq X $. Now, we see that from the fact that $ B_1\in \msigm{\mu^*} $,
	\begin{equation*}
		\begin{split}
			\om{A\intrs (B_1\union B_2)} &= \om{A\intrs (B_1\union B_2) \intrs B_1} +\om{A\intrs (B_1\union B_2) \intrs \comp{B_1}} \\
			&= \om{A\intrs B_1} + \om{A\intrs B_2 \intrs \comp{B_1}}
		\end{split}
	\end{equation*}
Similarly, we have from the fact $ B_2\in\msigm{\mu^*} $,
\begin{equation*}
	\begin{split}
		\om{A\intrs \comp{(B_1\union B_2)}} &= \om{A\intrs \comp{(B_1 \union B_2)} \intrs B_2} + \om{A\intrs \comp{(B_1 \union B_2)} \intrs \comp{B_2}}\\
		&= \om{A\intrs \comp{B_1} \intrs \comp{B_2}\intrs B_2} + \om{A\intrs \comp{B_1} \intrs \comp{B_2} \intrs \comp{B_2}}\\
		&= \om{\Phi} + \om{A\intrs \comp{B_1} \intrs \comp{B_2}}\\
		&= \om{A\intrs \comp{(B_1\union B_2)}}
	\end{split}
\end{equation*}
Now, adding the above results yield,
\begin{equation*}
	\begin{split}
		\om{A\intrs \comp{(B_1 \union B_2)}} + \om{A\intrs (B_1 \union B_2)} &= \om{A\intrs \comp{(B_1\union B_2)}} + \om{A\intrs B_1} + \om{A\intrs B_2 \intrs \comp{B_1}}\\
		&= \om{A\intrs \comp{B_1}\intrs \comp{B_2}} + \om{A\intrs \comp{B_1} \intrs B_2} + \om{A\intrs B_1}\\
		&= \om{A\intrs \comp{B_1}} + \om{A\intrs B_1}\\
		&= \om{A}.
	\end{split}
\end{equation*}
Hence, $ B_1\union B_2 $ is $ \mu^* $-measurable, so $ B_1\union B_2\in \msigm{\mu^*} $. Now, we can, for a finite collection of subsets in $ \msigm{\mu^*} $, we can proceed like above, to show that $ \msigm{\mu^*} $ is closed under finite union, hence showing that $ \msigm{\mu^*} $ is an algebra.\\\\
\textbf{Act 2.} \emph{$ \msigm{\mu^*} $ is a $ \sigma $-algebra.}\\
All that is left to show that $ \msigm{\mu^*} $ is a $ \sigma $-algebra is to show that it is closed under countable union. We have already proved closed nature under finite union. We extend it via induction principle. Suppose $ \{B_i\} $ is a sequence of disjoint subsets in $ \msigm{\mu^*} $. For this, we first prove\footnote{But why to prove Eq. \ref{E-2}? The motivation for Eq. \ref{E-2} comes from Part 1. More specifically, notice in the equation where we added $ \om{A\intrs \comp{(B_1 \union B_2)}} $ and $ \om{A\intrs (B_1 \union B_2)} $. Note it's $ 2^\text{nd} $ line, this is the case when $ n=2 $ in Eq. \ref{E-2} combined with the fact that $ B_i $'s are disjoint. Now why to take $ B_i $'s to be disjoint? The reason for this comes from the fact that for any infinite sequence of subsets $ \{A_i\} $, one can construct infinite sequence of disjoint subsets, that is : $ A_1, A_2 \intrs \comp{A_1}, A_3\intrs \comp{(A_1\union A_2)},\dots $ and it's union is again $ \bunion_n A_n $. Hence if we prove that a disjoint infinite sequence is closed under union, then we could prove that any infinite sequence of subsets is closed under union too!} using induction that, for all $ A\subseteq X $ and $ n\in \N $, 
\begin{equation}\label{E-2}
	\text{To Prove : }\om{A} = \sum_{i=1}^n \om{A\intrs B_i} + \om{A\intrs \left (\bintrs_{i=1}^{n} \comp{B_i}\right )}
\end{equation}
For the case when $ n=1 $, we see that it Eq. \ref{E-2} reduces to $ \om{A} = \om{A\intrs B_1} + \om{A\intrs \comp{B_1}} $. But since $ B_i \in \msigm{\mu^*}\;\forall \;i\in \N$, therefore this is trivially true. Now, by the induction principle, we assume that Eq. \ref{E-2} is true uptill $ n $ and then we try to prove it for $ n+1 $ step. For this, since $ B_{n+1} \in \msigm{\mu^*}$ is disjoint to all other $ B_i $'s, we have,
\begin{equation*}
	\begin{split}
		\om{A\intrs \bintrs_{i=1}^n \comp{B_i}} &= \om{\left ( A\intrs \bintrs_{i=1}^n \comp{B_i} \right ) \intrs B_{n+1}} +\om{\left ( A\intrs \bintrs_{i=1}^n \comp{B_i} \right ) \intrs \comp{B_{n+1}}} \\
		&= \om{A\intrs B_{n+1}} + \om{A\intrs \bintrs_{i=1}^{n+1} \comp{B_i}}
	\end{split}
\end{equation*}
where the last line follows from the fact that each $ B_i $ is disjoint to other $ B_j $'s, hence each $ \comp{B_j} $ would contain $ B_i $ and therefore $ B_{n+1}\subseteq \bintrs_{i=1}^n \comp{B_i}  $. Now, substituting the above equation in Eq. \ref{E-2} gives,
\newpage
\begin{equation*}
	\begin{split}
		\om{A} &= \sum_{i=1}^n \om{A\intrs B_i} + \om{A\intrs B_{n+1}} + \om{A\intrs \bintrs_{i=1}^{n+1} \comp{B_i}}\\
		&= \sum_{i=1}^{n+1} \om{A\intrs B_i} + \om{A\intrs \bintrs_{i=1}^{n+1} \comp{B_i}}
	\end{split}
\end{equation*}
Hence, by induction principle, Eq. \ref{E-2} is true for all $ n\in \N $. Hence, now we can write,
\begin{equation*}
	\begin{split}
		\om{A} &\ge \sum_{i=1}^\infty \om{A\intrs B_i} + \om{A\intrs \bintrs_{i=1}^\infty \comp{B_i}}\\
		&=\sum_{i=1}^\infty \om{A\intrs B_i} + \om{A\intrs \comp{\left (\bunion_{i=1}^\infty B_i\right)}}
	\end{split}
\end{equation*}
Now, to prove that $ \bunion_i B_i \in \msigm{\mu^*}$, we need to show
\[\text{To Show : }\om{A} \ge \om{A\intrs \bunion_{i} B_i} + \om{A\intrs \comp{\left (\bunion_i B_i\right )}}\]
This comes from previous result as follows:
\begin{equation}
	\begin{split}
		\om{A} &\ge \sum_{i=1}^\infty \om{A\intrs B_i} + \om{A\intrs \left (\bunion_{i=1}^\infty B_i\right )^c}\\
		&\ge \om{\bunion_{i=1}^\infty (A\intrs B_i)} + \om{A\intrs \left (\bunion_{i=1}^\infty B_i\right )^c} \\
		&= \om{A\intrs \bunion_{i=1}^\infty B_i} + \om{A\intrs \left (\bunion_{i=1}^\infty B_i\right )^c}
	\end{split}
\end{equation}
Therefore, $ \bunion_i B_i  \in  \msigm{\mu^*}$. Now, as the previous footnote mentions, for every infinite sequence $ \{C_i\} $ in $ \msigm{\mu^*} $, we have a disjoint sequence of subsets as $ C_1,C_2\intrs \comp{C_1} , C_3\intrs \comp{C_2} \intrs C_1, ... $. Now, this disjoint sequence is closed under union as we just showed and since union of this disjoint sequence is equal to the union of $ \{C_i\} $, hence $ \bunion_i C_i \in \msigm{\mu^*} $ for any sequence $ \{C_i\} $ in $ \msigm{\mu^*} $. Thus, $ \msigm{\mu^*} $ is a $ \sigma $-algebra.\\\\
\textbf{Act 3.}\emph{ $ \mu^* $ restricted to $ \msigm{\mu^*} $ is a measure.}\\
Consider $ \{B_n\} $ be an infinite sequence of subsets in $ \msigm{\mu^*} $. Now, by finite subadditivity, we trivially have
\[\om{\bunion_i B_i} \le \sum_i \om{B_i}\]
Moreover, from Part 2 and setting $ A = \union_i B_i $, we get:
\begin{equation*}
	\begin{split}
		\om{\bunion_i B_i} &\ge \sum_j\om{\bunion_i B_i \intrs B_j} + \om{\bunion_i B_i \intrs \left (\bunion_i B_i\right )^c}\\
		&= \sum_j \om{B_j} + \om{\Phi}\\
		&= \sum_j \om{B_j}. 
	\end{split}
\end{equation*}
We hence have the complete proof.
\end{proof}
\hrulefill
\newpage
\begin{proposition}\label{P-11}
	Every interval of form $ (-\infty, b] $ is Lebesgue measurable.
\end{proposition}
\begin{proof}
	All we need to show that for $ B= (-\infty, b] $ and for all $ A\subseteq \R $, 
	\[\text{To Show : } \lom{A} \ge \lom{A\intrs B} + \lom{A\intrs \comp{B}}.\]
	Consider $ \{(a_i,b_i)\} \in \set{C}_A $. Therefore $ A \subseteq \bunion_i (a_i,b_i) $. Clearly, this implies that 
	\[\lom{A} \le \sum_i (b_i-a_i).\]
	Equivalently, $ \exists \;\epsilon >0 $ such that,
	\[ \lom{A} + \epsilon \ge \sum_i (b_i-a_i). \]
	Now, it's also clear that $ (a_k,b_k)\intrs B $ and $ (a_k,b_k) \intrs \comp{B}$ are disjoint for any $ k\in\N $. But we know that \textbf{for disjoint $ \lambda^{*} $-measurable sets $ C $ and $ D $, it's true that $ \lom{C\union D} = \lom{C} + \lom{D} $, which can be extended for countable union}\footnote{We need it's proof too.}. Hence, we can write:
	\[(b_k - a_k) = \lom{(a_k,b_k)} = \lom{(a_k,b_k)\intrs B \bunion (a_k,b_k)\intrs \comp{B}} = \lom{(a_k,b_k)\intrs B} + \lom{(a_k,b_k)\intrs \comp{B}}.\]
	Now since $ A\subseteq \bunion_i (a_i,b_i) $, therefore $ A\intrs B \subseteq \bunion_i (a_i,b_i)\intrs B $. Similarly, $ A\intrs \comp{B} \subseteq \bunion_i (a_i,b_i)\intrs \comp{B} $.\\
	This implies that,
	\begin{equation*}
		\begin{split}
			\lom{A\intrs B}&\le \lom{\bunion_i (a_i,b_i)\intrs B}\\
			\lom{A\intrs \comp{B}} &\le \lom{\bunion_i (a_i,b_i)\intrs \comp{B}}
		\end{split}
	\end{equation*}
Adding these equations, we get the desired result:
\begin{equation*}
	\begin{split}
		\lom{A\intrs B} + \lom{A\intrs \comp{B}} &\le \lom{\bunion_i (a_i,b_i)\intrs B} + \lom{\bunion_i (a_i,b_i)\intrs \comp{B}}\\
		&\le \sum_i \lom{(a_i,b_i)\intrs B}+\sum_i \lom{(a_i,b_i)\intrs \comp{B}}\\
		&= \sum_i \lom{(a_i,b_i)}\;\;\;\;\text{(Shown above.)}\\
		&= \sum_i (b_i-a_i)\\
		&\le \lom{A} + \epsilon
	\end{split}
\end{equation*}
which is true for all $ \epsilon >0 $, therefore 
\[\lom{A\intrs B} + \lom{A\intrs \comp{B}} \le \lom{A}.\]
\end{proof}
\hrulefill
\begin{proposition}\label{P-12}
	Every Borel subset of $ \R $ is Lebesgue Measurable.
\end{proposition}
\begin{proof}
	From Proposition \ref{P-11}, we have that the subsets of the form $ (-\infty,b] $ are Lebesgue measurable. Remember that a subset $ B\subseteq \R $ is called borel if $ B\in \bor{\R} $; that is, it is present in the smallest $ \sigma $-algebra generated by open subsets of $ \R $. \\
	Now, since $ (-\infty,b] $ is Lebesgue measurable, therefore $ (-\infty,b] \in \msigm{\lambda^*} $. But since $ \sigma $-algebra $ \bor{\R} $ is closed under complements, complement of open subset is closed and $ (-\infty,b] $ is closed, therefore all intervals of form $ (-\infty,b] $ are also in $ \bor{\R} $. Since $ \bor{\R} $ is the smallest $ \sigma $-algebra containing intervals of form $ (-\infty,b] $, therefore $ \bor{\R} \subseteq \msigm{\lambda^*} $.
\end{proof}
\hrulefill
\newpage
\begin{definition}
	(\textbf{Lebesgue Measure}) The restriction of Lebesgue outer measure on $ \R $ to the collection $ \msigm{\lambda^*} $ of Lebesgue measurable subsets of $ \R $ is called \emph{Lebesgue Measure}. It would be denoted by $ \lambda $. Hence, we would work with the measure space $ (\R,\msigm{\lambda^*},\lambda) $.
\end{definition}
\hrulefill
\subsection{Does $ \lom{E} = 0 $ implies $ E $ is countable?}
We would construct today a set which has measure 0, but not countable(!).
\begin{enumerate}
	\item {Take $ E_0 = [0,1] $.}
	\item {Remove $ [1/3,2/3] $ from $ E_0 $ to form $ E_1 = [0,1/3] \union [2/3,1] $.}
	\item {Proceed in the same way to form $ E_2 = [0,1/9]\union [2/9,3/4] \union [2/3,7/9]\union [8/9,1] $.}
	\item {At $ n^{th} $ step, $ E_n $ contains $ 2^n $ subintervals and each of which is of length $ \frac{1}{3^n} $.}
	\item {We clearly have $ E_0 \supset E_1 \supset E_2 \supset \dots $.}
	\item {Here, note that each $ E_n $ is a closed and compact subset of $ \R $.}
	\item {The set 
\[P = \bintrs_{n=0}^{\infty} E_n \;\;\text{is known as \textbf{Cantor Set}.}\]	
}
\end{enumerate}
\subsubsection{Properties of Cantor Set}
\begin{proposition}
	Lebesgue Measure of Cantor Set is 0.
\end{proposition}
\begin{proof}
	Note that Cantor Set is Lebesgue measurable as it is countable intersection of closed sets, hence it is present in the Borel $ \sigma $-algebra $ \bor{\R} $ and hence is also in $ \msigm{\lambda^*} $ (Proposition \ref{P-12}). Hence, instead of $ \lambda^* $, we can now write $ \lambda $ as $ P\in \msigm{\lambda^*} $. Now, measure of Cantor set $ P $ can be written as:
	\begin{equation*}
		\begin{split}
			\lm{P} &= \lm{\bintrs_n E_n}\\
			&= \limit{n\to\infty}{\lm{E_n}}\;\;\;\;\text{(Proposition \ref{P-7})}\\
			&= \limit{n\to\infty}{\frac{2^n}{3^n}}\\
			&= \limit{n\to\infty}{\frac{1}{1.5^n}}\\
			&= 0.
		\end{split}
	\end{equation*}
\end{proof}
\hrulefill
\begin{proposition}
	Cantor Set is Uncountable(!)
\end{proposition}
\begin{proof}
 We will show that there exists a bijection between Cantor Set and an uncountable set, specifically ternary system. For this, consider the ternary representation of every number in $ [0,1] $. What this means is that every number in $ [0,1] $ can be represented only using the numbers $ 0,1 $ and 2. Hence, one write $ \frac{1}{3} $ as 0.1 and $ \frac{2}{3} $ as 0.2. Now, $ (1/3,2/3) = \comp{E_1} \intrs [0,1] $ is the set that has been removed from the process of creating $ E_1 $ from $ E_0 $. Clearly, every number in this $ \comp{E_1}\intrs [0,1] $ is of the form $ 0.1 \dots $ where $ \dots $ are all combinations of $ 0,1 $ and 2. Therefore, we are now left with the $ E_1 $ that has all the numbers represented as $ 0.0\dots $ or $ 0.2\dots $. \\
 As we saw in the generation of $ E_1 $, the generation of $ E_2 $ from $ E_1 $ would hence involve removing numbers of the forms $ 0.01\dots $ and $ 0.21\dots $. And hence $ E_2 $ would then be the set of numbers whose first two decimal places are restricted to NOT have the digit 1; that is, $ E_2 $ would be of form $ 0.02\dots, 0.00\dots, 0.20\dots, 0.22\dots $.\\
 Continuing like this, we see that $ E_n $ would have in ternary representation, all those numbers whose first $ n $ digits are NOT 1. Hence, for any $ p \in P $, $ p $ would have the ternary representation constructed only from 0 and 2, but NOT 1.\\
 Now, consider the map $ f : P \to [0,1] $ such that $ f(p) $ replaces each occurence of $ 2 $ by 1 in the ternary representation of $ p $. We now show that this map is surjective(!) so that $ P $ has atleast as many elements as $ [0,1] $. To show this, take any $ x\in [0,1] $ in it's ternary form, and replace all $ 1 $ by $ 2 $ and denote it as $ x^\prime $. Clearly, $ x^\prime $ would be in $ P $ as $ x^{\prime} $ has all decimal digits generated by 0 and 2. But $ f(x^{\prime}) $ would be opposite action and would be equal to $ x $. Therefore, we showed that for any $ x\in [0,1],\exists\;x^{\prime} \in P$ such that $ f(x^{\prime}) = x $. Hence $ f $ is surjective. Therefore $ P $ has atleast as many elements as $ [0,1] $. But since $ P\subseteq [0,1] $ therefore $ P $ has atmost as many elements as $ [0,1] $. This dichotomy suggests that
 \[\text{\emph{Cantor Set has as many elements as in $ [0,1] $ (!)}}\]
 But since $ [0,1] $ is uncountable, therefore, $ P $ is uncountable.
\end{proof}
\hrulefill

With this, we conclude that \textbf{for any set $ E\subseteq \R $, if $ \lom{E} = 0 $, then it's NOT necessarily true that $ E $ is countable}.\\
We now see an extremely interesting example of a Non-measurable set.
\subsection{A Non-measurable Set}
\begin{theorem}
	 There is a subset of $ \R $ that is not Lebesgue Measurable\footnote{See \cite{Solovay70} for more information.}.
\end{theorem}
\begin{proof} We construct the proof in the following \emph{Acts}:\\\\
	\textbf{Act 1.} \emph{Equivalence Relation on $ \R $.}\\
	Construct the following relation $ \sim  $ on $ \R $:
	\[x\sim y \equiv x-y \in \Q.\]
	Clearly, $ \sim $ is reflexive as $ x - x = 0$ is rational; it is also symmetric as negative of a rational is also a rational number; and it is also transitive as if $x-y $ and $ y-z $ is rational, then $ x-y + y-z = x-z$ is sum of two rationals, which is also rational. Hence $ \sim $ is an equivalence relation. Therefore $ \sim $ partitions the whole $ \R $ into equivalence classes. Note that each equivalence class of $ x $ would consist elements of the form $ \Q + x$. But since $ \Q $ is dense in $ \R $, therefore $ \Q +x $, that is each equivalence class, is dense in $ \R $. \\
	Now, each equivalence class clearly intersects $ (0,1) $, therefore, inducing the Axiom of Choice on the set of all equivalence classes, we can form a subset $ E\subset (0,1) $ which contains exactly one element from each of the equivalence classes. We will later prove that $ E $ is not Lebesgue Measurable. \\\\
	\textbf{Act 2.} \emph{$ E $ satisfies certain properties.}\\
	Consider the set $ \Q \intrs (-1,1) $. Clearly, this is countable as it's subset of $ \Q $. Then, consider $ \{r_n\} $ to be the enumeration of $ \Q\intrs (-1,1) $. Construct the sequence of subsets $E_n = E + r_n $. We now verify that $ \{E_n\} $ satisfies the following properties:
	\begin{enumerate}
		\item {{The sets $ E_n $ are disjoint.}}
		\item {{$ \bunion_n E_n $ is a subset of the interval $ (-1,2) $.}}
		\item {{The interval $ (0,1) $ is included in $ \bunion_n E_n $.}}
	\end{enumerate}
	\emph{Property 1} : Assume that $ E_n \intrs E_m \neq \Phi$ for some $ n,m\in \N $ such that $ n\neq m $. Then $ \exists e_1,e_2 \in E $ such that $ e_1 + r_{n} = e_2 + r_m $ which means that $ e_1 - e_2 = r_m - r_n \in \Q$. But this cannot happen as $ e_1, e_2 $ are elements of $ E $ and $ E $ contains exactly one element from the equivalence class of $ \sim $ intersected with $ (0,1) $. Therefore $ e_1-e_2 \notin \Q $. Which is a contradiction. Hence $ E_n \intrs E_m = \Phi$ for all $ n,m\in \N $ such that $ n\neq m $.\\
	\emph{Property 2} : Take $ x\in \bunion_n E_n $. This implies that $ x\in E_m $ for some $ m\in \N $. But $ E_m = E+r_m = \{e+r_m \where e\in E\} $. Since $ E\subset (0,1) $ and $ r_m \in \Q\intrs(-1,1) \subset (-1,1) $, therefore $x\in  E+r_m \subseteq (-1,2) $. Hence $ \bunion_n E_n \subseteq (-1,2) $.\\
	\emph{Property 3} : Take any $ x\in (0,1) $. Now take the $ e\in E $ such that $ x\sim e $, or $ x-e \in \Q $. Hence $ x \in \Q + e $. That is $ x= r + e $. But since $ 0<e<1 $ and $ 0<x<1 $, therefore $ r = x-e \in \Q \intrs (-1,1) $. Hence $ x \in E +r $ and if we denote $ r = r_n $ for some $ n\in \N $, we get $ x\in E+r_n = E_n $, therefore $ x\in \bunion_i E_i $. Hence $ (0,1)\subseteq \bunion_i E_i $.\\\\
	\textbf{Act 3.} \emph{$ E $ is Not Lebesgue Measurable.}  \\
	Assume that $ E $ is in-fact Lebesgue Measurable. Now since $ E_n $ are disjoint (Property 1), therefore we can write:
	\[\lm{\bunion_n E_n} = \sum_n \lm{E_n}.\]
	Now, since \textbf{Lebesgue Measure is translation invariant}\footnote{Proof?}, therefore $ \lm{E_n} = \lm{E + r_n} = \lm{E} $. Two cases now arise for $ \lm{\bunion_n E_n} $:
	\begin{enumerate}
		\item {If $ \lm{E} $ = 0 : Then $ \lm{\bunion_n E_n} = 0$. But
			\[\lm{(-1,2)} = 3 \le \lm{\bunion_n E_n}\;\;\;\;\text{(Property 3)}.\]
			Therefore we have a contradiction.}
		\item {If $ \lm{E} \neq 0 $ : Then $ \lm{\bunion_n E_n} = \sum_n \lm{E} = +\infty $. But
	\[\lm{\bunion_n E_n}\le \lm{(-1,2)}=3\;\;\;\;\text{(Property 2).}\]	
We again have a contradiction.	
}
	\end{enumerate}
Hence, the set $ E $ is just not Lebesgue Measurable!
\end{proof}
\hrulefill
\subsection{Regularity}
First consider the following proposition.
\begin{proposition}
	\label{P-15}
	Consider $ E \subseteq \R$. The following statements are equivalent:
	\begin{enumerate}
		\item {$ E $ is Lebesgue measurable.}
		\item {$ \forall \;\epsilon >0$, $ \exists  $ an open set $ O $ such that 
	\[E\subseteq O\;\text{and}\;\lom{O\setminus E} < \epsilon.\]	
	}
\item {$ \exists $ a $ G_\delta $ set $ G $ such that
\[E\subseteq G\;\text{and}\;\lom{G\setminus E} = 0.\]
}
	\end{enumerate}
\end{proposition}
%\begin{proposition}
%	Let $ A $ be a Lebesgue Measurable subset of $ \R $. Then there exist a $ G_\delta $ set $ G $ and an $ F_\sigma $ set $ F $ in $ \bor{\R} $ such that $ G\subseteq A \subseteq F $ and $ \lm{G-F} = 0 $.
%\end{proposition}
\begin{proof}The equivalence of each statement is as follows:\\
	\textbf{1 $ \implies $ 2.} Consider $ E\subseteq  \R$ to be Lebesgue measurable. As Question \ref{E2-2} in Exercise 2 shows, for any $ E\subseteq \R $ and any $ \epsilon > 0 $, there exists open set $ U $ such that $ E\subseteq U $ which satisfies
	\[\lom{U}\le \lom{E} + \epsilon.\]
	Now since $ E\subseteq U $, therefore,
	\begin{equation*}
		\begin{split}
			\lom{U\setminus E} &= \lom{U} - \lom{E}\\
			&\le \epsilon
		\end{split}
	\end{equation*}
\textbf{2 $ \implies $ 3.} Similarly, the Question \ref{E2-3} of Exercise 2 shows that there exists a $ G_\delta $ set $ G $ such that $ E\subseteq G $ which satisfies $ \lom{E} = \lom{G} $. This directly means that $ \lom{G\setminus E} = 0$ because $ E\subseteq G $ so $ \lom{G\setminus E} = \lom{G} -\lom{E} $.\\\\
\textbf{3 $ \implies $ 1.} Since $ G $ is $ G_\delta $ set therefore it is intersection of open sets in $ \R $. Now since any open set in $ \R $ is an union of open intervals (Homework I, \ref{H1-1}) which is Lebesgue Measurable and therefore $ G $ is Lebesgue Measurable. Now, we can write $ E $ as
\[E = G \setminus (G\setminus E)\]
where $ G\setminus E $ is such that (from Statement 3) $ \lom{G\setminus E} = 0 $, therefore, by Proposition \ref{P-10}, $ G\setminus E $ is Lebesgue Measurable. Hence $ E $ is also Lebesgue Measurable.

\end{proof}
\hrulefill

Now, consider the next proposition, which is dual of the above.
\begin{proposition}
	Consider $ E \subseteq \R$. The following statements are equivalent:
	\begin{enumerate}
		\item {$ E $ is Lebesgue measurable.}
		\item {$ \forall \;\epsilon >0$, $ \exists  $ closed set $ C $ such that 
			\[C\subseteq E\;\text{and}\;\lom{E\setminus C} < \epsilon.\]	
		}
		\item {$ \exists $ a $ F_\sigma $ set $ F $ such that
			\[F\subseteq E\;\text{and}\;\lom{E\setminus F} = 0.\]
		}
	\end{enumerate}
\end{proposition}
\begin{proof}
	Implications are as follows:\\
	\textbf{1 $ \implies  $ 2.} Suppose $ E \subseteq \R$ is Lebesgue Measurable. Note that if $ E $ is Lebesgue Measurable (that is $ E \in \msigm{\lambda^*} $), then $ \comp{E} $ is also Lebesgue Measurable as $ \msigm{\lambda^*} $ is a $ \sigma $-algebra (Theorem \ref{T-1}). Hence, using Proposition \ref{P-15} on $ \comp{E} $ gives us an open set $ O $ for all $ \epsilon>0 $ such that $ \comp{E}\subseteq O $ and $ \lom{O\setminus \comp{E}} < \epsilon $. Now let's take it's complement. Therefore, $ C = \comp{O} \subseteq E $ where $ C $ is clearly closed. Now, $ E\setminus \comp{O} = O \setminus\comp{E}$\footnote{It's not difficult to see as for any $ x\in E\setminus \comp{O} $, $ x\in E $ but $x\notin \comp{O}  $. Therefore, $ x\in O $ but $x \notin \comp{E} $, that is $ x\in O\setminus \comp{E}$. Similarly for the converse.}. Now, 
	\begin{equation*}
		\begin{split}
			\lom{E\setminus \comp{O}} &= \lom{O\setminus \comp{E}}\\
			&< \epsilon
		\end{split}
	\end{equation*}
which proves the first implication.\\\\
\textbf{2 $ \implies $ 3.} From Proposition \ref{P-15}, we have that $ \exists$ a $ G_\delta $ set $ G $ such that $ \comp{E} \subseteq G $ and $ \lom{G\setminus \comp{E}} =0 $. Note that the complement of countable intersection of open sets is countable union of closed sets. Therefore, $ F =\comp{G} $ is an $ F_\sigma $ set. Now, $ \comp{G} \subseteq \comp{(\comp{E})} = E $. Now, we know that $ E\setminus \comp{G} = G\setminus \comp{E}$. Therefore, we have the result as follows:
\begin{equation*}
	\begin{split}
		\lom{E\setminus\comp{G}} &= \lom{G\setminus\comp{E}}\\	
		&= 0.
	\end{split}
\end{equation*}
\textbf{3 $ \implies $ 1.} Since $ F $ is an $ F_\sigma $ set, therefore, $ F \in \msigm{\lambda^*}$. Moreover, as Statement 2 show, $ \lom{E\setminus F} = 0 $, thus by Proposition \ref{P-10}, $ E\setminus F \in \msigm{\lambda^*}$. Since,
\[E = F \union (E\setminus F)\]
that is $ E $ is union of two Lebesgue Measurable sets, therefore $ E \in \msigm{\lambda^*} $, completing the proof.
\end{proof}
\hrulefill

\newpage
\section{Measurable Functions}
We now see the definition and basic properties of Measurable Functions, which would later be used to define Lebesgue Integral.
\begin{definition}
	(\textbf{Measurable Function}) Let $ (X,\alg{A}) $ be a measurable space and let $ A \subseteq X$ which is in $ \alg{A} $. The function $ f : A\to [-\infty,+\infty] $, is called a Measurable function if 
	\[\{x \where f(x) > \alpha\}\;\text{for any }\alpha \in \R \text{ is Measurable (belongs in $\alg{A} $).}\]
\end{definition}
\begin{remark}
	Please note that the function $ f $ defined above has a measurable domain.
\end{remark}
\hrulefill
\begin{proposition}
	Let $ (X,\alg{A}) $ be a measurable space and $ A\in \alg{A} $. Let $ f : A\to [-\infty,+\infty]$ be a function. Then, the following statements are equivalent:
	\begin{enumerate}
		\item {$ f $ is a Measurable Function.}
		\item {For all $ \alpha \in \R $, the set $ \{x\where f(x)\ge \alpha\} \in \alg{A}$.}
		\item {For all $ \alpha \in \R $, the set $ \{x\where f(x)< \alpha\} \in \alg{A}$.}
		\item {For all $ \alpha \in \R $, the set $ \{x\where f(x)\le \alpha\} \in \alg{A}$.}
	\end{enumerate}
\end{proposition}
\begin{proof}
	The equivalence is shown as follows:\\
	\textbf{1 $ \implies $ 2.} Since $ f $ is a measurable, therefore for all $ \alpha \in \R $, the set $ \{x\where f(x) > \alpha\} \in \alg{A}$. This means that $ C_{\alpha - \frac{1}{n}} =\left \{x\where f(x) > \alpha - \frac{1}{n} \right \} \in \alg{A}$ for all $ n\in \N $. Now, the following set 
	\[C = \bintrs_{n} C_{\alpha - \frac{1}{n}} = \{x\where f(x) \ge \alpha\}.\]
	is Measurable as $ C \in \alg{A} $ because $ C_{\alpha - \frac{1}{n}} \in \alg{A}$ for any $ n\in \N $, hence the countable intersection would also be in $ \alg{A} $, hence measurable.\\
	\textbf{2 $ \implies $ 3.}
\end{proof}









\newpage
\begin{thebibliography}{9}
	\bibitem{Solovay70} 
	Solovay, R. (1970). A Model of Set-Theory in Which Every Set of Reals is Lebesgue Measurable. Annals of Mathematics, 92(1), second series, 1-56.
\end{thebibliography}
\end{document}
