\documentclass{article}
\usepackage[utf8]{inputenc}

\usepackage{amsfonts, amsmath, amssymb}
\usepackage[english]{babel}
% \usepackage{boisik}
\usepackage{amsthm}
\usepackage{graphicx}
\usepackage{mathrsfs}
\usepackage{centernot}
\usepackage[margin=0.5in]{geometry}
\usepackage{empheq}
\usepackage{calligra}
\usepackage{bm}
\usepackage[]{microtype}
%\usepackage[light,math]{anttor}
%\usepackage[T1]{fontenc}
%\usepackage[default]{fontsetup}
%\usepackage{gfsartemisia-euler}
%\usepackage[T1]{fontenc}

%\usepackage{gfsartemisia}
%\usepackage[T1]{fontenc}
\usepackage{mathpazo}
\usepackage[light]{CormorantGaramond}

\usepackage[alpine]{ifsym}

%\usepackage{tgbonum}
%\usepackage{cmbright}
%\usepackage{textcomp}
\usepackage{sectsty}
\sectionfont{\fontsize{20}{20}\selectfont}
%\titlefont{\fontsize{20}{20}\selectfont}
%\usepackage{mathpazo} 

\usepackage{hyperref} %Uncomment for Hyperlinked Table of Contents.

\hypersetup{
	colorlinks = true,
	citecolor=blue,
	filecolor=black,
	linkcolor=blue,
	urlcolor=blue
}
%\usepackage{cleveref}[nameinlink]

\usepackage[skins,theorems]{tcolorbox}
\tcbset{highlight math style={enhanced,
		colframe=red,colback=white,arc=0pt,boxrule=1pt}}

\newcommand*\widefbox[1]{\fbox{\hspace{2em}#1\hspace{2em}}}

\subsectionfont{\fontsize{15}{15}\selectfont}

\usepackage{xcolor}

\newcommand*{\myfont}{\fontfamily{cmr}\selectfont}
\newcommand{\id}[1]{{\myfont \text{id}_{#1}}}
\newcommand{\notset}[0]{\setminus}



\title{\Huge Measure Theory\\
	\Large Definitions, Propositions, Theorems \& Proofs}
\author{Animesh Renanse}
\date{\today}
\usepackage{amsthm}

\theoremstyle{definition}
\newtheorem{definition}{$\boxed{\star}$ Definition}
\newcommand{\tit}[1]{\textit{#1}}
\newtheorem{theorem}{$\boxed{\boxed{\circledast}}$ Theorem}

\newcommand{\nll}[0]{\newline\newline}
\theoremstyle{remark}
\newtheorem*{remark}{\textbf{Remark}}

\theoremstyle{definition}
\newtheorem{corollary}{$ \to $ Corollary}

\theoremstyle{definition}
\newtheorem{proposition}{$\checkmark$ Proposition}

\theoremstyle{definition}
\newtheorem*{lemma*}{Lemma}

\usepackage{mathtools}
\newcommand{\bb}[1]{\mathbb{#1}}
\newcommand{\theor}[0]{\boxed{\textbf{\textit{Theorem}}}}
\newcommand{\algo}[0]{\boxed{\textbf{\textit{Algorithm}}}}
\newcommand{\assum}[0]{\boxed{\textbf{\textit{Assumption}}}}
\newcommand{\formula}[0]{\boxed{\textbf{\textit{Formula}}}}
\newcommand{\defin}[0]{\boxed{\textbf{\textit{Definition}}}}
\newcommand{\abs}[1]{\left \vert #1\right \vert}
\newcommand{\vect}[1]{\bm{#1}}
\newcommand{\mat}[1]{\bm{\mathrm{#1}}}
\newcommand{\norm}[1]{\left \Vert #1 \right \Vert}
\newcommand{\innerp}[2]{\langle #1, #2 \rangle}
\newcommand{\expec}[2]{\mathbb{E}_{#1}\left[ #2\right]}
\newcommand{\seq}[1]{\left\{#1\right\}}
\newcommand{\pder}[2]{\frac{\partial #1}{\partial #2}}
\newcommand{\der}[2]{\frac{d #1}{d #2}}
\newcommand{\estif}[1]{\hat{\theta}\left( #1 \right)}
\newcommand{\esti}[0]{\hat{\theta}}



\newcommand{\defeq}[1]{
		\tcbhighmath[boxrule=1pt,arc=0pt,colback=yellow!10,colframe=black]{\begin{aligned}
				#1
		\end{aligned}}
	}
\newcommand{\theoreq}[1]{
		\tcbhighmath[boxrule=2pt,arc=0pt,colback=blue!10!white,colframe=black]{\begin{aligned}
				#1
		\end{aligned}}}
\renewcommand{\qedsymbol}{\ensuremath{\blacksquare}}

%%%%%%%%%%%%%%%%%%%%%%%%%%%%%%%%%%%%%%%Commands
\newcommand{\union}{\cup}
\newcommand{\intrs}{\cap}
\newcommand{\bunion}{\bigcup}
\newcommand{\bintrs}{\bigcap}
\newcommand{\interior}[1]{{#1}^{\circ}}
\newcommand{\closure}[1]{\overline{#1}}
\newcommand{\boundary}[1]{\partial #1}
\newcommand{\lsum}[2]{l(#1,\mathcal{#2})}
\newcommand{\usum}[2]{u(#1,\mathcal{#2})}
\newcommand{\where}{\;\vert\;}
\newcommand{\R}{\mathbb{R}}
\newcommand{\N}{\mathbb{N}}
\newcommand{\Q}{\mathbb{Q}}
\newcommand{\Z}{\mathbb{Z}}
\newcommand{\alg}[1]{\mathscr{#1}}
\newcommand{\bor}[1]{\mathscr{B}(#1)}
\newcommand{\comp}[1]{#1^{\text{c}}}
\newcommand{\m}[1]{\mu\left (#1\right )}
\newcommand{\limit}[2]{\underset{#1}{\lim}\; #2}
\newcommand{\pow}[1]{\mathscr{P}\left (#1\right )}
\newcommand{\om}[1]{\mu^*\left ( #1\right )}
\newcommand{\lom}[1]{\lambda^*\left (#1\right )}
\newcommand{\set}[1]{\mathscr{#1}}
\newcommand{\msigm}[1]{\set{M}_{#1}}
\DeclarePairedDelimiter{\ceil}{\lceil}{\rceil}
\newcommand{\lm}[1]{\lambda\left (#1\right )}
\newcommand{\inv}[1]{{#1}^{-1}}
\newcommand{\limitinf}[1]{\underset{#1}{\text{lim\;inf}}}
\newcommand{\limitsup}[1]{\underset{#1}{\text{lim\;sup}}}
\newcommand{\infim}[1]{\underset{#1}{\inf}}
\newcommand{\suprem}[1]{\underset{#1}{\sup}}
\newcommand{\setlow}[2]{\underbrace{\text{#2}}{#1}}
\newcommand{\restrict}[2]{\left . #1 \right \vert_{#2}}
\newcommand{\Lp}[1]{L^{p}\left (#1\right )}
\newcommand{\wnorm}[2]{\left (\int \abs{#2}^{#1}\right )^{\frac{1}{#1}}}
\newcommand{\LS}[2]{L^{#1}\left (#2\right )}
\newcommand{\prem}[1]{\mu_{0}\left (#1\right )}
\newcommand{\opm}[1]{\mu_{*}\left (#1\right)}
\newcommand{\bv}[1]{\mathcal{BV}\left (#1\right )}
\newcommand{\sm}[1]{\nu\left (#1\right )}
\newcommand{\ms}{\perp}
\newcommand{\tvsm}[1]{\abs{\nu}\left (#1\right )}  %Total Variation of a Signed Measure.
\newcommand{\sblf}[1]{\alg{B}\left (#1\right )} %Space of Bounded Linear Functionals. 

%\pagecolor{lightgray}
\begin{document}
\begin{titlepage}
	{\scshape\LARGE Indian Institute of Technology, Guwahati \par}
	\vspace{1cm}
	{\scshape\Large Pre-Final Year : $ 6^{th} $ Semester\par}
	\vspace{1.5cm}
	{\Huge\bfseries Measure Theory\par}
	\vspace{2cm}
	{\Large\itshape Prof. Pratyoosh \textsc{Kumar}\par}
	\vfill
	\emph{Written by}\par
	Animesh Renanse
	
	\vfill
	
	% Bottom of the page
	{\large \today\par}
\end{titlepage}
\tableofcontents
\newpage
\section{Introduction}
Measure Theory is a field of mathematics which deals with formalizing the notion of a \emph{measure} of a given quantity. Traditional notion of a \emph{measure} fails in lots of complicated scenario. Hence, we need a rigorous definition of it. The work in this direction was forwarded by Lebesgue and others in the dawn of $ 20^{th} $ century.
\subsection{Few introductory definitions}
These are few of the basic definitions that one might remember from real analysis.
\begin{itemize}
	\item{\textbf{Limit Points} : $ x\in X $ is called a \emph{limit point} of a subset $ S \subseteq X $ if $\forall\; r > 0 $, $ \exists\; a\neq x$ such that $ a\in S\intrs B_{r}(x) $. That is, ball of any size $ r $ around $ x $ contains atleast one point of $ S $.}
	\item{\textbf{Isolated Points} : $ y\in S $ is called an \emph{isolated point} of a subset $ S\subseteq X $ if $ \exists \;r > 0 $ such that $ (B_r(y)\setminus \{y\}) \intrs S = \Phi $. That is, $ B_r(y) $ contains no other point of $ S $ apart from $ y $. \begin{itemize}
			\item {Also note that every point of closure $ \closure{S} $ is either a limit point or an isolated point of $ S $.}
			\item {More specifically, \emph{any subset of $ \R^d $ is closed if and only if it contains all of it's limit points}.}
		\end{itemize} }
	\item{\textbf{Perfect Set} : $ A $ is called a perfect set if $ A = A^\prime $ where $ A^\prime $ is the set of all \emph{limit points} of $ A $. More conveniently, if $ A $ does not contain any isolated points then it is a perfect set. $ \R $ is a perfect set.}
	\item {\textbf{Symmetric Difference} : $ A $ and $ B $ are two sets then symmetric difference is $ A\Delta B = (A\setminus B) \cup (B\setminus A)$.}
	\item {\textbf{Power Set} : Collection of all subsets of a set $ S $, written as $ P(S) $.}
	\item {\textbf{Lower Bound} : 	A \emph{lower bound} of a subset $ S $ of a poset $ (P,\le) $ is an element $ a \in P$ such that $ a\le x $ for all $ x\in S $.}
	\item {\textbf{Infimum} : A lower bound $ p\in P $ is called an \emph{infimum} of $ S $ if for all lower bounds $ y $ of $ S $ in $ P $, $ y\le p $.}
	\item {\textbf{Limit Infimum} : For a sequence $ \{x_n\} $, limit inferior is defined by:
\begin{equation}\label{liminf}
	\begin{split}
	\limitinf{n\to\infty}\; x_n &= \limit{n\to \infty}{\left ( \infim{m\ge n} x_m \right )}\\
	&= \suprem{n\ge 0} \infim{m\ge n} x_m\\
	 &= \sup \{\inf\{x_m \where m\ge n\}\where n\ge 0\}.
	\end{split}
\end{equation}
}
	\item {\textbf{Upper Bound} : An \emph{upper bound} of a subset $ S $ of a poset $  P$ is an element $ b\in P $ such that $ b\ge x $ for all $ x\in S $.}
	\item {\textbf{Supremum} : An upper bound $ u\in P $ is called a \emph{supremum} of $ S $ if for all upper bounds $ z $ of $ S $ in $ P $, $ z \ge u $.}
	\item {\textbf{Limit Supremum} :  For a sequence $ \{x_n\} $, limit supremum is defined by:
	\begin{equation}\label{limsup}
		\begin{split}
			\limsup_{n\to \infty} x_n &= \limit{n\to \infty}{\left (\sup_{m\ge n} x_m\right )}\\
			&= \inf_{n\ge 0} \sup_{m\ge n} x_m\\
			&= \inf \{\sup\{x_m \where m\ge n\}\where n\ge 0\}
		\end{split}
	\end{equation}}
\item {\textbf{Limit} : Consider the sequence $ \{x_n\} $ in $ [-\infty,+\infty] $, then $ \lim_{n} x_n $ is defined as
	\[\liminf_{n\to \infty} x_n = \limsup_{n\to\infty}x_n := \lim_{n\to \infty} x_n.\]
}
	\item {\textbf{Lower Sum} : $ l(f,\mathcal{P}) $ is the sum of the minimum functional values at the partition. That is,
		\[\lsum{f}{P} = \sum_{i=0}^{n-1} m_i (a_{i+1} - a_i)\]
		where $ m_i = \inf\{f(x)\;\vert\;x\in [a_{i-1},a_i]\}$.}
	\item {\textbf{Upper Sum} : Similarly,
		\[\usum{f}{P} = \sum_{i=0}^{n-1} M_i (a_{i+1} - a_i)\]
		where $ M_i = \sup\{f(x)\where x\in [a_{i-1},a_i] \} $.\\
		Remember that the function is \emph{Riemann Integrable} if $ l(f,\mathcal{P}) = u(f,\mathcal{P}) $.}
	\item{\textbf{Countable Sets} : Note the following,
\begin{enumerate}
	\item{\emph{Cardinality} : Sets $ X $ and $ Y $ have the same cardinality if there exists a bijection from $ X $ to $ Y $.}
	\item{\emph{Finite Set} : A set is finite if it is empty or it has the same cardinality as $ \{1,2,\dots,n\} $ for some $ n\in \N $.}
	\item{\emph{Countably Infinite} : If the set has the same cardinality as $ \N $.}
	\item {\emph{Enumeration} : An enumeration of a countably infinite set $ X $ is a bijection of $ \N $ onto $ X $. That is, an enumeration is an infinite sequence $ \{x_n\} $ such that each of the $ x_i $'s are in $ X $ and each element of $ X $ is $ x_i $ for some $ i $.}
	\item {\emph{Countable} : A set is countable if it is finite or countably infinite. For example, $ \N $ is countable, $ \Q $ is also countable (!), $ \R \setminus \Q $ (irrationals) is not countable, $ \R $ is not countable.}
\end{enumerate}	
\item {\textbf{Totally Bounded} : A subset $ B\subseteq X $ is \emph{totally bounded} when it can be covered by a finite number of $ r $-balls for all $ r>0 $. That is,
\[\forall \; r > 0,\;\exists N\in \N,\;\exists a_1,\dots,a_N\in X\;\text{such that}\;B\subseteq \bunion_{n=1}^N B_{r}(a_n)\]
}
\item{\textbf{Compact Set} : A set $ K $ is said to be \emph{compact} when given any cover of balls of possibly unequal radii, there is a finite sub-collection of them that still covers the set $ K $. That is,
\[K\subseteq \bunion_{i}B_{r_i}(a_i)\implies \exists \;i_1,\dots,i_N,\;\;K \subseteq \bunion_{n=1}^N B_{r_{i_n}}(a_{i_n})\]
Note that \emph{compact metric spaces are totally bounded (!)}. Also, \emph{compact sets are closed}.
}
} 
\end{itemize}
The problem begins with Riemann Integrable functions when we see that functions like Dirichlet function ($ 1 $ on  irrational and $ 0 $ on rational points) can become \emph{measurable} even when the function is not continuous! This motivates the need of a formal notion of a \emph{measure}.
\subsection{Homework - I}
\begin{enumerate}
	\item{Every open set in $ \mathbb{R} $ can be written as disjoint union of open intervals.
\begin{proof}
	\label{H1-1}
	Let $ G\subseteq \R$ be a open subset. Now by definition of an open subset, we have that for any $ x\in G $, there exists atleast one open subset $ U $ such that $ x\in U \subseteq G $. Now consider the following union of all such open subsets of $ x $,
	\[U_x = \bunion_{x\in U\subseteq G} U\]
	It's now easy to see that $ U_x $ is the largest such subset of $ G $, as any other $ V\subseteq G $ such that $ x\in V $ is by definition contained in $ U_x $. Moreover, $ U_x $ is an interval as it is an arbitrary union of open intervals. Now, define the following relation on $ G $:
	\[y\sim x \;\iff\; y\in U_x\]
	Now we clearly have that $ x\in U_x $ (reflexive); for $ y\sim U_x $ we have $ U\subseteq U_x $ such that $ x,y\in U $, hence $ x\in U_y $ (symmetric); for $ x\in U_y $ and $ y\in U_z $, we have that $ x,y,z\in U_y $, since $ z\in U_y \subseteq G $ so $ U_y\subseteq U_z $, so $ x\in U_z $ (transitive). Hence $ \sim $ is an equivalence relation, hence $ \sim $ partitions the set $ G $. Denote the set of all equivalence classes as $ \mathcal{I} $ so we get 
	\[G = \bunion_{I\in \mathcal{I}} I\]
	such that $I_1\intrs I_2 = \Phi$ for any $ I_1,I_2\in \mathcal{I} $. Now note that for any $ I \in \mathcal{I}$ is open because each $ I $ is generated by the relation $ \sim $ such that $ y\sim x $ iff $ y\in U_x $. Hence for any $ z\in I $, we have $ z\in U_x \subseteq G$ where $ U_x $ is open. Therefore, we have $ G = \union_{I\in \mathcal{I}} I $ for disjoint open intervals in $ \mathcal{I} $.
\end{proof}	
}
	\item{Prove that every non-empty perfect subset of $ \mathbb{R} $ (or $ \mathbb{R}^n $) is uncountable. That is, if $ A= A^\prime $ then $ A $ is uncountable.
\begin{proof}
	Take $ A\subseteq \R $ to be a perfect subset. Since $ A $ it is perfect, therefore, it must contain all of it's limit points or, equivalently, contains no isolated points. Clearly, then, $ A $ cannot be finite, but can only be countably infinite or uncountable. If it is uncountable, then the proof is over. If $ A $ is countably infinite, then we can write $ A $ as the following :
	\[A= \{a_1,a_2,\dots\}.\]
	Construct a ball around $ a_{i_1} $ of any radius $ r_1 >0 $. Since $ A $ is perfect, therefore $ \exists $ $ a_{i_2} \in B_{r_{1}} (a_{i_1}) \intrs A = C_1$. Similarly, for some $ r_2>0 $, we have $ a_{i_3} \in B_{r_2}(a_{i_2}) \intrs B_{r_1}(a_{i_1}) \intrs A = C_2$ such that $ a_{i_1}\notin C_2 $ and so on. In general, we would have the following,
	\[a_{i_{n+1}} \in \left (\bintrs_{j=1}^n B_{r_j}(a_{i_j}) \right ) \intrs A = C_n.\]
	Now, consider $ C = \intrs_n C_n $. Since $ C_{n+1} \subseteq C_n $, therefore $ C\neq \Phi $. But, $ a_i\notin C $ for any $ i \in\N $ as $ a_i \notin C_{i+1} $. Therefore we have a contradiction. Hence $ A $ cannot by countably infinite, it must only be uncountable.
\end{proof}	
}
%\item {A subset of R is compact if, and only if, it is
%	closed and bounded.
%\begin{proof}
%	\emph{Do It!}
%\end{proof}
%}	
\item {In the definition of Lebesgue Outer Measure on $ \R $, one can instead take $ \alg{C}_A $ to be collection of infinite sequences of the any form from $ \{[a_n,b_n]\}, \{(a_n,b_n)\} $ or $ \{(a_n,b_n]\} $.
\begin{proof}
	\emph{Refer Proof of Proposition \ref{P-9}}.
\end{proof}
}
\item {Show the following:
\[\bunion_{n=1}^N E_n = \bunion_{n=1}^N \left (E_n \intrs \comp{\left ( \bunion_{k<n} E_k\right )}\right )\]
\begin{proof}
Take $ x\in \bunion_{n=1}^N E_n $. Then $ \exists \; E_k $ for some $a  $ such that $ x\in E_a $. Now, clearly, $ x\in E_a \subseteq \left ( \bunion_{k<a} E_k\right )^c $, hence $ x\in\left (E_a \intrs \comp{\left ( \bunion_{k<a} E_k\right )}\right )  $. Hence, we have $ \bunion_{n=1}^N E_n \subseteq  \bunion_{n=1}^N \left (E_n \intrs \comp{\left ( \bunion_{k<n} E_k\right )}\right ) $. The converse is easy to see too.
\end{proof}
}

\end{enumerate}
\newpage
\subsection{Measure of an Interval}
In the definition of Riemann Integral, we see that a function is Riemann Integrable if $ \usum{f}{P} = \lsum{f}{P} $. But it depends upon the definition of the \emph{length of an interval}.\\
Note that in the traditional notion of \emph{measure}, the number of points has no contribution in length. Hence the question becomes:
\[\text{\emph{How to measure a set $ A $ in $ \R $?}}\]
For this, we define \emph{Outer Measure}.
\subsubsection{Outer Lebesgue Measure}
The concept of this topic began from the following question : If we consider the cauchy sequence of Riemann Integrable functions on $ \R $, will it converge to a Riemann Integrable function?\\
We follow the following notion of measure (atleast for the first part of the course). We do approximation by approximation.\\\\
Suppose $ A\subseteq \R $. We define $ m^*(A) $ as the \emph{Outer Lebesgue Measure} defined by
\begin{equation*}
	\begin{split}
		m^*(A) = \inf \left \{\sum_{i=1}^\infty l(I_i)\where A\subseteq \bunion_{i}I_i\right \}
	\end{split}
\end{equation*}
where $ I_i = [a_i,b_i) $. Since we are taking infimum of the the collection, hence Note that it is approximating measure from outside. Also note that this is different from the definition of measurable sets.\\
Outer measure is a function $ m^* : P(\R) \to [0,\infty] $. 
%\subsubsection{Properties}
%We have the following properties:
%\begin{itemize}
%	\item{$ m^*(\phi) = 0 $.}
%	\item{$ m^*(\{a\}) = 0 $.}
%	\item{}
%\end{itemize}
%\begin{lemma}
%	$ A = I $ then $ m^*(A) = l(I) $ if $ I $ is a closed interval.
%\end{lemma}
\newpage
\section{Measures}
\subsection{Algebras \& Sigma-Algebras}
\begin{definition}
	(\textbf{Algebra/Field}) Let $ X $ be an arbitrary set. A collection $ \alg{A} \subseteq P(X)$ of subsets of $ X $ is an algebra on $ X $ if:
	\begin{itemize}
		\item {$ X\in \alg{A} $.}
		\item {$ A\in \alg{A} \implies A^c \in \alg{A}$.}
		\item {For each finite sequence $ A_1,A_2,\dots, A_n \in \alg{A}$ implies that
	\[\bunion_{i=1}^n A_i \in \alg{A}\]	
	 }
 \item {For each finite sequence $ A_1,A_2,\dots,A_n\in\alg{A} $ implies that
\[\bintrs_{i=1}^nA_i \in\alg{A}\] 
}
	\end{itemize}
\end{definition}
\hrulefill
\begin{definition}
	(\textbf{$ \sigma $-Algebra/$ \sigma $-Field}) Let $ X $ be an arbitrary set. A collection $ \alg{A} \subseteq P(X)$ of subsets of $ X $ is a $ \sigma $-algebra on $ X $ if:
	\begin{itemize}
		\item {$ X\in \alg{A} $.}
		\item {$ A\in\alg{A} \implies A^c\in \alg{A}$.}
		\item {For each infinite sequence $ \{A_i\} $ such that $ A_i\in \alg{A} $, it implies that
	\[\bunion_{i=1}^\infty A_i \in\alg{A}\]	
	}
\item{For each infinite sequence $ \{A_i\} $ such that $ A_i\in \alg{A} $, it implies that
	\[\bintrs_{i=1}^\infty A_i \in\alg{A}\]} 
	\end{itemize}
\end{definition}
\hrulefill
\begin{proposition}
	\label{P-1}
	Let $ X $ be a set. Then the intersection of an arbitrary non-empty collection of $ \sigma $-algebras on $ X $ is a $ \sigma $-algebra on $ X $.
\end{proposition}
\begin{proof}
	Consider a collection $ \alg{C} $ of $ \sigma $-algebras on $ X $. Denote $ \alg{A} = \bintrs\alg{C} $ as intersection of all $ \sigma $-algebras in $ \alg{C}$. We can now easily see that any subset in $ \alg{A} $ would be present in every $ \sigma $-algebra present in collection $ \alg{C} $, hence, it would obey all properties of a $ \sigma $-algebras. Therefore, $ \alg{A} $ is a $ \sigma $-algebra.
\end{proof}
\begin{corollary}
	Let $ X $ be a set and let $ \alg{F}\subseteq P(X) $ be a family of subsets of $ X $. Then there exists a smallest $ \sigma $-algebra on $ X $ that includes $ \alg{F} $.
\end{corollary}
\begin{proof}
	Consider any given family $ \alg{F} \subseteq P(X)$ and just take intersection of the family $ \alg{C} $ of all $ \sigma $-algebras which contains $ \alg{F} $ to construct this smallest $ \sigma $-algebra. 
\end{proof}
\hrulefill
\begin{definition}
	(\textbf{Generated $ \sigma $-algebra}) The smallest $ \sigma $-algebra on $ X $ containing a given family $ \alg{F}  \subseteq P(X) $ of subsets is called the $ \sigma $-algebra \emph{generated} by $ \alg{F} $, denoted as $ \sigma(\alg{F}) $.
\end{definition}
\hrulefill
\begin{definition}
	(\textbf{Borel $ \sigma $-algebra on $ \R^d $}) It is the $ \sigma $-algebra on $ \R^d $ generated by the collection of all open subsets of $ \R^d $, denoted as $ \alg{B}(\R^d) $.
\end{definition}
\hrulefill
\begin{definition}
	(\textbf{Borel Subsets of $ \R^d $}) Any $ A \subseteq \R^d$ is called a Borel subset of $ \R^d $ if $ A\in \bor{\R^d} $.
\end{definition}
\hrulefill
\newpage
\begin{proposition}
	\label{P-2}
	The Borel $ \sigma $-algebra on $ \R $, $ \bor{\R} $, of Borel subsets of $ \R $ is  generated by each of the following collection of sets:
	\begin{enumerate}
		\item {The collection of all closed subsets of $ \R $.}
		\item {The collection of all subintervals of $ \R $ of the form $ (-\infty,b] $.}
		\item {The collection of all subintervals of $ \R $ of the form $ (a,b] $.}
	\end{enumerate}
\end{proposition}
\begin{proof}
	To show all of these, consider the three $\sigma$-algebras $ \alg{A}_1, \alg{A}_2, \alg{A}_3 $ corresponding to conditions 1,2 \& 3 respectively and try to prove $ \alg{A}_3\subseteq \alg{A}_2\subseteq \alg{A}_1 \subseteq \bor{\R}$ together with $ \bor{\R}\subseteq \alg{A}_3 $. The first three inclusions are trivial to see. For the case that $ \bor{\R}\subseteq \alg{A}_3 $, simply note that any open subset can be made by unions of the sets of form $ (a,b] $ and by Homework-I,1, each open set is union of open subsets.
\end{proof}
\hrulefill
\begin{proposition}
	\label{P-3}
	The $ \sigma $-algebra $ \bor{\R^d} $ of Borel subsets of $ \R^d $ is generated by each of the following collections:
	\begin{enumerate}
		\item{The collection of all closed subsets of $ \R^d $.}
		\item {The collection of all closed half-spaces in $ \R^d $ that have the form $ \{(x_1,\dots,x_d)\where x_i\le b\} $ for some index $ i $ and some $ b\in \R $.}
		\item {The collection of all rectangles in $ \R^d $ that have the form 
	\[\{(x_1,\dots,x_d)\where a_i< x_i\le b_i\;\text{for}\;i=1,\dots,d\}\]	
	}
	\end{enumerate}
\end{proposition}
\begin{proof}
	Almost the same as in Proposition \ref{P-2}. $\alg{A}_1 \subseteq \bor{\R^d}$ trivially by definition. $ \alg{A}_2 \subseteq \alg{A}_1 $ as $ \{(x_1,\dots,x_d)\where x_i\le b\} $ is closed itself. $ \alg{A}_3\subseteq \alg{A}_2 $ by the observation that $ \{(x_1,\dots,x_d)\where a_i< x_i\le b_i\} $ is made by the difference of two subsets of the form $ \{(x_1,\dots,x_d)\where x_i\le b_i\} $ and $ \{(x_1,\dots,x_d)\where x_i> a_i\} $, the latter is the complement of a certain subset in $ \alg{A}_2 $, moreover, $ \{(x_1,\dots,x_d)\where a_i< x_i\le b_i\;\text{for}\;i=1,\dots,d\} $ is then constructed by intersection of $ d $ such subsets. Finally, $ \bor{\R^d} \subseteq \alg{A}_3$ can be seen via the fact that open subsets in $ \R^d $ are made by union of rectangles of type 3 and as such, they are called open subsets. 
\end{proof}
\hrulefill

\emph{Conditions for an algebra to become a $ \sigma $-algebra.}
\begin{proposition}
	\label{P-4}
	Let $ X $ be a set and let $ \alg{A} $ be an algebra on $ X $. Then, $ \alg{A} $ is a $ \sigma $-algebra on $ X $ if \textbf{either}
	\begin{itemize}
		\item {$ \alg{A} $ is closed under the formation of \textbf{unions} of \emph{increasing} sequence of sets, or,}
		\item {$ \alg{A} $ is closed under the formation of \textbf{intersections} of \emph{decreasing} sequence of sets.}
	\end{itemize}
\end{proposition}
\begin{proof}
	Take any countably infinite collection of subsets $ A_1,A_2,\dots \in \alg{A}$ where $ \alg{A} $ is an algebra. Due to the definition of an algebra, we have that $ C_n = \bunion_{i = 1}^n A_i \in \alg{A}$ for any $ n\ge 1\in \Z_+ $. Now note that $ C_1\subseteq C_2 \subseteq \dots  $, that is, the sequence $ \{C_n\} $ forms an increasing sequence of sets. Hence, by the requirement of the question, we have that $ \bunion_{i=1}^\infty C_i \in \alg{A} $. But then we also have that $ \bunion_{i = 1}^\infty A_i \subseteq \bunion_{i=1}^\infty C_i \in \alg{A}$. Hence we have the required condition for part 1. For part 2, we can see that $ \comp{C_1} \supseteq \comp{C_2} \supseteq \dots $ is a decreasing sequence of sets. Then we must have, by the requirement of the question, that $ \bintrs_{i=1}^{\infty} \comp{C_i} = \comp{\left (\bunion_{i=1}^\infty C_i\right )} \in \alg{A}$. But then by definition of algebra, we must have $ \bunion_{i=1}^\infty C_i\in \alg{A} $, which already contains the countably infinite union $ \bunion_{i=1}^\infty A_i $.
\end{proof}
\hrulefill
\newpage
\subsection{Measures}
\begin{definition}
	(\textbf{Countably Additive Function}) Let $ X $ be a set and $ \alg{A} $ be a $ \sigma $-algebra on $ X $. Function $ \mu : \alg{A} \longrightarrow [0,+\infty]$ is said to be \emph{countably additive} \textbf{if} it satisfies:
	\[\mu\left (\bunion_{i=1}^\infty A_i\right ) = \sum_{i=1}^\infty \mu(A_i) \]
	for \textbf{each} infinite sequence $ \{A_i\} $ of \textbf{disjoint} sets in $ \alg{A} $.
\end{definition}
\hrulefill
\begin{definition}
	(\textbf{Measure}) A \emph{measure} on $ \alg{A} $ is a function $ \mu: \alg{A} \to [0,+\infty] $ that is \textbf{countably additive} and satisfies:
	\[\mu(\Phi) = 0.\]
\end{definition}
\begin{remark}
	This is sometimes also referred as \emph{countably additive measure} on $ \alg{A} $.
\end{remark}
\hrulefill
\begin{definition}
	We have following definitions to compactly represent above definitions:
	\begin{enumerate}
		\item {(\textbf{\emph{Measure Space}}) If $ X $ is a set, $ \alg{A} $ is a $ \sigma$-algebra on $ X $ and if $ \mu $ is a measure on $ \alg{A} $, then the triple $ (X,\alg{A}, \mu) $ is called a \emph{measure space}.}
		\item {(\textbf{\emph{Measurable Space}}) If $ X $ is a set and $ \alg{A} $ is a $ \sigma $-algebra on $ X $, then the pair $ (X,\alg{A}) $ is called a \emph{measurable space}.}
	\end{enumerate}
\end{definition}
\hrulefill
\begin{proposition}
	\label{P-5}
	Let $ (X,\alg{A},\mu) $ be a measure space and let $ A, B\in \alg{A} $ such that $ A\subseteq B$. Then,
	\begin{itemize}
		\item {We have $ \mu(A) \le \mu(B) $.}
		\item {Additionally, if $ A $ satisfies that $ \mu(A)< +\infty $, then:
	\[\mu(B-A) = \mu(B) - \mu(A).\]	
	}
	\end{itemize}
\end{proposition}
\begin{proof}
	Note that $ A $ and $ B\intrs \comp{A} $ are disjoint sets in the sigma algebra $ \alg{A} $. Hence we can write, by countably additive property of $ \mu $, that:
	\begin{equation*}
		\begin{split}
			\mu(A\union \left (B\intrs \comp{A}\right )) &= \mu(B)\\
			&=  \mu(A) + \mu(B\intrs \comp{A})
		\end{split}
	\end{equation*}
Since $ \mu(B\intrs \comp{A}) \ge 0$, hence $ \mu(A)\le \mu(B) $. Moreover, if $ \mu(A) <\infty $, then we can additionally write $ \mu(B\intrs \comp{A})  = \mu(B) - \mu(A)$.
\end{proof}
\hrulefill
\begin{definition}
	Let $ \mu $ be a measure on a measurable space $ (X,\alg{A}) $. Then,
	\begin{itemize}
		\item {(\textbf{\emph{Finite Measure}}) If $ \mu(X)< +\infty $.}
		\item {(\textbf{\emph{$ \sigma $-Finite Measure}}) If $ X = \bunion_{i}A_i $ where $ A_i \in\alg{A} $ such that $ \mu(A_i)< +\infty $ for all $ i\in \N $.}
	\end{itemize}
\end{definition}
\begin{remark}
	In other words, a subset $ A\in\alg{A} $ is \emph{$ \sigma $-finite} if it is a union of a countable sequence of sets that are in $ \alg{A} $ and are of finite measure under $ \mu $.
\end{remark}
\hrulefill
\subsubsection{Elementary Properties of Measures}
\begin{proposition}
	\label{P-6}
	Let $ (X,\alg{A}, \mu) $ be a measure space. If $ \{A_k\} $ is an \textbf{arbitrary} sequence of sets that belong to $ \alg{A} $, then,
	\[\mu\left (\bunion_{k=1}^\infty A_k\right) \le \sum_{k=1}^\infty \mu(A_k).\]
\end{proposition}
\begin{proof}
	Denote $ B_1 = A_1 $ and $ B_i = A_i \intrs \comp{\left (\bunion_{k=1}^{i-1} A_k \right)} $. Note that $ B_i $ and $ B_j $ are disjoint for distinct $ i $ and $ j $. Since $ \{A_k\}\in \alg{A} $, therefore $ \{B_i\} \in \alg{A}$. Moreover, $ \bunion_{i=1}^{\infty} B_i = \bunion_{k=1}^\infty A_k $ by construction. We then get,
	\begin{equation*}
		\begin{split}
			\mu\left (\bunion_{k=1}^\infty A_k\right ) = \mu\left (\bunion_{i=1}^\infty B_i\right ) &= \sum_{i=1}^\infty \mu(B_i)\\
			&\le \sum_{i=1}^\infty \mu(A_i)\;\;\;\;\text{($ \because \;B_i \subseteq A_i$ by construction.)}
		\end{split}
	\end{equation*}
Hence proved.
\end{proof}
\hrulefill
\newpage
\begin{proposition}
	\label{P-7}
	Let $ (X,\alg{A},\mu) $ be a measure space.
	\begin{itemize}
		\item {If $ \{A_k\} $ is an \textbf{increasing} sequence of sets in $ \alg{A} $, then
	\[\m{\bunion_{k}A_k} = \limit{k}{\m{A_k}}.\]	
	}
\item {If $ \{A_k\} $ is a \textbf{decreasing} sequence of sets in $ \alg{A} $ \textbf{and} if $ \m{A_n} < + \infty$ holds for some $ n $, then
\[\m{\bintrs_k A_k} = \limit{k}{\m{A_k}}.\]
}
	\end{itemize}
\end{proposition}
\begin{proof}
	Consider $ \{A_i\} \in \alg{A} $ be an increasing sequence of subsets of $ X $. Define $ B_1 = A_1 $ and $ B_i = A_i - A_{i-1} = A_i \intrs \comp{A_{i-1}}$. Clearly, $ B_i $ are disjoint for distinct $ i $. Hence, we can now write,
	\[\m{\bunion_i B_i} =\sum_{i=1}^\infty \m{B_i}.\]
	Moreover, $ \bunion_i B_i = \bunion_k A_k $ and $ B_i = \bunion_{j=1}^i A_j $, therefore we can modify the above as
	\[\m{\bunion_k A_k} = \limit{k\to \infty}{\sum_{i=1}^k \m{\bunion_{j=1}^i A_j}}= \limit{k\to\infty }{\m{\bunion_{i=1}^k \bunion_{j=1}^i A_j}} = \limit{k\to \infty }{\m{A_k}}\]
	Now, consider $ \{A_k\} $ to be a decreasing sequence of subsets in $ \alg{A} $. Take $ n\in \N $ such that $ \m{A_n} < + \infty$. Denote $ B_i = A_n - A_{n+i} $. We have each $ B_i $ to be disjoint by construction for any two $ i $'s and $ \{B_i\} $ is increasing. Moreover, $ \bunion_k B_k = A_n - \bintrs_{i}A_{n+i}$. Hence by first part, we have:
	\begin{equation*}
		\begin{split}
			\m{\bunion_{k}B_k} &= \m{A_n - \bintrs_i A_{n+i}}\\
			&= \m{A_n} - \m{\bintrs_i A_{n+i}}\;\;\;\;\text{(Since $ \{A_k\} $ is decreasing)}\\
			&= \limit{k\to \infty}{\m{B_k}} = \limit{k\to \infty}{\m{A_n - A_{n+k}}}\;\;\text{(Part 1.)}\\
			&= \m{A_n} - \limit{k\to\infty}{\m{A_{n+k}}}
		\end{split}
	\end{equation*} 
Hence, we get the second result (from $ 2^{nd} $ and $ 4^{th} $ equality).
\end{proof}
\hrulefill

\emph{The following partial converse of Proposition \ref{P-7} helps in checking whether finitely additive measure is countably additive.}
\begin{proposition}
	\label{P-8}
	Let $ (X,\alg{A}) $ be a measurable space and let $ \mu$ be a \textbf{finitely additive} measure on $ (X,\alg{A}) $. Then $ \mu $ is a countably additive measure if \textbf{either}
	\begin{itemize}
		\item {The following holds for each \textbf{increasing sequence} $ \{A_k\} $ of sets that belong to $ \alg{A} $:
	\[\limit{k}{\m{A_k}} = \m{\bunion_k A_k}\]	
	or,
	}
\item {The following holds for each \textbf{decreasing sequence} $ \{A_k\} $ that belongs to $ \alg{A} $ which \textbf{satisfies} $ \bintrs_k A_k = \Phi $:
\[\limit{k}\m{A_k} = 0.\]
 }
	\end{itemize}
\end{proposition}
\begin{proof}
	Consider any disjoint sequence of sets $ \{A_i\} $ in $ \alg{A} $. Let the first part be true. We can construct an increasing sequence from $ \{A_i\} $ by considering a new sequence $ \{B_i\} $ where $ B_k = \bunion_{j=1}^k A_j  $. Now, finite additivity of $ \mu $ means that $ \m{B_k}  = \m{\bunion_{j=1}^{k} A_j} = \sum_{j=1}^k \m{A_j}$. Also, by the condition of part 1, we have $ \limit{k\to \infty}{\m{B_k}}  = \m{\bunion_{k} B_k} = \m{\bunion_k A_k}$. Hence proved part 1.\\
	For part 2, we simply have for any disjoint sequence $ \{A_i\} $ in $ \alg{A} $, a decreasing sequence $ \{B_k\} $ given as $ B_k = \bunion_{i=k}^\infty A_i $. Again, we can write,
	\begin{equation*}
		\begin{split}
			\m{\bunion_{i=1}^\infty A_i} &= \m{\left (\bunion_{i=1}^k A_i\right ) \bunion \left (\bunion_{j=k+1}^\infty A_j\right )}\\
			&= \m{\left (\bunion_{i=1}^k A_i\right ) \bunion B_{k+1}}\\
			&= \m{\bunion_{i=1}^k A_i} + \m{B_{k+1}}\;\;\text{(Since these two are disjoint.)}\\
			&= \sum_{i=1}^k \m{A_i} + \m{B_{k+1}}   
		\end{split}
	\end{equation*}
	Taking $ \limit{k\to \infty}{} $ both sides then yields us 
	\[\m{\bunion_{i=1}^\infty A_i} = \sum_{i=1}^\infty \m{A_i} + 0 \]
as $ \limit{k\to \infty}{\m{B_k}} =0$ is given in the assumption of part 2.
\end{proof}
\hrulefill
\newpage
\subsection{Outer Measures}
\begin{definition}
	(\textbf{Outer Measure}) Let $ X $ be a set and let $  \pow{X} $ be the collection of all subsets of $ X $. An \emph{outer measure} on $ X $ is a function $ \mu^* : \pow{X} \longrightarrow [0,+\infty] $ such that:
	\begin{itemize}
		\item {For the empty set $ \Phi $,
	\[\om{\Phi} = 0\]	
	}
		\item {If $ A\subseteq B\subseteq X $, then
	\[\om{A}\le \om{B}.\]	
	}
\item {If $ \{A_n\} $ is an infinite sequence of subsets of $ X $, then
\[\om{\bunion_n A_n} \le \sum_n \om{A_n}\]
}
	\end{itemize}
\end{definition}
\hrulefill
\begin{definition}
	(\textbf{Lebesgue Outer Measure on $ \R $}) For each subset $ A\subseteq \R $, let $ \alg{C}_A $ be the \emph{set of all infinite sequences} $ \{(a_i,b_i)\} $ of \emph{bounded open intervals} such that $ A\subseteq \bunion_i (a_i,b_i) $. That is, 
	\[\alg{C}_A = \{\left \{(a_i,b_i)\}\where A\subseteq \union_i (a_i,b_i)\;\text{and}\;a_i,b_i\in \R\right \}\]
	Then, $ \lambda^* : \pow{\R}\longrightarrow [0,+\infty]$ is the Lebesgue outer measure, defined by:
	\begin{equation}
		\lom{A} = \inf \left \{\left .\sum_{i} (b_i - a_i)\;\right \vert\;\{(a_i,b_i)\}\in \alg{C}_A \right \}
	\end{equation} 
\end{definition}
\hrulefill

\emph{To verify that $ \lambda^* $ is indeed an outer measure.}
\begin{proposition}
	\label{P-9}
	Lebesgue outer measure on $ \R $ is an outer measure and it assigns to each subinterval of $ \R $ it's length.
\end{proposition}
\begin{proof}
	Denote $ \set{C}_A = \{\{(a_i,b_i)\}\where A \subseteq \union_i(a_i,b_i)\}$. To show that $ \lambda^* $ is an outer measure, we first need to show that $ \lom{\Phi} = 0 $. For that, consider the set of all infinite sequences $ \{(a_i,b_i)\} \in \alg{C}_\Phi$, that is (trivially) $\Phi\subseteq  \union_{i}(a_i,b_i) $, such that $ \sum_{i}(b_i - a_i) <\epsilon $ for all $ \epsilon>0 $. Then, if we denote $ \alg{L}_A = \left \{\sum_{i} (b_i - a_i)\where \{(a_i,b_i)\}\in \alg{C}_A\right \} $, then $ \inf\alg{L}_\Phi = 0$ as for any lower bound $ l $ of $ \alg{L}_A $, if $ l>0 $ then $ \exists \; \{(a_i,b_i)\}\in  \set{C}_\Phi$ such that $ \sum_i(b_i - a_i) < l $, hence $ l\le 0 $, or $ \inf\set{L}_\Phi = 0$. \\
	Second, we need to show that if $ A\subseteq B \subseteq X $, then $ \lom{A}\le \lom{B} $. For this, consider $ A\subseteq B $. Clearly, we have that $ \set{C}_B \subseteq \set{C}_A $, therefore $ \set{L}_B \subseteq \set{L}_A $ and hence $ \inf \set{L}_B \ge \inf \set{L}_A $.\\
	Third, we need to show that for any infinite sequence $ \{A_n\} $ of subsets of $ X $,
	\[\lom{\bunion_n A_n}\le \sum_{n}\lom{A_n}\]
	For this, consider the Lebesgue outer measure of $ A_n $, that is, $ \lom{A_n} $. We must have, that for any infinite sequence $ \{(a_{n,i}, b_{n,i})\} \in \set{C}_{A_n} $, that 
	\[\sum_{i=1}^\infty (b_{n,i} - a_{n,i}) \ge \lom{A_n}.\]
	Hence, consider that the difference is upper bounded according to $ n $, that is the sequence $ \{(a_{n,i},b_{n,i})\}\in \set{C}_{A_n} $ is such that,
	\[\sum_{i=1}^\infty (b_{n,i} - a_{n,i}) - \lom{A_n} \le \epsilon/2^n.\]
	 Now, we can cover the entire $ \bunion_i A_i $ by the union of the above intervals, that is, 
	 \[\bunion_i A_i \subseteq \bunion_n \bunion_i (a_{n,i},b_{n,i}).\]
	 Now, we know that
	 \begin{equation*}
	 	\begin{split}
	 		\lom{\bunion_iA_i} &= \inf\set{L}_{\union_iA_i}.
	 	\end{split}
	 \end{equation*}
	 But since
	 \[\sum_{n}\sum_i (b_{n,i}- a_{n,i}) \in \set{L}_{\union_i A_i},\]
	 and
	 \[\sum_n \left (\sum_i(b_{n,i}-a_{n,i}) - \lom{A_n}\right ) \le \sum_n \epsilon/2^n\]
	 which is equal to 
	 \[\sum_n \sum_i (b_{n,i}-a_{n,i}) - \sum_n \lom{A_n} \le \epsilon \times 1\]
	 or, 
	 	 \[\sum_n \sum_i (b_{n,i}-a_{n,i}) \le \sum_n \lom{A_n} +\epsilon \]
	 	 and since $ \lom{\bunion_i A_i} = \inf\set{L}_{\bunion_i A_i} $, therefore,
	 	 \[\lom{\bunion_i A_i} \le \sum_n\sum_i (b_{n,i}-a_{n,i}) \le \sum_n\lom{A_n}\]
Hence proved. \\\\
Now, we need to show that $ \lambda^* $ assigns each subinterval it's length.\\
For this first show that $ \lom{[a,b]} \le b-a $. This is easy to show if we take,
\[[a,b] = \bunion_{i} (a_i,b_i)\]
where $ (a_1,b_1) = (a,b) $, $ (a_i,b_i) = (a-\epsilon/2^i,a) $ for all even $ i $ and $ (a_j,b_j) = (b,b+\epsilon/2^j) $ for all odd $ j $. Now,
\begin{equation*}
	\begin{split}
		\sum_i (b_i-a_i) &= (b-a) + \sum_{i=2,4,\dots} \epsilon/2^i + \sum_{i=3,5,\dots} \epsilon/2^i\\
		&= b-a + \sum_{i=1,2,\dots} \epsilon/2^i\\
		&= b-a + \epsilon
	\end{split}
\end{equation*}
therefore $ \lom{[a,b]} = \inf \set{L}_{[a,b]} \le b-a + \epsilon $ for all $ \epsilon>0 $, hence $ \lom{[a,b]} \le b-a $. \\
Now, to show the converse that $ b-a \le\lom{[a,b]} $, we first note that $ [a,b] $ is compact, so for any infinite cover $ \{(a_i,b_i)\} \in \set{C}_{[a,b]} $, there exists a finite subcover $ \{(a_i,b_i)\}_{i=1}^n $ of $ [a,b] $. Now, since $ \lambda^* $ is an outer measure, therefore,
\[  b-a \le \sum_{i=1}^n \lom{(a_i,b_i)}\le \sum_{i=1}^\infty \lom{(a_i,b_i)} \in \set{L}_{[a,b]} \] 
Therefore, $ b-a $ is a lower bound of $ \set{L}_{[a,b]} $ and hence $ b-a \le \inf \set{L}_{[a,b]} = \lom{[a,b]} $.\\
Hence $ \lom{[a,b]} = b-a $.\\
Now since, one can construct subintervals of the form $ (a,b] $ or $ [a,b) $ from the following manner:
\[(a,b] \subseteq (a,b) \bunion \left (\bunion_{n}[b,b+\epsilon/2^n]\right )\]
from which we get that $ \lom{(a,b]} \le b-a  $ and also,
\[[a,b] \subseteq (a,b] \bunion \left ( \bunion_n [a-\epsilon/2^n ,a] \right )\]
which yields $ b-a \le \lom{(a,b]} $. Similarly for $ (-\infty,b] $ to show that $ \lom{(-\infty,b]} = +\infty $. 
\end{proof}
\hrulefill
\newpage
\subsection{Exercises - II}
\begin{enumerate}
	\item{\begin{proof}
			Note that $ \Q  $ is dense, that is $ \closure{\Q} = \R $. Therefore, $ \closure{\Q\intrs [0,1]} = [0,1] $. Now, since \emph{closure of union is union of closures}\footnote{Proof done in diary : \emph{$ 3^{rd} $ March, 2018} - The date of reference.}, then we get,
			\[\Q\intrs [0,1] \subseteq \closure{\Q\intrs [0,1]} = [0,1]\subseteq \bunion_{n=1}^m \closure{(a_n,b_n]} = \bunion_{n=1}^m [a_n,b_n]\]
			Now since $ \lambda^* $ is an outer measure, therefore,
			\[\lom{[0,1]} = 1 \le \lom{\bunion_{n=1}^m [a_n,b_n]} \le \sum_{n=1}^m l(I_n) \]
			because $ [0,1] $ is a subinterval, closure of unions is union of closures and infinite subadditivity implies finite subadditivity. Hence proved.
		\end{proof}
}
\item {\begin{proof}
		\label{E2-2}
		Since $ \lom{A} = \inf \set{L}_A$ where
		\[\set{L}_A = \left \{\sum_{i} (b_i - a_i)\where \{(a_i,b_i)\} \in \set{C}_A \right \}\]
		and 
		\[\set{C}_A = \{\{(a_i,b_i)\} \where A\subseteq \union_n (a_n,b_n) \}.\]
		Therefore, for any $ \epsilon >0 $, and if we write $ \lom{A} +\epsilon $, then $ \exists$ $ \{(a_i,b_i)\} \in \set{C}_A $ such that $ \lom{A} + \epsilon \ge \sum_i (b_i - a_i) \ge \lom{\bunion_i (a_i,b_i)} $. We can clearly see that $ \bunion_i (a_i,b_i) $ is union of open intervals and since arbitrary union of open intervals is open, therefore, we have the union of the sequence $ \{(a_i,b_i)\} $ as the needed open interval.
	\end{proof}
}
\item {\begin{proof}
		\label{E2-3}
		From the previous question 2, we have that for $ A \subseteq \R $ and for any $ \epsilon>0 $, there exists an open subset $ O\supseteq A $ such that $ \lom{O} \le \lom{A} + \epsilon $. Now, consider the sequence $ \epsilon_n = 1/n $ and the corresponding open subsets $ O_n $ containing $ A $. Clearly, $ A \subseteq \bintrs_n O_n $. Since $ G = \bintrs_n O_n$ is countable intersection of open subsets, therefore $ G $ is a $ G_\delta $ set. Now, note that $ G = \bintrs_n O_n \subseteq O_m $ for any $ m\in \N $. Therefore, $ G \setminus E \subseteq O_m\setminus E $ for any $ m\in \N $. Hence,
		\begin{equation*}
			\begin{split}
				\lom{G\setminus E} &\le \lom{O_m \setminus E}\\
				&= \lom{O_m} - \lom{E}\;\;\;\;\text{(Since $ E\subseteq O_m $)}\\
				&\le \frac{1}{n}
			\end{split}
		\end{equation*} 
 		for all $ n\in \N $. Therefore $ \lom{G\setminus E} = 0 $ which implies that 
 		\[\lom{G} = \lom{E}\]
 		as $ E\subseteq G $. 
%		We have $ A\subseteq \R $. For this, we have 
%		\[\lom{A} = \inf \set{L}_A\]
%		Therefore, for any $ \{(a_i,b_i)\} \in \set{C}_A $, 
%		\[\lom{A} \le \sum_i (b_i-a_i).\]
%		Now, consider,
%		\[B^n = \bunion_i (a_i^n, b_i^n)\;\text{for any }\{(a_i^n,b_i^n)\} \in \set{C}_A\]
%		Now, since $ B^n $ covers $ A $, that is $ A\subseteq B^n$ for all $ n $ so $ B^n $ is open as arbitrary union of open sets is open, therefore, 
%		\[A\subseteq \bintrs_n B^n = C\]
%		where $\bintrs_n B^n = C $ is clearly $ G_\delta $. Now we need to show both have the same outer measure. For this, we first trivially have that $ \set{C}_C \subseteq \set{C}_A $ because $ A\subseteq C $ and any infinite cover of $ C $ would then cover $ A $. But moreover, we also see that any infinite sequence $ \{(a_i^n,b_i^n)\}\in \set{C}_A $ for some $ n $, we have 
%		\[C =\bintrs_nB^n \subseteq B^n = \bunion_i(a_i^n,b_i^n).\]
%		That is, any infinite sequence in $ \set{C}_A $ also covers $ C $. Therefore $ \set{C}_A \subseteq \set{C}_C $. Hence $ \set{C}_A = \set{C}_C $. So that $ \set{L}_A = \set{L}_C $ and hence $ \lom{A} = \lom{C} $.
\end{proof}}
\item {\begin{proof}
		Take any $ B\subseteq \R $. We are given a subset $ A\subseteq \R $ such that $ \lom{A} = 0 $. Note that 
		\[\lom{B}\le \lom{A\union B}\le \lom{A} + \lom{B} = \lom{B}\]
		Hence we already have that $ \lom{B}= \lom{A\union B} $. Now to show that $ \lom{A\union B} = \lom{B- A} $, first note that $ B-A = B\intrs \comp{A} $. Clearly, $ B\intrs \comp{A} \subseteq B $, therefore,
		\[\lom{B\intrs \comp{A}}\le \lom{B} = \lom{A\union B}.\]
		Now, to show the converse to complete the proof, we note that $ A\union B = A \union (B\intrs \comp{A}) $, hence, by countable sub-additivity, we finally have,
		\[\lom{A\union (B\intrs \comp{A})} = \lom{A\union B} = \lom{B} \le \lom{A} + \lom{B\intrs \comp{A}} = \lom{B\intrs \comp{A}}\]
		Hence the proof is complete.
\end{proof}}
\item {\begin{proof}
		Take $ A\subseteq \R $ such that it is countable. Now, $ A $ can either be finite or countably infinite. Let's first take $ A $ to be finite, so that $ \exists n\in \N $ such that
		\[A = \{a_1,a_2\dots,a_n\}.\]
		Now, consider $ \{(a_i,b_i)\} \in \set{C}_A$ such that it covers each point with an exponentially decreasing interval, as follows:
		\[\{I_j\} = \left \{(a_1-\frac{\epsilon}{2},a_1 + \frac{\epsilon}{2} ), \dots, (a_n - \frac{\epsilon}{2},a_n + \frac{\epsilon}{2}), (a_1 -\frac{\epsilon}{2^2},a_1 + \frac{\epsilon}{2^2}),\dots\right \}\]
		In other words,
		\[\{(a_i,b_i)\} = \left \{ \left (a_{i\%n + 1} - \frac{\epsilon}{2^{\ceil{i\% n}}}, a_{i\%n + 1} + \frac{\epsilon}{2^{\ceil{i\% n}}}\right ) \right \}\]
		We clearly have that this sequence $ \{(a_i,b_i)\}\in \set{C}_A $ and since $ \lom{A} = \inf \set{L}_A $, therefore we must have that 
		\[\lom{A} \le \sum_{i} (b_i - a_i) = \sum_i \frac{2\epsilon}{2^{\ceil{i\%n}}}= 4\epsilon = \epsilon^\prime\]
		Now since $ \lambda^* : \pow{\R} \to [0,+\infty] $, therefore, $ \lom{A} = 0 $ as $ \epsilon >0$ is arbitrary. So far, we have proved that finite set has outer measure zero. For countably infinite set, we would have $ A = \{a_1,a_2,\dots\} $. On this, consider the cover $ \{(a_i,b_i)\} = \left \{\left ( a_i - \frac{\epsilon}{2^{i+1}}, a_i + \frac{\epsilon}{2^{i+1}} \right )\right \} $. Clearly, this sequence $ \{(a_i,b_i)\} $ is in $ \set{C}_A $. Moreover,
		\[\lom{A} \le \sum_i (b_i - a_i) = \sum_i \frac{\epsilon}{2^i} = \epsilon\]
		Now since $ \epsilon >0 $ is arbitrary, therefore, $ \lom{A} = 0 $ even for countably infinite set $ A $.\\
		Now since $ \lom{\R} \neq 0 $, therefore $ \R $ is not countable.
\end{proof}}
\item {\begin{proof}
		Take any open interval $ (a,b) $. We can see that 
		\[ (a,b) \subseteq (a,b]  \]
		and
		\[(a,b] \subseteq (a,b) \union \bunion_n (b,b+\epsilon/2^n]\]
		Therefore,
		\begin{equation*}
			\begin{split}
				\lom{(a,b)} &\le \lom{(a,b]}\\
				&= b-a.
			\end{split}
		\end{equation*}
		and
		\begin{equation*}
			\begin{split}
				\lom{(a,b]} =b-a &\le \lom{(a,b) \union \bunion_{n} (b,b+\epsilon/2^n]} \\
				&\le \lom{(a,b)}  \sum_{n} b - (b+\epsilon/2^n)\\
				&= \lom{(a,b)} + \sum_n b-b -\epsilon/2^n\\
				&= \lom{(a,b)} + \sum_n \epsilon/2^n\\
				&= \lom{(a,b)}.
			\end{split}
		\end{equation*}
	Hence, $ l((a,b)) = \lom{(a,b)} = b-a $. Hence, one can take interval $ \{(a_n,b_n)\} $ as the sequence of intervals in the definition of Lebesgue Outer Measure as well, instead of $ \{(a_n,b_n]\} $.
\end{proof}}
\item {\begin{proof}
		Consider $ \set{C}_A = \left \{\{(a_i,b_i)\}\where A \subseteq \bunion_i(a_i,b_i)\right \} $ and $ \set{C}_{\alpha A} = \left \{ \{(c_i,d_i)\}\where \alpha A \subseteq  \bunion_i (c_i,d_i) \right \} $. We will show that 
		\[\text{To Show : }\set{C}_{\alpha A} = \alpha \set{C}_{A} = \left \{\{\alpha(a_i,b_i)\} \where A \subseteq \bunion_i(a_i,b_i)\right \}.\]
		First, take any infinite sequence $ \{(a_i,b_i)\}\in \set{C}_A $. Then $ \{\alpha(a_i,b_i)\} \in \set{C}_{\alpha A} $ because if $ A \subseteq \bunion_i (a_i,b_i) $, then $ \alpha A \subseteq \bunion_i \alpha(a_i,b_i) $ where $ \alpha(a_i,b_i) = (\alpha a_i,\alpha b_i) $ if $ \alpha > 0 $ or $ = (\alpha b_i, \alpha a_i) $ if $ \alpha <0 $. Hence, 
		\[ \alpha \set{C}_A \subseteq \set{C}_{\alpha A}. \]
		To see the converse, first note that the case of $ \alpha = 0 $ is trivial, so we would take $ \alpha \neq 0 $. Now, take $ \{(c_i,d_i)\}\in \set{C}_{\alpha A} $. Hence $ \alpha A \subseteq \bunion_i (c_i,d_i)  $. Now, $ \frac{1}{\alpha} \cdot \alpha A = A \subseteq \bunion_i \frac{1}{\alpha}\left (c_i,d_i\right ) $. Therefore, $ \{\frac{1}{\alpha}(c_i,d_i)\} \in \set{C}_A \implies \{(c_i,d_i)\} \in \alpha \set{C}_A$. Hence $ \set{C}_{\alpha A} \subseteq \alpha \set{C}_A $. Hence $ \set{C}_{\alpha A} = \alpha\set{C}_A $. Hence, we get that $ \set{L}_{\alpha A} = \abs{\alpha}\set{L}_A $ because $ \sum_i l(\alpha I_n) = \sum_{i} \alpha(b_i-a_i)$ if $ \alpha > 0$ or $ = \sum_i \alpha(a_i-b_i) $ if $ \alpha<0 $. Hence we can write $ \alpha\set{L}_A = \{ \sum_i \abs{\alpha}(b_i - a_i)\where \{(a_i,b_i)\} \in \set{C}_A  \} $. Now, $ \inf \set{L}_{\alpha A} = \inf \abs{\alpha} \set{L}_A \implies \lom{\alpha A} = \abs{\alpha} \lom{A}$.
\end{proof}}
\item {\begin{proof}\footnote{This is the Borel-Cantelli Lemma(!)}
		We have that $ B $ is the set of all points of $ \R $ which are present in infinite number of subsets $ A_n $. Now consider $ x\in B $. Then, there exists $ \{n\} $ such that the first element of the sequence $ n_1 $ is such that $ x\in A_{n_1} $, or $ x\in \bunion_{n\ge n_1} A_n $. Then, for any $ x\in B $, we have $ x\in\bintrs_{m=1}^{\infty}  \bunion_{n\ge m} A_n $. Hence,
		\[B\subseteq \bintrs_{m=1}^{\infty}  \bunion_{n\ge m} A_n.\]
		Now, take $ x\in \bintrs_{m=1}^{\infty} \bunion_{n\ge m} A_n $. Then $ x\in \bunion_{n\ge m} A_n $ for all $ m\in \N $. This implies that $ x\in A_{k_m} $ for some $ k_m \ge m $ for all $ m $. This implies that $ x\in A_n $ for infinitely many $ n $. Hence $ x\in B $, so that,
		\[B = \bintrs_{m=1}^{\infty}\bunion_{n\ge m} A_n.\]
		We now show the following:
		\begin{equation}
			\label{E2-8.1}
		\text{To Show : }	\left (\bunion_{n\ge m_0} A_n\right ) \setminus \left (\bintrs_{m=1}^{\infty} \bunion_{n\ge m} A_n\right ) = \bunion_{n\ge m_0} A_n.
		\end{equation}
		for any $ m_0 \in \N $.
%\[\left (\bintrs_{m=1}^{\infty} \bunion_{n\ge m} A_n\right ) = \bunion_n A_n\]
		To show this, we first note that the following is easily deduced:
		\[\left (\bintrs_{m=1}^{\infty} \bunion_{n\ge m} A_n\right ) \subseteq \bunion_n A_n\]
		because for any $ x\in \left (\bintrs_{m=1}^{\infty} \bunion_{n\ge m} A_n\right ) $, $ \exists k \in\N$ such that $ x \in  \bunion_{n\ge k} A_n \subseteq \bunion_{n} A_n$. Now, note the following:
		\begin{equation}
			\label{E2-8.2}
			\begin{split}
				\bunion_{n=1}^{m-1} A_n &\subseteq \comp{\left(\bunion_{n\ge m} A_n\right )}\\
				\bunion_{m=1}^{\infty}\bunion_{n=1}^{m-1} A_n&\subseteq \bunion_{m=1}^{\infty} \comp{\left (\bunion_{n\ge m} A_n\right )}\\
				\bunion_{n} A_n &\subseteq \bunion_{m=1}^{\infty} \comp{\left (\bunion_{n\ge m} A_n\right )}
			\end{split}
		\end{equation} 
	Therefore, if we now expand Eq. \ref{E2-8.1}, we get the following:
	\begin{equation}\label{E2-8.3}
		\begin{split}
			\left (\bunion_{n\ge m_0} A_n\right ) \setminus \left (\bintrs_{m=1}^{\infty} \bunion_{n\ge m} A_n\right ) &= \bunion_{n\ge m_0} A_n \bintrs \comp{\left (\bintrs_{m=1}\bunion_{n\ge m} A_n\right )}\\
			&= \bunion_{n\ge m_0} A_n \bintrs \comp{\left (\bunion_{m=1}^{\infty} \bintrs_{n\ge m}\comp{A_n}\right )}\\
			&= \bunion_{n\ge m_0}A_n \bintrs \left (\bunion_{m=1}^{\infty}\comp{\left (\bunion_{n\ge m}A_n\right )} \right) \\
			&= \bunion_{n\ge m_0}A_n\;\;\;\;\text{(Equation \ref{E2-8.2})}
		\end{split}
	\end{equation}
Therefore, 
\begin{equation*}
	\begin{split}
		\lom{\left (\bunion_{n\ge m_0} A_n\right ) \setminus \left (\bintrs_{m=1}^{\infty} \bunion_{n\ge m} A_n\right )} &= \lom{\bunion_{n\ge m_0} A_n} - \lom{\left (\bintrs_{m=1}^{\infty} \bunion_{n\ge m} A_n\right )}\\
		\lom{\left (\bunion_{n\ge m_0} A_n\right ) \setminus \left (\bintrs_{m=1}^{\infty} \bunion_{n\ge m} A_n\right )} - \lom{\bunion_{n\ge m_0} A_n} &= \lom{\left (\bintrs_{m=1}^{\infty} \bunion_{n\ge m} A_n\right )}\\
		0 &= \lom{\left (\bintrs_{m=1}^{\infty} \bunion_{n\ge m} A_n\right )} = \lom{B} \;\;\;\;\text{(From Eq. \ref{E2-8.3})}
	\end{split}
\end{equation*}
Hence, proved.

 \end{proof}}
\item {\begin{proof} \textbf{This proof is wrong, write the correct solution from the \emph{Diary}.} We divide the proof as follows:\\\\
		\textbf{Act 1.} \emph{$ A $ and $ B $ are disjoint.}\\
		Take two subsets $ A,B \subseteq \R $ such that $ d(A,B) >0 $. Now, take any arbitrary pair $ a,b $ such that $ a\in A $ and $ b\in B $. Due to the fact that $ d(A,B) = \inf \{\abs{x-y}\where x\in A, \;y\in B\}$, therefore we have
		\begin{equation*}
			\begin{split}
				\abs{a-b} &\ge \inf\{ \abs{x-y}\where x\in X\;,\;\;y\in Y\}\\
				&>0 \;\;\;\;\;(\text{As per question}).
			\end{split}
		\end{equation*}
	Therefore, for any arbitrary $ a\in A $ and $ b\in B $, we have that $ \abs{a-b} >0 $. This implies that either $ a < b $ or $ a > b $, but NOT $ a = b $ for any pair $ a,b$. This further implies that for any $ a\in A $ there does not exists $ b\in B $ such that $ a=b $. Hence $ A \intrs B  = \Phi$. \\\\
	\textbf{Act 2.} \emph{$ \lambda^* $ for disjoint sets is countably additive.}\\
	Now we show that for any subsets $ A, B\subseteq \R $ satisfying $ A\intrs B =\Phi $, we have
	\[\lom{A\union B} = \lom{A} + \lom{B}.\]
	For this, we first trivially have due to countable sub-additivity of $ \lambda^* $ that $ \lom{A \union B} \le \lom{A} + \lom{B} $. All that remains to be shown is 
	\[\lom{A\union B}\ge \lom{A} + \lom{B}.\]
	For this, first consider the set relation which is trivial to show
	\[A\union B \supseteq A\setminus (A\intrs B) \union B\setminus (A\intrs B) \union (A\intrs B).\]
	as for any $ x\in A\setminus (A\intrs B) \union B\setminus (A\intrs B) \union (A\intrs B) $, either $ x\in A $ or $ x\in B $. Now, due to the fact that $ A\intrs B = \Phi $, we get the following conclusion:
	\begin{equation*}
		\begin{split}
			 B\setminus (A\intrs B) &= B\intrs \comp{(A\intrs B)}\\
			 &= B\intrs (\comp{A} \union \comp{B})\\
			 &= (B\intrs \comp{A}) \union (B\intrs \comp{B})\\
			 &= B \union \Phi \;\;\;\;\;\text{($ B\subseteq \comp{A} $)}\\
			 &= B
		\end{split}
	\end{equation*}
Similarly, $ A\setminus (A\intrs B) = A $. Combining both of these results, we get,
\begin{equation*}
	\begin{split}
		\lom{A\union B} &= \lom{A\setminus (A\intrs B) \union B\setminus (A\intrs B) \union (A\intrs B)}\\
		&\le \lom{A} + \lom{B} + \lom{\Phi}\\
		&= \lom{A} + \lom{B}\;\;\;\;\;\text{($ \lom{\Phi}= 0 $)}
	\end{split}
\end{equation*}
Hence we have that $ \lom{A\union B} = \lom{A} + \lom{B} $ for disjoint sets $ A $ and $B $ of $ \R $. Therefore, for the sets $ A $ and $ B $ satisfying the question's criterion, we have the required result.
 \end{proof}}
\end{enumerate}
\newpage
\subsection{Lebesgue Measurability.}
\begin{definition}
	(\textbf{$ \mu^* $-measurable subset}) Let $ X $ be a set and let $ \mu^* $ be an \emph{outer measure} on $ X $. A subset $ B\subseteq X $ is $ \mu^* $-measurable if: 
	\[\om{A} = \om{A\intrs B} + \om{A\intrs \comp{B}}\]
	holds \textbf{for all} subsets $ A \subseteq X$.
\end{definition}
\hrulefill
\begin{definition}
	(\textbf{Lebesgue Measurable subset of $ \R $}) A subset $ B\subseteq \R $ is called a Lebesgue measurable subset of $ \R $ if \textbf{$ B $ is $ \lambda^* $-measurable.} That is, for \textbf{any} $ A\subseteq \R $, we must have:
	\[\lom{A} = \lom{A\intrs B} + \lom{A\intrs \comp{B}}\]
\end{definition}
\begin{remark}
	Important to note are the following:
	\begin{itemize}
		\item {Due to sub-additivity of $ \mu^* $ and $ A \subseteq (A\intrs B) \union (A\intrs \comp{B}) $, we already have that 
	\[\om{A} \le  \om{A\intrs B} + \om{A\intrs \comp{B}}\]	
for \emph{any} subsets $ A,B\subseteq X $.	
}
\item[$ \bigstar $]{Due to the above fact, \emph{all that remains to be shown to ascertain that $ B\subseteq \R $ is $ \mu^* $-measurable} is to show the following converse:
\[\om{A}\ge \om{A\intrs B} + \om{A\intrs \comp{B}}.\]
\textbf{for all} $ A\subseteq X $.
}
	\end{itemize}
\end{remark}
\hrulefill
\begin{proposition}
	\label{P-10}
	Let $ X $ be a set and let $ \mu^* $ be an outer measure on $ X $. Then each subset $ B\subseteq X $ that satisfies $ \om{B} = 0 $ \textbf{or} that satisfies $ \om{\comp{B}} = 0 $ is $ \mu^* $-measurable.
\end{proposition}
\begin{proof}
	This result actually proves that for subset $ B \subseteq X$ which has zero outer measure under $ \mu^* $, any other subset $ A\subseteq X $ would be such that $ \om{A\intrs B} = 0 $(!) After proving this, and from the remark above, we would just be left to show that if $ \om{B} = 0 $, then $ \om{A} \ge \om{A\intrs B} + \om{A\intrs \comp{B}}$. We show the former here, from which the latter follows naturally.\\\\
	Consider $ B\subseteq X $ such that $ \om{B} = 0 $. It's true that $ A\intrs B \subseteq B $. Now since $ \mu^* $ is an outer measure on $ X $, therefore, we must have $ \om{A\intrs B} \le \om{B} = 0.$ This implies that $ \om{A\intrs B} = 0 $. Now, we would see that the required condition follows naturally from the previous. First, note the following:
	\[A\intrs B \subseteq A\;\text{and}\; A\intrs \comp{B} \subseteq A.\]
	Hence, we can write:
	\[\om{A\intrs B} \le \om{A}\;\text{and}\;\om{A\intrs \comp{B}}\le \om{A}.\]
	 Now if $ \om{B} = 0 $, then $ \om{A\intrs B} = 0 $ and then in the second inequality, we would have:
	 \begin{equation*}
	 	\begin{split}
	 		\om{A\intrs \comp{B}} + \om{A\intrs B} \le \om{A} + 0
	 	\end{split}
	 \end{equation*}
 	Or, if $ \om{\comp{B}}  = 0$, then $ \om{A\intrs \comp{B}}= 0 $ and then in the first inequality, we would have:
 	\begin{equation*}
 		\begin{split}
 			\om{A\intrs B} + \om{A\intrs \comp{B}} \le \om{A} + 0.
 		\end{split}
 	\end{equation*}
 Hence, $ B $ is $ \mu^* $-measurable for any $ B\subseteq X $ which satisfies that either $ \om{B} = 0 $ or $ \om{\comp{B}} = 0 $.
\end{proof}
\hrulefill

\emph{The following theorem is a fundamental fact about outer measures.}
\newpage

\begin{theorem}
	\label{T-1}
	Let $ X $ be a set, let $ \mu^* $ be an outer measure on $ X $ and let $ \set{M}_{\mu^*} $ be the collection of all $ \mu^* $-measurable subsets of $ X $. Then,
	\begin{itemize}
		\item {$ \set{M}_{\mu^*} $ is a $ \sigma $-algebra.}
		\item {The restriction of $ \mu^* $ to $ \set{M}_{\mu^*} $ is a measure on $ \set{M}_{\mu^*} $.}
	\end{itemize}
\end{theorem}
\begin{proof}
	\textbf{Act 1.} \emph{$ \msigm{\mu^*} $ is an algebra.}\\
		First, it is clear that $ X,\Phi\in \msigm{\mu^*} $ from Proposition \ref{P-10}, because $ \om{\Phi} = \om{\comp{X}} = 0 $. Now, if $ B\in \msigm{\mu^*} $, then $ \om{A} = \om{A\intrs B} + \om{A\intrs \comp{B}} \;\forall\;A\subseteq X$. But if we replace $ B $ by $ \comp{B} $ in the above, we would get the same equation, hence $ \comp{B} \in \msigm{\mu^*} $. So $ \msigm{\mu^*} $ is closed under complements. Now, to show closed nature under finite unions, we take any two subsets $ B_1, B_2\in \msigm{\mu^*} $ and show that $ A\union B\in \msigm{\mu^*} $. First we have
		\begin{equation*}
			\begin{split}
				\om{A} &= \om{A\intrs B_1} + \om{A\intrs \comp{B_1}}\\
				&= \om{A\intrs B_2} + \om{A\intrs \comp{B_2}}
			\end{split}
		\end{equation*}
	for any $ A\subseteq X $. Now, we see that from the fact that $ B_1\in \msigm{\mu^*} $,
	\begin{equation*}
		\begin{split}
			\om{A\intrs (B_1\union B_2)} &= \om{A\intrs (B_1\union B_2) \intrs B_1} +\om{A\intrs (B_1\union B_2) \intrs \comp{B_1}} \\
			&= \om{A\intrs B_1} + \om{A\intrs B_2 \intrs \comp{B_1}}
		\end{split}
	\end{equation*}
Similarly, we have from the fact $ B_2\in\msigm{\mu^*} $,
\begin{equation*}
	\begin{split}
		\om{A\intrs \comp{(B_1\union B_2)}} &= \om{A\intrs \comp{(B_1 \union B_2)} \intrs B_2} + \om{A\intrs \comp{(B_1 \union B_2)} \intrs \comp{B_2}}\\
		&= \om{A\intrs \comp{B_1} \intrs \comp{B_2}\intrs B_2} + \om{A\intrs \comp{B_1} \intrs \comp{B_2} \intrs \comp{B_2}}\\
		&= \om{\Phi} + \om{A\intrs \comp{B_1} \intrs \comp{B_2}}\\
		&= \om{A\intrs \comp{(B_1\union B_2)}}
	\end{split}
\end{equation*}
Now, adding the above results yield,
\begin{equation*}
	\begin{split}
		\om{A\intrs \comp{(B_1 \union B_2)}} + \om{A\intrs (B_1 \union B_2)} &= \om{A\intrs \comp{(B_1\union B_2)}} + \om{A\intrs B_1} + \om{A\intrs B_2 \intrs \comp{B_1}}\\
		&= \om{A\intrs \comp{B_1}\intrs \comp{B_2}} + \om{A\intrs \comp{B_1} \intrs B_2} + \om{A\intrs B_1}\\
		&= \om{A\intrs \comp{B_1}} + \om{A\intrs B_1}\\
		&= \om{A}.
	\end{split}
\end{equation*}
Hence, $ B_1\union B_2 $ is $ \mu^* $-measurable, so $ B_1\union B_2\in \msigm{\mu^*} $. Now, we can, for a finite collection of subsets in $ \msigm{\mu^*} $, we can proceed like above, to show that $ \msigm{\mu^*} $ is closed under finite union, hence showing that $ \msigm{\mu^*} $ is an algebra.\\\\
\textbf{Act 2.} \emph{$ \msigm{\mu^*} $ is a $ \sigma $-algebra.}\\
All that is left to show that $ \msigm{\mu^*} $ is a $ \sigma $-algebra is to show that it is closed under countable union. We have already proved closed nature under finite union. We extend it via induction principle. Suppose $ \{B_i\} $ is a sequence of disjoint subsets in $ \msigm{\mu^*} $. For this, we first prove\footnote{But why to prove Eq. \ref{E-2}? The motivation for Eq. \ref{E-2} comes from Part 1. More specifically, notice in the equation where we added $ \om{A\intrs \comp{(B_1 \union B_2)}} $ and $ \om{A\intrs (B_1 \union B_2)} $. Note it's $ 2^\text{nd} $ line, this is the case when $ n=2 $ in Eq. \ref{E-2} combined with the fact that $ B_i $'s are disjoint. Now why to take $ B_i $'s to be disjoint? The reason for this comes from the fact that for any infinite sequence of subsets $ \{A_i\} $, one can construct infinite sequence of disjoint subsets, that is : $ A_1, A_2 \intrs \comp{A_1}, A_3\intrs \comp{(A_1\union A_2)},\dots $ and it's union is again $ \bunion_n A_n $. Hence if we prove that a disjoint infinite sequence is closed under union, then we could prove that any infinite sequence of subsets is closed under union too!} using induction that, for all $ A\subseteq X $ and $ n\in \N $, 
\begin{equation}\label{E-2}
	\text{To Prove : }\om{A} = \sum_{i=1}^n \om{A\intrs B_i} + \om{A\intrs \left (\bintrs_{i=1}^{n} \comp{B_i}\right )}
\end{equation}
For the case when $ n=1 $, we see that it Eq. \ref{E-2} reduces to $ \om{A} = \om{A\intrs B_1} + \om{A\intrs \comp{B_1}} $. But since $ B_i \in \msigm{\mu^*}\;\forall \;i\in \N$, therefore this is trivially true. Now, by the induction principle, we assume that Eq. \ref{E-2} is true uptill $ n $ and then we try to prove it for $ n+1 $ step. For this, since $ B_{n+1} \in \msigm{\mu^*}$ is disjoint to all other $ B_i $'s, we have,
\begin{equation*}
	\begin{split}
		\om{A\intrs \bintrs_{i=1}^n \comp{B_i}} &= \om{\left ( A\intrs \bintrs_{i=1}^n \comp{B_i} \right ) \intrs B_{n+1}} +\om{\left ( A\intrs \bintrs_{i=1}^n \comp{B_i} \right ) \intrs \comp{B_{n+1}}} \\
		&= \om{A\intrs B_{n+1}} + \om{A\intrs \bintrs_{i=1}^{n+1} \comp{B_i}}
	\end{split}
\end{equation*}
where the last line follows from the fact that each $ B_i $ is disjoint to other $ B_j $'s, hence each $ \comp{B_j} $ would contain $ B_i $ and therefore $ B_{n+1}\subseteq \bintrs_{i=1}^n \comp{B_i}  $. Now, substituting the above equation in Eq. \ref{E-2} gives,
\newpage
\begin{equation*}
	\begin{split}
		\om{A} &= \sum_{i=1}^n \om{A\intrs B_i} + \om{A\intrs B_{n+1}} + \om{A\intrs \bintrs_{i=1}^{n+1} \comp{B_i}}\\
		&= \sum_{i=1}^{n+1} \om{A\intrs B_i} + \om{A\intrs \bintrs_{i=1}^{n+1} \comp{B_i}}
	\end{split}
\end{equation*}
Hence, by induction principle, Eq. \ref{E-2} is true for all $ n\in \N $. Hence, now we can write,
\begin{equation*}
	\begin{split}
		\om{A} &\ge \sum_{i=1}^\infty \om{A\intrs B_i} + \om{A\intrs \bintrs_{i=1}^\infty \comp{B_i}}\\
		&=\sum_{i=1}^\infty \om{A\intrs B_i} + \om{A\intrs \comp{\left (\bunion_{i=1}^\infty B_i\right)}}
	\end{split}
\end{equation*}
Now, to prove that $ \bunion_i B_i \in \msigm{\mu^*}$, we need to show
\[\text{To Show : }\om{A} \ge \om{A\intrs \bunion_{i} B_i} + \om{A\intrs \comp{\left (\bunion_i B_i\right )}}\]
This comes from previous result as follows:
\begin{equation}
	\begin{split}
		\om{A} &\ge \sum_{i=1}^\infty \om{A\intrs B_i} + \om{A\intrs \left (\bunion_{i=1}^\infty B_i\right )^c}\\
		&\ge \om{\bunion_{i=1}^\infty (A\intrs B_i)} + \om{A\intrs \left (\bunion_{i=1}^\infty B_i\right )^c} \\
		&= \om{A\intrs \bunion_{i=1}^\infty B_i} + \om{A\intrs \left (\bunion_{i=1}^\infty B_i\right )^c}
	\end{split}
\end{equation}
Therefore, $ \bunion_i B_i  \in  \msigm{\mu^*}$. Now, as the previous footnote mentions, for every infinite sequence $ \{C_i\} $ in $ \msigm{\mu^*} $, we have a disjoint sequence of subsets as $ C_1,C_2\intrs \comp{C_1} , C_3\intrs \comp{C_2} \intrs C_1, ... $. Now, this disjoint sequence is closed under union as we just showed and since union of this disjoint sequence is equal to the union of $ \{C_i\} $, hence $ \bunion_i C_i \in \msigm{\mu^*} $ for any sequence $ \{C_i\} $ in $ \msigm{\mu^*} $. Thus, $ \msigm{\mu^*} $ is a $ \sigma $-algebra.\\\\
\textbf{Act 3.}\emph{ $ \mu^* $ restricted to $ \msigm{\mu^*} $ is a measure.}\\
Consider $ \{B_n\} $ be an infinite sequence of subsets in $ \msigm{\mu^*} $. Now, by finite subadditivity, we trivially have
\[\om{\bunion_i B_i} \le \sum_i \om{B_i}\]
Moreover, from Part 2 and setting $ A = \union_i B_i $, we get:
\begin{equation*}
	\begin{split}
		\om{\bunion_i B_i} &\ge \sum_j\om{\bunion_i B_i \intrs B_j} + \om{\bunion_i B_i \intrs \left (\bunion_i B_i\right )^c}\\
		&= \sum_j \om{B_j} + \om{\Phi}\\
		&= \sum_j \om{B_j}. 
	\end{split}
\end{equation*}
We hence have the complete proof.
\end{proof}
\hrulefill
\newpage
\begin{proposition}\label{P-11}
	Every interval of form $ (-\infty, b] $ is Lebesgue measurable.
\end{proposition}
\begin{proof}
	All we need to show that for $ B= (-\infty, b] $ and for all $ A\subseteq \R $, 
	\[\text{To Show : } \lom{A} \ge \lom{A\intrs B} + \lom{A\intrs \comp{B}}.\]
	Consider $ \{(a_i,b_i)\} \in \set{C}_A $. Therefore $ A \subseteq \bunion_i (a_i,b_i) $. Clearly, this implies that 
	\[\lom{A} \le \sum_i (b_i-a_i).\]
	Equivalently, $ \exists \;\epsilon >0 $ such that,
	\[ \lom{A} + \epsilon \ge \sum_i (b_i-a_i). \]
	Now, it's also clear that $ (a_k,b_k)\intrs B $ and $ (a_k,b_k) \intrs \comp{B}$ are disjoint for any $ k\in\N $. But we know that \textbf{for disjoint $ \lambda^{*} $-measurable sets $ C $ and $ D $, it's true that $ \lom{C\union D} = \lom{C} + \lom{D} $, which can be extended for countable union}\footnote{We need it's proof too.}. Hence, we can write:
	\[(b_k - a_k) = \lom{(a_k,b_k)} = \lom{(a_k,b_k)\intrs B \bunion (a_k,b_k)\intrs \comp{B}} = \lom{(a_k,b_k)\intrs B} + \lom{(a_k,b_k)\intrs \comp{B}}.\]
	Now since $ A\subseteq \bunion_i (a_i,b_i) $, therefore $ A\intrs B \subseteq \bunion_i (a_i,b_i)\intrs B $. Similarly, $ A\intrs \comp{B} \subseteq \bunion_i (a_i,b_i)\intrs \comp{B} $.\\
	This implies that,
	\begin{equation*}
		\begin{split}
			\lom{A\intrs B}&\le \lom{\bunion_i (a_i,b_i)\intrs B}\\
			\lom{A\intrs \comp{B}} &\le \lom{\bunion_i (a_i,b_i)\intrs \comp{B}}
		\end{split}
	\end{equation*}
Adding these equations, we get the desired result:
\begin{equation*}
	\begin{split}
		\lom{A\intrs B} + \lom{A\intrs \comp{B}} &\le \lom{\bunion_i (a_i,b_i)\intrs B} + \lom{\bunion_i (a_i,b_i)\intrs \comp{B}}\\
		&\le \sum_i \lom{(a_i,b_i)\intrs B}+\sum_i \lom{(a_i,b_i)\intrs \comp{B}}\\
		&= \sum_i \lom{(a_i,b_i)}\;\;\;\;\text{(Shown above.)}\\
		&= \sum_i (b_i-a_i)\\
		&\le \lom{A} + \epsilon
	\end{split}
\end{equation*}
which is true for all $ \epsilon >0 $, therefore 
\[\lom{A\intrs B} + \lom{A\intrs \comp{B}} \le \lom{A}.\]
\end{proof}
\hrulefill
\begin{proposition}\label{P-12}
	Every Borel subset of $ \R $ is Lebesgue Measurable.
\end{proposition}
\begin{proof}
	From Proposition \ref{P-11}, we have that the subsets of the form $ (-\infty,b] $ are Lebesgue measurable. Remember that a subset $ B\subseteq \R $ is called borel if $ B\in \bor{\R} $; that is, it is present in the smallest $ \sigma $-algebra generated by open subsets of $ \R $. \\
	Now, since $ (-\infty,b] $ is Lebesgue measurable, therefore $ (-\infty,b] \in \msigm{\lambda^*} $. But since $ \sigma $-algebra $ \bor{\R} $ is closed under complements, complement of open subset is closed and $ (-\infty,b] $ is closed, therefore all intervals of form $ (-\infty,b] $ are also in $ \bor{\R} $. Since $ \bor{\R} $ is the smallest $ \sigma $-algebra containing intervals of form $ (-\infty,b] $, therefore $ \bor{\R} \subseteq \msigm{\lambda^*} $.
\end{proof}
\hrulefill
\newpage
\begin{definition}
	(\textbf{Lebesgue Measure}) The restriction of Lebesgue outer measure on $ \R $ to the collection $ \msigm{\lambda^*} $ of Lebesgue measurable subsets of $ \R $ is called \emph{Lebesgue Measure}. It would be denoted by $ \lambda $. Hence, we would work with the measure space $ (\R,\msigm{\lambda^*},\lambda) $\footnote{From this point on-wards, whenever this text mentions that a given set is measurable in space $ (X,\alg{A},\mu) $, it must be assumed that the given set is in $ \alg{A} $, given that there is no ambiguity.}.
\end{definition}
\hrulefill
\subsection{Does $ \lom{E} = 0 $ implies $ E $ is countable?}
We would construct today a set which has measure 0, but not countable(!).
\begin{enumerate}
	\item {Take $ E_0 = [0,1] $.}
	\item {Remove $ (1/3,2/3) $ from $ E_0 $ to form $ E_1 = [0,1/3] \union [2/3,1] $.}
	\item {Proceed in the same way to form $ E_2 = [0,1/9]\union [2/9,1/3] \union [2/3,7/9]\union [8/9,1] $.}
	\item {At $ n^{th} $ step, $ E_n $ contains $ 2^n $ subintervals and each of which is of length $ \frac{1}{3^n} $.}
	\item {We clearly have $ E_0 \supset E_1 \supset E_2 \supset \dots $.}
	\item {Here, note that each $ E_n $ is a closed and compact subset of $ \R $.}
	\item {The set 
\[P = \bintrs_{n=0}^{\infty} E_n \;\;\text{is known as \textbf{Cantor Set}.}\]	
}
\end{enumerate}
\subsubsection{Properties of Cantor Set}
\begin{proposition}
	Lebesgue Measure of Cantor Set is 0.
\end{proposition}
\begin{proof}
	Note that Cantor Set is Lebesgue measurable as it is countable intersection of closed sets, hence it is present in the Borel $ \sigma $-algebra $ \bor{\R} $ and hence is also in $ \msigm{\lambda^*} $ (Proposition \ref{P-12}). Hence, instead of $ \lambda^* $, we can now write $ \lambda $ as $ P\in \msigm{\lambda^*} $. Now, measure of Cantor set $ P $ can be written as:
	\begin{equation*}
		\begin{split}
			\lm{P} &= \lm{\bintrs_n E_n}\\
			&= \limit{n\to\infty}{\lm{E_n}}\;\;\;\;\text{(Proposition \ref{P-7})}\\
			&= \limit{n\to\infty}{\frac{2^n}{3^n}}\\
			&= \limit{n\to\infty}{\frac{1}{1.5^n}}\\
			&= 0.
		\end{split}
	\end{equation*}
\end{proof}
\hrulefill
\begin{proposition}
	Cantor Set is Uncountable(!)
\end{proposition}
\begin{proof}
 We will show that there exists a bijection between Cantor Set and an uncountable set, specifically ternary system. For this, consider the ternary representation of every number in $ [0,1] $. What this means is that every number in $ [0,1] $ can be represented only using the numbers $ 0,1 $ and 2. Hence, one write $ \frac{1}{3} $ as 0.1 and $ \frac{2}{3} $ as 0.2. Now, $ (1/3,2/3) = \comp{E_1} \intrs [0,1] $ is the set that has been removed from the process of creating $ E_1 $ from $ E_0 $. Clearly, every number in this $ \comp{E_1}\intrs [0,1] $ is of the form $ 0.1 \dots $ where $ \dots $ are all combinations of $ 0,1 $ and 2. Therefore, we are now left with the $ E_1 $ that has all the numbers represented as $ 0.0\dots $ or $ 0.2\dots $. \\
 As we saw in the generation of $ E_1 $, the generation of $ E_2 $ from $ E_1 $ would hence involve removing numbers of the forms $ 0.01\dots $ and $ 0.21\dots $. And hence $ E_2 $ would then be the set of numbers whose first two decimal places are restricted to NOT have the digit 1; that is, $ E_2 $ would be of form $ 0.02\dots, 0.00\dots, 0.20\dots, 0.22\dots $.\\
 Continuing like this, we see that $ E_n $ would have in ternary representation, all those numbers whose first $ n $ digits are NOT 1. Hence, for any $ p \in P $, $ p $ would have the ternary representation constructed only from 0 and 2, but NOT 1.\\
 Now, consider the map $ f : P \to [0,1] $ such that $ f(p) $ replaces each occurence of $ 2 $ by 1 in the ternary representation of $ p $. We now show that this map is surjective(!) so that $ P $ has atleast as many elements as $ [0,1] $. To show this, take any $ x\in [0,1] $ in it's ternary form, and replace all $ 1 $ by $ 2 $ and denote it as $ x^\prime $. Clearly, $ x^\prime $ would be in $ P $ as $ x^{\prime} $ has all decimal digits generated by 0 and 2. But $ f(x^{\prime}) $ would be opposite action and would be equal to $ x $. Therefore, we showed that for any $ x\in [0,1],\exists\;x^{\prime} \in P$ such that $ f(x^{\prime}) = x $. Hence $ f $ is surjective. Therefore $ P $ has atleast as many elements as $ [0,1] $. But since $ P\subseteq [0,1] $ therefore $ P $ has atmost as many elements as $ [0,1] $. This dichotomy suggests that
 \[\text{\emph{Cantor Set has as many elements as in $ [0,1] $ (!)}}\]
 But since $ [0,1] $ is uncountable, therefore, $ P $ is uncountable.
\end{proof}
\hrulefill

With this, we conclude that \textbf{for any set $ E\subseteq \R $, if $ \lom{E} = 0 $, then it's NOT necessarily true that $ E $ is countable}.\\
We now see an extremely interesting example of a Non-measurable set.
\subsection{A Non-measurable Set}
\begin{theorem}
	 There is a subset of $ \R $ that is not Lebesgue Measurable\footnote{See \cite{Solovay70} for more information.}.
\end{theorem}
\begin{proof} We construct the proof in the following \emph{Acts}:\\\\
	\textbf{Act 1.} \emph{Equivalence Relation on $ \R $.}\\
	Construct the following relation $ \sim  $ on $ \R $:
	\[x\sim y \equiv x-y \in \Q.\]
	Clearly, $ \sim $ is reflexive as $ x - x = 0$ is rational; it is also symmetric as negative of a rational is also a rational number; and it is also transitive as if $x-y $ and $ y-z $ is rational, then $ x-y + y-z = x-z$ is sum of two rationals, which is also rational. Hence $ \sim $ is an equivalence relation. Therefore $ \sim $ partitions the whole $ \R $ into equivalence classes. Note that each equivalence class of $ x $ would consist elements of the form $ \Q + x$. But since $ \Q $ is dense in $ \R $, therefore $ \Q +x $, that is each equivalence class, is dense in $ \R $. \\
	Now, each equivalence class clearly intersects $ (0,1) $, therefore, inducing the Axiom of Choice on the set of all equivalence classes, we can form a subset $ E\subset (0,1) $ which contains exactly one element from each of the equivalence classes. We will later prove that $ E $ is not Lebesgue Measurable. \\\\
	\textbf{Act 2.} \emph{$ E $ satisfies certain properties.}\\
	Consider the set $ \Q \intrs (-1,1) $. Clearly, this is countable as it's subset of $ \Q $. Then, consider $ \{r_n\} $ to be the enumeration of $ \Q\intrs (-1,1) $. Construct the sequence of subsets $E_n = E + r_n $. We now verify that $ \{E_n\} $ satisfies the following properties:
	\begin{enumerate}
		\item {{The sets $ E_n $ are disjoint.}}
		\item {{$ \bunion_n E_n $ is a subset of the interval $ (-1,2) $.}}
		\item {{The interval $ (0,1) $ is included in $ \bunion_n E_n $.}}
	\end{enumerate}
	\emph{Property 1} : Assume that $ E_n \intrs E_m \neq \Phi$ for some $ n,m\in \N $ such that $ n\neq m $. Then $ \exists e_1,e_2 \in E $ such that $ e_1 + r_{n} = e_2 + r_m $ which means that $ e_1 - e_2 = r_m - r_n \in \Q$. But this cannot happen as $ e_1, e_2 $ are elements of $ E $ and $ E $ contains exactly one element from the equivalence class of $ \sim $ intersected with $ (0,1) $. Therefore $ e_1-e_2 \notin \Q $. Which is a contradiction. Hence $ E_n \intrs E_m = \Phi$ for all $ n,m\in \N $ such that $ n\neq m $.\\
	\emph{Property 2} : Take $ x\in \bunion_n E_n $. This implies that $ x\in E_m $ for some $ m\in \N $. But $ E_m = E+r_m = \{e+r_m \where e\in E\} $. Since $ E\subset (0,1) $ and $ r_m \in \Q\intrs(-1,1) \subset (-1,1) $, therefore $x\in  E+r_m \subseteq (-1,2) $. Hence $ \bunion_n E_n \subseteq (-1,2) $.\\
	\emph{Property 3} : Take any $ x\in (0,1) $. Now take the $ e\in E $ such that $ x\sim e $, or $ x-e \in \Q $. Hence $ x \in \Q + e $. That is $ x= r + e $. But since $ 0<e<1 $ and $ 0<x<1 $, therefore $ r = x-e \in \Q \intrs (-1,1) $. Hence $ x \in E +r $ and if we denote $ r = r_n $ for some $ n\in \N $, we get $ x\in E+r_n = E_n $, therefore $ x\in \bunion_i E_i $. Hence $ (0,1)\subseteq \bunion_i E_i $.\\\\
	\textbf{Act 3.} \emph{$ E $ is Not Lebesgue Measurable.}  \\
	Assume that $ E $ is in-fact Lebesgue Measurable. Now since $ E_n $ are disjoint (Property 1), therefore we can write:
	\[\lm{\bunion_n E_n} = \sum_n \lm{E_n}.\]
	Now, since \textbf{Lebesgue Measure is translation invariant}\footnote{Proof?}, therefore $ \lm{E_n} = \lm{E + r_n} = \lm{E} $. Two cases now arise for $ \lm{\bunion_n E_n} $:
	\begin{enumerate}
		\item {If $ \lm{E} $ = 0 : Then $ \lm{\bunion_n E_n} = 0$. But
			\[\lm{(-1,2)} = 3 \le \lm{\bunion_n E_n}\;\;\;\;\text{(Property 3)}.\]
			Therefore we have a contradiction.}
		\item {If $ \lm{E} \neq 0 $ : Then $ \lm{\bunion_n E_n} = \sum_n \lm{E} = +\infty $. But
	\[\lm{\bunion_n E_n}\le \lm{(-1,2)}=3\;\;\;\;\text{(Property 2).}\]	
We again have a contradiction.	
}
	\end{enumerate}
Hence, the set $ E $ is just not Lebesgue Measurable!
\end{proof}
\hrulefill
\subsection{Regularity}
First consider the following proposition.
\begin{proposition}
	\label{P-15}
	Consider $ E \subseteq \R$. The following statements are equivalent:
	\begin{enumerate}
		\item {$ E $ is Lebesgue measurable.}
		\item {$ \forall \;\epsilon >0$, $ \exists  $ an open set $ O $ such that 
	\[E\subseteq O\;\text{and}\;\lom{O\setminus E} < \epsilon.\]	
	}
\item {$ \exists $ a $ G_\delta $ set $ G $ such that
\[E\subseteq G\;\text{and}\;\lom{G\setminus E} = 0.\]
}
	\end{enumerate}
\end{proposition}
%\begin{proposition}
%	Let $ A $ be a Lebesgue Measurable subset of $ \R $. Then there exist a $ G_\delta $ set $ G $ and an $ F_\sigma $ set $ F $ in $ \bor{\R} $ such that $ G\subseteq A \subseteq F $ and $ \lm{G-F} = 0 $.
%\end{proposition}
\begin{proof}The equivalence of each statement is as follows:\\
	\textbf{1 $ \implies $ 2.} Consider $ E\subseteq  \R$ to be Lebesgue measurable. As Question \ref{E2-2} in Exercise 2 shows, for any $ E\subseteq \R $ and any $ \epsilon > 0 $, there exists open set $ U $ such that $ E\subseteq U $ which satisfies
	\[\lom{U}\le \lom{E} + \epsilon.\]
	Now since $ E\subseteq U $, therefore,
	\begin{equation*}
		\begin{split}
			\lom{U\setminus E} &= \lom{U} - \lom{E}\\
			&\le \epsilon
		\end{split}
	\end{equation*}
\textbf{2 $ \implies $ 3.} Similarly, the Question \ref{E2-3} of Exercise 2 shows that there exists a $ G_\delta $ set $ G $ such that $ E\subseteq G $ which satisfies $ \lom{E} = \lom{G} $. This directly means that $ \lom{G\setminus E} = 0$ because $ E\subseteq G $ so $ \lom{G\setminus E} = \lom{G} -\lom{E} $.\\\\
\textbf{3 $ \implies $ 1.} Since $ G $ is $ G_\delta $ set therefore it is intersection of open sets in $ \R $. Now since any open set in $ \R $ is an union of open intervals (Homework I, \ref{H1-1}) which is Lebesgue Measurable and therefore $ G $ is Lebesgue Measurable. Now, we can write $ E $ as
\[E = G \setminus (G\setminus E)\]
where $ G\setminus E $ is such that (from Statement 3) $ \lom{G\setminus E} = 0 $, therefore, by Proposition \ref{P-10}, $ G\setminus E $ is Lebesgue Measurable. Hence $ E $ is also Lebesgue Measurable.

\end{proof}
\hrulefill

Now, consider the next proposition, which is dual of the above.
\begin{proposition}\label{P-16}
	Consider $ E \subseteq \R$. The following statements are equivalent:
	\begin{enumerate}
		\item {$ E $ is Lebesgue measurable.}
		\item {$ \forall \;\epsilon >0$, $ \exists  $ closed set $ C $ such that 
			\[C\subseteq E\;\text{and}\;\lom{E\setminus C} < \epsilon.\]	
		}
		\item {$ \exists $ a $ F_\sigma $ set $ F $ such that
			\[F\subseteq E\;\text{and}\;\lom{E\setminus F} = 0.\]
		}
	\end{enumerate}
\end{proposition}
\begin{proof}
	Implications are as follows:\\
	\textbf{1 $ \implies  $ 2.} Suppose $ E \subseteq \R$ is Lebesgue Measurable. Note that if $ E $ is Lebesgue Measurable (that is $ E \in \msigm{\lambda^*} $), then $ \comp{E} $ is also Lebesgue Measurable as $ \msigm{\lambda^*} $ is a $ \sigma $-algebra (Theorem \ref{T-1}). Hence, using Proposition \ref{P-15} on $ \comp{E} $ gives us an open set $ O $ for all $ \epsilon>0 $ such that $ \comp{E}\subseteq O $ and $ \lom{O\setminus \comp{E}} < \epsilon $. Now let's take it's complement. Therefore, $ C = \comp{O} \subseteq E $ where $ C $ is clearly closed. Now, $ E\setminus \comp{O} = O \setminus\comp{E}$\footnote{It's not difficult to see as for any $ x\in E\setminus \comp{O} $, $ x\in E $ but $x\notin \comp{O}  $. Therefore, $ x\in O $ but $x \notin \comp{E} $, that is $ x\in O\setminus \comp{E}$. Similarly for the converse.}. Now, 
	\begin{equation*}
		\begin{split}
			\lom{E\setminus \comp{O}} &= \lom{O\setminus \comp{E}}\\
			&< \epsilon
		\end{split}
	\end{equation*}
which proves the first implication.\\\\
\textbf{2 $ \implies $ 3.} From Proposition \ref{P-15}, we have that $ \exists$ a $ G_\delta $ set $ G $ such that $ \comp{E} \subseteq G $ and $ \lom{G\setminus \comp{E}} =0 $. Note that the complement of countable intersection of open sets is countable union of closed sets. Therefore, $ F =\comp{G} $ is an $ F_\sigma $ set. Now, $ \comp{G} \subseteq \comp{(\comp{E})} = E $. Now, we know that $ E\setminus \comp{G} = G\setminus \comp{E}$. Therefore, we have the result as follows:
\begin{equation*}
	\begin{split}
		\lom{E\setminus\comp{G}} &= \lom{G\setminus\comp{E}}\\	
		&= 0.
	\end{split}
\end{equation*}
\textbf{3 $ \implies $ 1.} Since $ F $ is an $ F_\sigma $ set, therefore, $ F \in \msigm{\lambda^*}$. Moreover, as Statement 2 show, $ \lom{E\setminus F} = 0 $, thus by Proposition \ref{P-10}, $ E\setminus F \in \msigm{\lambda^*}$. Since,
\[E = F \union (E\setminus F)\]
that is $ E $ is union of two Lebesgue Measurable sets, therefore $ E \in \msigm{\lambda^*} $, completing the proof.
\end{proof}
\hrulefill
\begin{definition}\label{D-15}
	(\textbf{Complete Measure Space}) The measure space $ (X,\alg{A},\mu) $ is complete if the for any $ A\in \alg{A} $ such that $ \m{A} = 0 $ implies that for any subset $ B\subseteq A $,
	\[\m{B} = 0.\]
\end{definition}
\begin{remark} Trivial to see are the following:
	\begin{itemize}
		\item {Hence, if $ \mu^* $ is an outer measure defined on $ X $, then the space $ (X,\msigm{\mu^*}, \mu^*) $ is complete (follows from Proposition \ref{P-10}).}
		\item {This means that the Lebesgue outer measure restricted to Lebesgue measurable subsets of $ \R $, $ (\R,\msigm{\lambda^*}, \lambda) $ is complete.}
	\end{itemize}
	
\end{remark}
\hrulefill
\begin{definition}
	(\textbf{Completion of a Measure Space}) Let $ (X,\alg{A}) $ be a measurable space and let $ \mu $ be a measure on $ \alg{A} $. The completion of $ \alg{A} $ under $ \mu $ is the collection $ \alg{A}_\mu $ of subsets $ A\subseteq X $ for which there are sets $ E $ and $ F $ in $ \alg{A} $ such that 
	\[E\subseteq A\subseteq F\]
	and
	\[\m{F-E} = 0\footnote{Note that, in Exercise III, Q. 2, we proved that for any $A\in \alg{A} $, this is trivially true. That is, all $ \alg{A} $-measurable subsets are $ \alg{A}_\mu $-measurable. In particular, $ E $ was a $ F_\sigma $ set and $ F $ was a $ G_\delta $ set.}.\]
\end{definition}
\hrulefill
\newpage
\section{Measurable Functions}
We now see the definition and basic properties of Measurable Functions, which would later be used to define Lebesgue Integral.
\begin{definition}
	(\textbf{Measurable Function}) Let $ (X,\alg{A}) $ be a measurable space and let $ A \subseteq X$ which is in $ \alg{A} $. The function $ f : A\to [-\infty,+\infty] $, is called a Measurable function\footnote{One writes $ f $ as $ \alg{A} $-measurable function to denote the $ \sigma $-algebra over whose subset the function $ f $ is defined.} if 
	\[\{x \where f(x) > \alpha\}\;\text{for any }\alpha \in \R \text{ is Measurable (belongs in $\alg{A} $).}\]
\end{definition}
\begin{remark}
	Please note that the function $ f $ defined above has a measurable domain. 
\end{remark}
\hrulefill
\begin{proposition}\label{P-17}
	Let $ (X,\alg{A}) $ be a measurable space and $ A\in \alg{A} $. Let $ f : A\to [-\infty,+\infty]$ be a function. Then, the following statements are equivalent:
	\begin{enumerate}
		\item {$ f $ is a Measurable Function.}
		\item {For all $ \alpha \in \R $, the set $ \{x\where f(x)\ge \alpha\} \in \alg{A}$.}
		\item {For all $ \alpha \in \R $, the set $ \{x\where f(x)< \alpha\} \in \alg{A}$.}
		\item {For all $ \alpha \in \R $, the set $ \{x\where f(x)\le \alpha\} \in \alg{A}$.}
	\end{enumerate}
\end{proposition}
\begin{proof}
	The equivalence is shown as follows:\\
	\textbf{1 $ \implies $ 2.} Since $ f $ is a measurable, therefore for all $ \alpha \in \R $, the set $ \{x\where f(x) > \alpha\} \in \alg{A}$. This means that $ C_{\alpha - \frac{1}{n}} =\left \{x\where f(x) > \alpha - \frac{1}{n} \right \} \in \alg{A}$ for all $ n\in \N $. Now, the following set 
	\[C = \bintrs_{n} C_{\alpha - \frac{1}{n}} = \{x\where f(x) \ge \alpha\}.\]
	is Measurable as $ C \in \alg{A} $ because $ C_{\alpha - \frac{1}{n}} \in \alg{A}$ for any $ n\in \N $, hence the countable intersection would also be in $ \alg{A} $, hence measurable.\\
	\textbf{2 $ \implies $ 3.}
	Since $ \{x\where f(x) \ge \alpha\} \in \alg{A} $, therefore it's complement $ \{x\where f(x) < \alpha\} \in \alg{A}$ for any $ \alpha \in \R $.\\
	\textbf{3 $ \implies $ 4.} Since $ \{x\where f(x)< \alpha\}\in \alg{A} $ for any $ \alpha\in \R $, thus, $ C_{\alpha +\frac{1}{n}} = \{x\where f(x) < \alpha + \frac{1}{n}\} \in \alg{A}$ for all $ n \in \N $, hence 
	\[C = \bintrs_{n} C_{\alpha + \frac{1}{n}} = \{x\where f(x)\le \alpha\}\]
	and since each $ C_{\alpha + \frac{1}{n}} \in \alg{A} $, therefore $ C \in \alg{A}$.\\
	\textbf{4 $ \implies $ 1.} Since $ \{x\where f(x) \le \alpha \} \in \alg{A} $ then it's complement $ \{x\where f(x) >\alpha\} $ for all $ \alpha \in \R $, making $ f $ measurable.
\end{proof}
\hrulefill
\begin{proposition}
	\label{P-18}
	The following are basic examples of measurable functions:
	\begin{itemize}
		\item {If $ f $ is a Measurable function, then the set $ \{x\where f(x) = \alpha\} $ is measurable for all $ \alpha \in R $.}
		\item {Constant functions are Measurable.}
		\item {The characteristic function $ \chi_A $ defined by:
	\[\chi_A(x) = \begin{cases}
		1 & x\in A\\
		0 & x\notin A
	\end{cases}\]	
is Measurable if and only if $ A $ is measurable.	
}
\item {Continuous functions are Measurable.}
\item {Let $ (X,\alg{A}) $ be a measurable space. If $ f $ and $ g $ are Measurable functions on $ X $, then the sets
\begin{equation*}
	\begin{split}
		&\{x\in X\where f(x) \neq g(x)\}\\
		&\{x\in X\where f(x)< g(x)\}
	\end{split}
\end{equation*}
are measurable (belongs to $ \alg{A} $).
}
\item {\footnote{Question 3 of Exercise 3.} Monotone functions are Measurable.}
\item {\footnote{Question 4 of Exercise 3.}Consider $ f : \R \to \R $ is a differentiable function. Then $ f^{\prime} $ is a $ \lambda $-Measurable function.}
	\end{itemize}
\end{proposition}
\begin{proof}
	The\textbf{ first }example is trivial to see in light of Proposition \ref{P-17} by taking intersection of $ \{x\where f(x)\le \alpha\} $ and $ \{x\where f(x) \ge \alpha\} $, both of which are measurable.\\
	For \textbf{second}, consider the constant function $ f(x) = b \;\forall \; x \in \R$. Now, for all $ \alpha \in\R $, consider the set $ \inv{f}((\alpha,\infty)) =  \{x\where f(x)>\alpha\}$.  If $ b > \alpha $, then we are done, if $ b\le \alpha $, then by previous result, $ \{x\where f(x) \le \alpha\} $ is also measurable (equal to $ \R $ and $ \R\in \alg{A} $).\\
	For \textbf{third} example, consider the set $ \inv{\chi_A}(\alpha,\infty) = \{x\where \chi_A(x) > \alpha\} $ for any $ \alpha \in \R $. If $ \alpha >1 $, then $ \inv{f}(\alpha,\infty) =  \Phi \in \alg{A}$. If $ \alpha = 1 $, then $\inv{f}[\alpha,\infty) = A$, since $ \chi_A(x) $ is given measurable, hence $ A $ is measurable. Now, Assume that $ A $ is measurable. Then consider the set $ \inv{\chi_A}(\alpha,\infty) $ for any $ \alpha\in \R $. As we saw previously, the case for $ \alpha > 1 $ is trivial. For $ 0< \alpha \le 1 $, $ \inv{\chi_A}(\alpha,\infty) = A \in \alg{A}$. Finally, for $ \alpha\le 0 $, $ \inv{\chi_A} (-\infty,\alpha]  = \Phi \in \alg{A}$. Thus, $ \chi_A $ is measurable.\\
	For \textbf{fourth}, since $ f $ is continuous (so inverse of open sets is open, by definition), therefore $ \inv{f} (\alpha,\infty) $ is open in $ \R $, hence it must be Borel, hence measurable for any $ \alpha \in \R $.\\
	For \textbf{fifth}, since $ f $ and $ g $ are measurable. Then due to next Proposition \ref{P-19}, we know that $ f-g $ is also measurable. This means that for any $ \alpha \in \R $,
	\begin{equation*}
		\begin{split}
			\{x\in X\where f(x)-g(x) < \alpha\}
		\end{split}
	\end{equation*}
is measurable. Now set $ \alpha = 0 $ to get the result. Moreover, from this, we also get that $ \{x\in X\where f(x) -g(x) >0\} $ is also measurable. Hence,
\[\{x\in X \where f(x)-g(x)\neq 0\} = \{x\in X\where f(x)- g(x) < 0\} \bunion \{x\in X \where f(x)-g(x) > 0\}\]
is also measurable.\\
For \textbf{sixth}, we proceed as follows:\\
Consider the function $ f : A \to \R $ where $ A \in \msigm{\lambda^*}$ to be monotone. Now, consider the following two sets for any $ \alpha \in \R $:
\begin{equation*}
	\begin{split}
		A_1 &= \{x\in A \where f(x) > \alpha\}\\
		A_2 &= \left (\inv{f}(\alpha),\infty\right ) \intrs A
	\end{split}
\end{equation*}
Now, take any $ x\in A_1 $, then $ f(x) > \alpha  \implies x > \inv{f}(\alpha)$. Now if $ \inv{f}(\alpha) \intrs A = \Phi $, then $ \{x\in A \where f(x) > \alpha\} = \Phi $ which is trivially measurable and we would be done. If however $ \inv{f}(\alpha) \intrs A \neq \Phi $, then $ \inv{f}(\alpha) = \{y\in A \where f(y) = \alpha\} $ so that $ f(y) > \alpha $ implies that $ y > \inv{f}(\alpha) $ so that $ f(y) > f(\inv{f}(\alpha))  = \alpha$. Therefore, $ x > \inv{f}(\alpha) $, that is $ x\in A_2 $, proving that $ A_1 \subseteq A_2 $. \\
Similarly, take $ x\in A_2 $, therefore 
\begin{align*}
	x &> \inv{f}(\alpha) &&\\
	f(x) &> f\left (\inv{f}(\alpha)\right )\\
	f(x) &> \alpha\\
	x&\in \{x\where f(x) > \alpha\}\\
	x\in A_1.
\end{align*}
Therefore $ A_2 \subseteq A_1 $. Hence, $ A_1 = A_2 $. But since $ \left (\inv{f}(\alpha),\infty\right ) $ is an interval, hence measurable and $ A $ is given measurable, therefore $ A_2 = \left (\inv{f}(\alpha),\infty\right ) \intrs A$ is measurable, which makes $ A_1 = \{x\where f(x) > \alpha\}= A_2$ measurable for all $ \alpha \in \R $.\\
For \textbf{seventh}, the result is simple to see since we are given that $ f $ is $ \lambda $-Measurable due to continuity (see Statement 4). Therefore, we can define the sequence of functions $ \{f_n\} $ as follows:
\begin{equation*}
	\begin{split}
		f_n(x) &= \frac{f\left (x + \frac{1}{n}\right ) - f(x)}{\frac{1}{n}} \forall\;x\in \R.
	\end{split}
\end{equation*}
As we can see, $ f_n $ is $ \lambda $-Measurable due to Proposition \ref{P-19}. Hence, we can see that because $ f^{\prime}(x) = \lim_{h\to 0}\frac{f(x+h)-f(x)}{h} $ for any $ x\in \R $, and since $ \lim_{n\to \infty} f_n(x) = f(x)$, therefore $ f_n \to f^\prime $ is $ \lambda $-Measurable (Proposition \ref{P-20}).
\end{proof}
\hrulefill
\newpage
\begin{proposition} \label{P-19}
	Let $ (X,\alg{A}) $ be a measurable space and let $ A \in \alg{A}$. Consider two Measurable functions $ f,g : A \longrightarrow [0,+\infty] $ and $ c\in \R $. Then,
	\begin{enumerate}
		\item {$ f+ c $,}
		\item {$ f\pm g $}
		\item {$ cf $,}
		\item {$ fg $}
	\end{enumerate}
are also Measurable.
\end{proposition}
\begin{proof}
	\textbf{1.} Since $ f $ is measurable, therefore the set $ \{x\where f(x) > \alpha- c\} = \{x\where f(x) + c\ge \alpha\} $ is measurable for any $ \alpha \in \R $.\\
	\textbf{2.} Both $ f $ and $ g $ are given measurable. The set $ \inv{(f+g)} (\alpha,\infty) $ can be written as:
	\begin{equation*}
		\begin{split}
			\inv{(f+g)}(\alpha, \infty)  &= \{x \where f(x) + g(x) > \alpha\}\\
			&= \{x\where f(x) > \alpha - g(x)\}\\
			&= \{x\where f(x) > b\}
		\end{split}
	\end{equation*}
where $ b\in [-\infty ,\alpha] $. Note that the case where $ g(x) = +\infty $ is trivial as $ f(x) > \alpha - (+\infty) \equiv f(x) > -\infty$, which is by definition of co-domain of $ f $. Now since $ \{x\where f(x) > b \}$ is measurable for any $ b\in \R \supset (-\infty,\alpha] $ for any $ \alpha\in \R $, therefore $ \inv{(f+g)}(\alpha,\infty) $ is measurable for any $ \alpha\in \R $. \\
\textbf{3.} Note that for $ c=0 $, the function becomes constant and hence measurable (Proposition \ref{P-18}). Consider the set $ \inv{(cf)}(\alpha,\infty)$. We can write this as follows, 
\begin{equation*}
	\begin{split}
		 \inv{(cf)}(\alpha,\infty) &= \{x\where cf(x) > \alpha\}\\
		 &= \{x\where f(x)> \alpha / c\}
	\end{split}
\end{equation*}
where $ c\neq 0 $. Since $ f $ is measurable, therefore $ \{x\where f(x)>\alpha/c\} $ is also measurable for any $ \alpha \in \R $. Hence $ cf $ is measurable.\\
\textbf{4.} Consider the set $ \inv{(f^{2})}(-\infty,\alpha) $ for any $ \alpha \in \R $. 
\begin{equation*}
	\begin{split}
		\inv{(f^{2})}(-\infty,\alpha) &= \{x\where f^2(x) < \alpha\}\\
		&= \{x\where -\sqrt{\alpha} < f(x) < \sqrt{\alpha}\}\\
		&= \{x\where f(x) < \sqrt{\alpha}\} \bintrs \{x\where f(x) > -\sqrt{\alpha}\}
	\end{split}
\end{equation*}
Therefore if $ f $ is Measurable, then $ f^{2} $ is Measurable. With this, we can simply write $ fg $ as:
\begin{equation*}
	\begin{split}
		fg = \frac{(f+g)^{2} - (f-g)^{2} }{4}
	\end{split}
\end{equation*}
which, by previous results (2 \& 3), is measurable. 
\end{proof}
\hrulefill
\begin{proposition}\footnote{Question 1 of Exercise 3.}
Let $ (X,\alg{A}) $ be a measurable space. Consider a function $ f : A\to \R $ where $ A\in \alg{A} $. Then the following are equivalent:
\begin{enumerate}
	\item {$ f $ is a $ \alg{A} $-Measurable function.}
	\item {$ \inv{f}(U) $ is a measurable set $ \forall $ open sets $ U \subseteq \R $.}
	\item {$ \inv{f}(C) $ is a measurable set $ \forall $ closed sets $ C \subseteq \R $.}
	\item {$ \inv{f}(B) $ is a measurable set $ \forall $ borel sets $ B \in \bor{R} $.}
\end{enumerate}	
\end{proposition}
\begin{proof}
	\emph{The proof is exactly the same as of Proposition \ref{P-24}.}
\end{proof}

\hrulefill
\begin{definition}
	(\textbf{Sequence of fuctions}) If $ \{f_n\} $ is a sequence of $ [-\infty,+\infty] $ valued functions on $ A $, then $ \sup_{n} f_n : A \to [-\infty,+\infty]$ is defined by
	\[\left (\sup_{n} f_n\right )(x) = \sup\{f_n(x)\where n \in \N\}.\]
\end{definition}
\begin{remark}
	One similarly defines the following:
	\begin{itemize}
		\item {The infimum:
	\[\left (\inf_{n} f_n\right )(x) = \inf\{f_n(x)\where n \in \N\}.\]	
	}
\item {The limit supremum:
\[\left (\limsup_{n} f_n\right )(x) = \limsup\{f_n(x)\where n \in \N\}.\]
}
\item {The limit infimum:
\[\left (\liminf_{n} f_n\right )(x) = \liminf\{f_n(x)\where n \in \N\}.\]
}
\item {The limit:
\[\left (\lim_{n} f_n\right )(x) = \lim\{f_n(x)\where n \in \N\}.\]
}
	\end{itemize}
\end{remark}
\hrulefill
\begin{proposition}\label{P-20}
	Let $ (X,\alg{A}) $ be a measurable space and let $ A\in \alg{A} $. Consider $ \{f_n\} $ be a sequence of $ [-\infty,+\infty] $-valued measurable functions on $ A $. Then,
	\begin{enumerate}
		\item {The functions $ \sup_{n}f_n $ and $ \inf_n f_n $ are measurable.}
		\item {The functions $ \limsup_{n}f_n $ and $ \liminf_{n} f_n$ are measurable.}
		\item {The function $ \lim_{n} f_n $ (whose domain is $ \{x\in A \where \limsup_nf_n(x) = \liminf_nf_n(x)\} $) is measurable.}
	\end{enumerate}
\end{proposition}
\begin{proof}
	Note that the set $ \inv{(\sup_{n}f_n)}(-\infty,\alpha] = \{x\in A\where (\sup_n f_n)(x) \le \alpha\} = \bintrs_n \{x\in A \where f_n(x)\le \alpha\}$. Therefore $ \sup_n f_n $ is measurable. Similarly, $ \inv{(\inf_n f_n)}(-\infty,\alpha)  = \{x\in A \where (\inf_nf_n)(x) <\alpha\} = \bunion_n \{x\in A\where f_n(x)<\alpha\}$. Now, denote $ g_k = \sup_{n\ge k}f_n $ and $ h_k = \inf_{n\ge k}f_n $. But since $ \limsup_{n} f_n = \inf_{n\ge 0} \sup_{k\ge n} f_k = \inf_{n\ge 0} g_n$ and $ \{g_n\} $ is measurable by $ 1^{st} $ property, therefore $ \limsup_{n} f_n $ is also measurable, similarly for $ \liminf_{n} $.
\end{proof}
\hrulefill
\newpage
\subsection{Almost Everywhere Property.}
\begin{definition}
	(\textbf{$ \mu $-almost everywhere}) Let $ (X,\alg{A},\mu) $ be a measure space. A property $ P $ of points of $ X $ is said to hold $ \mu $-almost everywhere if the set
	\[N = \{x\in X\where P \text{ does not hold for }x \}\]
	has measure zero. That is,
	\[\m{N} = 0.\]
	
\end{definition}
\begin{remark}
	Note that it's not necessary for the set $ N $ to belong in $ \alg{A} $. The only requirement is for the set $ N $ to be contained in a set $ F \in \alg{A} $ and then $ \m{F} = 0 $ (which automatically implies that $ \om{N} = 0 $).\\\\
	\textbf{But}, if $ \mu $ is complete then $ N \in \alg{A} $. See Definition \ref{D-15}.
\end{remark}
\hrulefill
\begin{definition}
	(\textbf{Almost everywhere convergence}) If $ \{f_n\} $ is a sequence of functions on $ X $ and $ f $ is a function on $ X $, then
	\[\{f_n\}\longrightarrow f\;\text{almost everywhere.}\]
	if the set
	\[\{x\in X\where f(x)\neq \lim_nf_n(x)\}\]
  is of measure zero.
\end{definition}
\hrulefill
\begin{proposition}\label{P-21}
	Let $ (X,\alg{A},\mu) $ be a measure space and let $ f $ and $ g $ be extended real valued functions on $ X $ that are equal almost everywhere. If $ \mu $ is \textbf{complete} and if $ f $ is Measurable, then $ g $ is also Measurable.
\end{proposition}
\begin{proof}
Consider the region of non-equality as
\[N = \{x\where f(x)\neq g(x)\}.\]
Given to us is the fact that $ \om{N} = 0 $ and since $ \mu $ is complete, so $ N\in \alg{A} $. Now, consider the following for any $ \alpha\in \R $:
\[\{x\where g(x)\ge \alpha\} = \left (\{x\where g(x)\ge \alpha\} \intrs N\right ) \bunion \left (\{x\where g(x)\ge \alpha\}\intrs \comp{N}\right ).\]
Denote the set $ A = \{x\where g(x)\ge \alpha\} \intrs N $ and $B =\{x\where g(x)\ge \alpha\}\intrs \comp{N}$. Since for any $ x\in\left (\{x\where g(x)\ge \alpha\}\intrs \comp{N}\right ) $, $ f(x) = g(x) $, therefore, we can equivalently write $ B = \left (\{x\where f(x)\ge \alpha\}\intrs \comp{N}\right ) $. Now $ \comp{N} \in \alg{A}$ and due to Measurability of $ f $, $ \{x\where f(x)\ge \alpha\}\in \alg{A} $. Hence $ B\in \alg{A} $. Finally, due to $ \{x\where g(x)\ge \alpha\} \intrs N \subseteq N $ and $ \mu $ being complete with $ \m{N} = 0 $, we get $ \{x\where g(x)\ge \alpha\} \intrs N \in \alg{A} $, completing the proof. 
\end{proof}
\hrulefill
\begin{proposition}
	Let $ (X,\alg{A},\mu) $ be a measure space, let $ \{f_n\} $ be sequence of extended real valued functions on $ X $ and let $ f $ be an extended real valued function on $ X $ such that
	\[\{f_n\}\longrightarrow f\;\text{almost everywhere.}\]
	If $ \mu  $ is \textbf{complete} and if each $ f_n $ is Measurable, then $ f $ is Measurable.
\end{proposition}
\begin{proof}
	As Proposition \ref{P-20} shows, $ \liminf_n f_n $ and $ \limsup_{n} f_n $ are Measurable. As the given condition shows, $ \liminf_{n}f_n $ is equal to $ f $ for almost all $ X $. Hence Proposition \ref{P-21} implies that $ f $ is also Measurable.
\end{proof}
\hrulefill
\subsubsection{A striking result}
The following is an interesting result about Lebesgue Measurable functions on $ \R $.
\begin{proposition}
	Ci
\end{proposition}
\newpage
\subsection{Cantor Set, again.}
With the new tool in hand (measurable functions), we now turn back to the ever-interesting Cantor set, this time, to prove the sheer size of the $ \sigma $-algebra $ \msigm{\lambda^*} $ in comparison to the Borel $ \sigma $-algebra $ \bor{\R} $. In particular we show that $ \bor{\R} \subsetneq \msigm{\lambda^*} $.\\
But before that, we look at following results:
\begin{proposition}
	Show that the function $ \phi $ defined by
	\begin{equation*}
		\begin{split}
			\phi &: [0,1]\longrightarrow P\;\text{(Cantor Set), such that,}\\
			\phi(\alpha) &= \sum_{n=1}^{\infty} \frac{2b_n}{3^n}\;\text{for }\alpha\in[0,1]
		\end{split}
	\end{equation*} 
where $ b_n \in \{0,1\}\;\forall \; n\in\N$ is Measurable in $ \msigm{\lambda^*} $.
\end{proposition}
\begin{proof}
	Note that $ \phi(\alpha) $ thus maps a decimal number to it's binary representation $ \{b_n\} $. First, we define the following function:
	\begin{equation*}
		\begin{split}
			\phi_n &: [0,1]\longrightarrow \{0,1\}\\
			\phi_n(\alpha) &= b_n. 
		\end{split}
	\end{equation*}
That is, $ \phi_n $ maps $ \alpha $ to it's $ n^{th} $ binary digit. We can see that $ \phi_n(\alpha) $ can be written as the following:
\begin{equation*}
	\begin{split}
		\phi_n(\alpha) = \chi_{E_n} = \begin{cases}
			1&\text{ if }  \alpha \in E_n \\
			0&\text{ otherwise.}
		\end{cases}
	\end{split}
\end{equation*}
where $ E_n $ is the intersection of countable sequence of sub-intervals of $ [0,1] $. Hence $ E_n $ is a Lebesgue measurable subset of $ \R $, so it is in $ \msigm{\lambda^*} $. But, as Proposition \ref{P-18}, statement 3 shows, $ \chi_{E_n} = \phi_n $ is then a Measurable function.\\
Now, the following arguments:
\begin{equation*}
	\begin{split}
		\frac{2}{3^n} \phi_n(\alpha) &= \frac{2b_n}{3^{n}} \text{ is measurable (Proposition \ref{P-19}).}\\
		\left \{\frac{2\phi_n(\alpha)}{3^n}\right \}&\text{ is a sequence of measurale functions.}\\
		\left \{\sum_{k=1}^{n} \frac{2\phi_n(\alpha)}{3^n}\right \}&\text{ is also a sequence of measurable functions (Proposition \ref{P-19}).}\\
		\lim_{n\to \infty} \sum_{k=1}^{n} \frac{2\phi_n(\alpha)}{3^n}&\text{ is a measurable function (Proposition \ref{P-20}).}
	\end{split}
\end{equation*} 
Hence the function which maps each real from $ [0,1] $ to it's binary representation is measurable.
\end{proof}
\hrulefill
\begin{proposition}\label{P-24}
	Let $ (X,\alg{A}) $ be a measurable space. If $ f $ is a $ \alg{A} $-Measurable function on $ A $ and $ B\in \bor{\R} $, then $ \inv{f}(B) \in \alg{A}$.
\end{proposition}
\begin{proof}
	Denote $ \alg{D} $ be the following set:
	\[\alg{D} = \{B\subseteq \R\where \inv{f}(B) \in \alg{A}\}.\]
	Now, note that,
	\begin{enumerate}
		\item {Since $ \inv{f}(\R) = A \in \alg{A}$, therefore $ \R \in \alg{D} $.}
		\item {If $ B \in \alg{D} $, then 
	\begin{equation*}
		\begin{split}
			\comp{B} &= \R \intrs \comp{B}\\
		\end{split}
	\end{equation*}	
and
\begin{equation*}
	\begin{split}
		\inv{f}(\comp{B}) &= \inv{f}(\R\intrs \comp{B})\\
		&= \inv{f}(\R) \intrs \inv{f}(\comp{B})\\
		&=A\intrs \left (\inv{f}(B)\right )^{\text{c}}\\
	\end{split}
\end{equation*}
Now since $ A \in \alg{A}$ and $ \inv{f}(B) \in \alg{A} $ because $ B\in \alg{D} $, therefore $ \inv{f}(\comp{B})\in \alg{A} $ so that $ \comp{B} \in \alg{D} $.
	}
\item {We know that from the basic results of set functions that
\begin{equation*}
	\begin{split}
		\inv{f}\left (\bunion_n B_n\right ) &= \bunion_n \inv{f}(B_n)
	\end{split}
\end{equation*}
}
	\end{enumerate}
Hence $ \alg{D} $ is a $ \sigma $-algebra on $ \R $ (!) Now, due to measurability of $ f $, we know that the set $ \{x\where f(x)>\alpha\} $ is in $ \alg{A} $, which is equivalent to saying that $ \inv{f}(\alpha,\infty) \in \alg{A}$. This hence means that $ (\alpha,\infty) \in \alg{D} $ for any $ \alpha\in \R $. Proposition \ref{P-2} showed that a $ \sigma$-algebra generated by such subsets of $ \R $ is $ \bor{\R} $. Hence, for any $ B\in \bor{\R} $, we have that $ B\in \alg{D} $. Therefore for any Borel set $ B $, $\inv{f}(B) \in \alg{A}$\footnote{This is a very interesting way to prove such a statement. Notice how we analyzed the set of all possible subsets of $ \R $ for which $ \inv{f}(B) \in \alg{A}$ right from the start!}.
\end{proof}
\hrulefill

\textcolor{red}{\emph{\textbf{WHY IS THIS SUBSECTION STILL NOT COMPLETE!?}}}
\newpage
\subsection{Sequence of Functions approximating a Measurable Function.}
We now show that any measurable function can be defined in terms of a simple function and a step function. For this, we first define what we mean by simple functions in Definition \ref{D-21}. Before that, let's see few more interesting-but-basic properties of measurable functions. 
\begin{proposition}\label{P-25}
	Let $ (X,\alg{A}) $ be a measurable space and $ f $ be an extended real valued function on $ A\in \alg{A} $. Define the following:
	\[f^{+}(x) = \max (f(x),0)\;\text{and } f^{-}(x) = -\min (f(x),0).\]
	Then, $ f $ is measurable if and only if $ f^{+} $ and $ f^{-} $ both are measurable on $ A $.
\end{proposition}
\begin{proof}
	If $ f $ is measurable, then $ \{x\where f(x)\ge \alpha\} $ is measurable. Note that $ f^{+}(x) \ge 0$. Hence, for the case when $ \alpha > 0 $, the set $ \{x\where f^{+}(x)\ge \alpha\} = \{x\where f(x) > \alpha\}$ which is measurable due to measurability of $ f $. Similarly, if $ \alpha = 0 $, then $ \{x\where f^{+}(x)\ge 0\} = \{x\where f(x)> 0\} \bunion \{x\where f(x) = 0\} $ in which both sets are measurable in view of Proposition \ref{P-18}. Finally, for $ \alpha<0 $, we have $ \{x\where f^{+}(x)> \alpha\} = \{x\where f^{+}(x)\ge 0\}$ which again is measurable. Now, $ f^{-}(x) = -\min(f(x),0) = \max(-f(x),0)$ and since $ -f $ is also measurable (Proposition \ref{P-19}), therefore if $ f $ is a measurable function, then $ f^{+} $ and $ f^{-} $ are both measurable functions too.\\
	To show the converse, note that $ f = f^{+} - f^{-} $ and since both are measurable, therefore $ f $ is also measurable (Proposition \ref{P-19}).
\end{proof}
\begin{remark}
	Due to the above result, we can hence deduce that if $ f $ is a $ \alg{A} $-Measurable function then,
	\[\abs{f} = f^{+} + f^{-}\;\text{is a Measurable function on $ A $.}\]
\end{remark}
\hrulefill
\begin{proposition}
	Let $ (X,\alg{A}) $ be a measurable space and $ A\in \alg{A} $. Let $ f : A\to [-\infty,+\infty]$. Then,
	\begin{enumerate}
		\item {If $ f $ is $ \alg{A} $-Measurable and if $ B $ is a subset of $ A $, then the restriction $ f_{B} $ of $ f $ to $ B $ is also $ \alg{A} $-Measurable. }
		\item {If $ \{B_n\} $ is a sequence of sets that belong to $ \alg{A} $ such that $ A = \bunion_{n}B_n $ and $ f_{B_n} $ is $ \alg{A}$-Measurable for each $ n $, then $ f $ is also $ \alg{A} $-Measurable.}
	\end{enumerate}
\end{proposition}
\begin{proof}
	The first result follows directly from the following observation:
	\[\{x \in B\where f_b(x)>\alpha\} = B\bintrs \{x\in A\where f(x)>\alpha\}\]
	and the second result follows from the following:
	\[\{x\in A\where f(x)>\alpha\} = \bunion_{n} \{x\in B_n\where f_{B_n}(x) > \alpha\}.\]
	both for any $ \alpha\in \R $.
\end{proof}
\hrulefill
\begin{definition}\label{D-21}
	(\textbf{Simple Function}) A function is called simple if it has only finitely many values. Equivalently, we say that $ f $ is simple if we can write it as the following:
	\[f = \sum_{k=1}^{N}\alpha_k \chi_{E_k}\;,\;\;\alpha_{k}\in \R\]
	where each $ E_k $ is a measurable set of finite measure.
\end{definition}
\begin{remark}
	Note that \begin{itemize}
		\item {If $ E_k $ are intervals, then we say $ f $ to be a \textbf{step function}.}
	\end{itemize}
\end{remark}
\hrulefill

\emph{The following Proposition asserts that \textbf{any Measurable function can be approximated by an increasing sequence of simple functions.}}
\begin{proposition}\label{P-27}
	Let $ (X,\alg{A}) $ be a measurable space and let $ A \in \alg{A} $ with $ f :A\to[0,+\infty] $ be a Measurable function on $ A $. Then there exists a sequence $ \{f_n\} $ of \textbf{simple} $ [0,+\infty) $-valued Measurable functions on $ A $ that satisfy 
	\[f_1(x)\le f_2(x)\le f_3(x)\le \dots\]
	and 
	\[\lim_{n\to\infty } f_{n}(x) = f(x)\]
	for any $ x\in A $.
\end{proposition}
\begin{proof}
	For the proof, construct the following sequence of sets, by dividing the whole interval $ [0,n] $ for any $ n \in \N$ into $ n2^n $ number of intervals each of length $ \frac{1}{2^n} $ and denote the following set:
	\[A_{n,k} = \left \{x\in A\left\vert\frac{k-1}{2^n}\le f(x)<\frac{k}{2^n}\right.\right \}\]
	for any $ n\in \N $ and $ k = 1,2,\dots,n2^n $. With this construction, we can now define the following function for each $ n $:
	\begin{equation*}
		\begin{split}
			\phi_n &: A \to [0,\infty), \text{ defined as}\\
			\phi_n(x)&= \begin{cases}
				\frac{k-1}{2^n}\;&\text{ if }x\in A_{n,k}\;\text{for any $ k=1,2,\dots,n2^n $}\\
				n\;&\text{ if } x\in A- \bunion_k A_{n,k}.
			\end{cases} 
		\end{split}
	\end{equation*}
	Note that we can alternatively write $ \phi_n(x) $ as the following (with more clarity):
	\begin{equation*}
		\begin{split}
			\phi_n(x) = \begin{cases}
				\frac{k-1}{2^n} &\text{ if }f(x)\le n\;,\;\;\text{where } \frac{k-1}{2^n} \le f(x) < \frac{k}{2^n} \;\text{for some k} \in  \{1,2,\dots,n2^n\} \\
				n &\text{ if }f(x)>n.
			\end{cases}
		\end{split}
	\end{equation*}
	We now show that $ \phi_n(x)\le \phi_{n+1}(x) \;\forall\;x\in A$. Let's first show this for $ f(x) \le n $. \\
	If $ f(x)\le n $, then,
	\begin{equation*}
		\begin{split}
			\phi_n(x) = \frac{k_0  -1}{2^n}\;\text{for some }k_0 \in \{1,2,\dots,n2^n\}
		\end{split}
	\end{equation*}
	such that $ \frac{k_0-1 }{2^n}\le f(x) < \frac{k_0}{2^n} $. Now, two cases arises:
	\begin{itemize}
		\item {\textbf{If} $ \frac{k_0-1}{2^n} \le f(x) < \frac{2k_0 -1}{2^{n+1}} $ : This is just the case that $ f(x) $ lies in the first half of the interval $ \left [\frac{k_0-1}{2^n}, \frac{k_0}{2^n}\right ] $. Hence, in this case we get that:
			\[\frac{k_0-1}{2^n} = \frac{2k_0 - 2}{2^{n+1}} \le f(x) < \frac{2k_0 - 1}{2^{n+1}}\]
			such that $ \phi_n(x) = \frac{k_0-1}{2^n} = \phi_{n+1}(x) $.	
		}
		\item {\textbf{If} $ \frac{2k_0 -1}{2^{n+1}} \le f(x) < \frac{k_0}{2^n}$ : This is the case when $ f(x) $ lies in the second half of the interval. In this case, we see that,
			\[\frac{2k_0 -1}{2^{n+1}} \le f(x) < \frac{2k_0}{2^{n+1}} = \frac{k_0}{2^{n}}\]
			so that $ \phi_n(x) = \frac{k_0 -1}{2^n} = \frac{2k_0 -2}{2^{n+1}} < \frac{2k_0 -1}{2^{n+1}} = \phi_{n+1}(x) $.
		}
	\end{itemize}
	Hence from both the cases, we have $ \phi_n(x) \le \phi_{n+1} (x) $ for all $ x \in A$ such that $ f(x)\le n $. One can similarly see the same result for $ n<f(x) \le n+1$ and for $ f(x) >n+1 $, $ \phi_n(x) \le \phi_{n+1}(x) $ follows trivially. Hence, we have proved that $ \forall n\in \N $ and $ x\in A $, 
	\begin{equation}
		\label{monotone}
		\phi_n(x) \le \phi_{n+1}(x).
	\end{equation}
	Now, one can write the function $ \phi_n $ as the following combination too:
	\begin{equation}
		\begin{split}
			\phi_n(x) &= \sum_{k=1}^{n2^n} \frac{k-1}{2^n}\chi_{A_{n,k}} + n\chi_{A - \bunion_k A_{n,k}}
		\end{split}
	\end{equation}
	Due to the above representation of $ \phi_n $, the following steps becomes easier (\& interesting) to see.\\
	Now, first note that $ A_{n,k} $ is a measurable set because it's intersection of two measurable sets. Moreover, $ A- \bunion_{k}A_{n,k} $ is also a measurable set. Hence, in view of Proposition \ref{P-18}, Statemen\textit{\textit{}}t 3, we get that $ \phi_n(x) $ is a Measurable function for any $ n\in \N $. Therefore, $ \{\phi_n\} $ is a sequence of Measurable functions adhering \eqref{monotone}. We again find two cases:
	\begin{itemize}
		\item {\textbf{If $ f $ is finite} : Now since $ f $ is finite, therefore $ \exists n_0 \in \N$ such that $ f(x)\le n_0 $. Hence, one can further deduce the following for all $ n\ge n_0 $ (hence $ f(x) \le n_0 \le n$),
			\begin{equation*}
				\begin{split}
					f(x) - \phi_n(x) &=  f(x) - \frac{k-1}{2^{n}}\;\;\text{for some }k\in\{1,2,\dots,n2^n\}\;\text{such that}\;\frac{k-1}{2^n}\le f(x)< \frac{k}{2^n}\\
					&<\frac{1}{2^n}
				\end{split}
			\end{equation*}
			Hence, as $ n\to \infty $, $ \abs{f(x) - \phi_n(x)} \to 0 $. 
		}
		\item {\textbf{If $ f $ is infinite for some $ x\in A $} : If $ f $ is infinite, then $ \forall n\in \N$, $ f(x) > n$. Hence,
			\[\phi_n(x) = n\text{ for all }n\in N.\]
			Therefore $ \lim_{n\to \infty} \phi_{n}(x) = +\infty = f(x) $ for particular $ x\in A $ where $ f $ is infinity.
		}
	\end{itemize}
	Hence, in both cases, $ \{\phi_n\} $ converges to $ f $. The proof is therefore complete.
\end{proof}
\hrulefill

\emph{The following can be considered as an\textbf{ important corollary of the above Proposition}.}
\begin{proposition}
	Let $ (X,\alg{A}) $ be a measurable space and let $ A \in \alg{A} $ with $ f :A\to[-\infty	,+\infty] $ be a Measurable function on $ A $. Then there exists a sequence $ \{f_n\} $ of \textbf{simple} $ (-\infty,+\infty) $-valued Measurable functions on $ A $ that satisfy 
	\[\abs{f_1(x)}\le \abs{f_2(x)}\le \abs{f_3(x)}\le \dots\]
	and 
	\[\lim_{n\to\infty } f_{n}(x) = f(x)\]
	for any $ x\in A $.
\end{proposition}
\begin{proof}
	 	Since $ f $ is a Measurable function, therefore $ f^{+} $ and $ f^{-} $ are Measurable functions too (Proposition \ref{P-25}). Now, since any function $ f $ can be written as 
	 	\[f = f^{+} - f^{-}\]
	 	therefore, by Proposition \ref{P-27}, we have two sequences $ \{f_n^{(1)}\} $ and $ \{f_n^{(2)}\} $ such that
	 	\[f_n^{(1)}\longrightarrow f^{+}\text{ and }f_n^{(2)}\longrightarrow f^{-}\]
		where $ f_1^{(1)}(x) \le f_2^{(1)}(x)\le \dots $ and $f_1^{(2)}(x) \le f_2^{(2)}(x) \le \dots $. Denote 
		\[f_n(x) = f_n^{(1)}(x) - f_n^{(2)}(x) \]
		Therefore, we see that
		 \[\abs{f_n(x)} = f_n^{(1)}(x) + f_n^{(2)}(x) \le f_{n+1}^{(1)}(x) + f_{n+1}^{(2)}(x) = \abs{f_{n+1}(x)}\]
		Now, we can deduce that
		\begin{equation*}
			\begin{split}
				\abs{f(x) - f_n(x)} &= \abs{f^{+}(x) - f^{-}(x) - f_n^{(1)}(x) + f_n^{(2)}(x) }\\
				&= \abs{f^{+}(x) - f_n^{(1)}(x) - \left (f^{-}(x) - f_n^{(2)}(x)\right )}\\
				&\le  \abs{f^{+}(x) - f_n^{(1)}(x)} + \abs{f^{-}(x) - f_n^{(2)}(x)}\\
				&\to 0 + 0
			\end{split}
		\end{equation*}
	Hence proved.
\end{proof}
\hrulefill
\newpage
\subsubsection{Replacing \emph{simple} functions by \emph{step} functions\protect\footnote{Unfortunately, this section onwards, we would tentatively only work with the Lebesgue Measure space $ (\R,\msigm{\lambda^*},\lambda) $ for pedagogical reasons of the instructor, so the abstract versions of the following results would need to wait.}.}
We now prove a similar result akin to Proposition \ref{P-27}, where we show that \emph{\textbf{any measurable function can be approximated by a sequence of step functions, almost everywhere}}. But before that, we prove \textbf{a basic fact about Lebesgue measurable sets with finite measure.}
\begin{proposition}\label{P-29}
	For any $ \lambda $-measurable set $ E $ of finite measure and a given $ \epsilon > 0 $, there exists a finite sequence of open intervals $ \{I_n\}_{n=1}^{N} $ such that 
	\[\lm{E\Delta \left (\bunion_{n=1}^{N}I_n\right )} < \epsilon.\]
\end{proposition}
\begin{proof}
	Take any $ \epsilon > 0 $, then we have for any set $ E \subseteq\R $, a sequence of open intervals $ \{I_n\} $ such that $ E\subseteq \bunion_n I_n $ and $ \lm{\bunion_n I_n} \le \lm{E} + \epsilon $ or $ \lm{\bunion_n I_n \setminus E} \le \epsilon <2\epsilon$. Now since $ \{I_n\} $ is a disjoint sequence, therefore, $ \lm{\bunion_n I_n} = \sum_n \lm{I_n} $ and due to the fact that $ \lm{E} < + \infty $, we get that $ \sum_n  \lm{I_n} < + \infty$.\\
%	 Now, we note the following:
%	\begin{equation*}
%		\begin{split}
%			E\Delta \bunion_n I_n &= \left (E \intrs \left (\bintrs_n \comp{I_n}\right )\right ) \bunion \left (\comp{E} \intrs \bunion_n I_n\right )\\
%			&= \Phi \bunion \left (\comp{E} \intrs \bunion_n I_n\right ) \;\;\;\;\text{($ \because \;E\subseteq \union_n I_n$ .)}\\
%			&= \bunion_n I_n \setminus E
%		\end{split}
%	\end{equation*}
Now, since $ \lm{E} < + \infty $, therefore the sum $ \sum_{n} \lm{I_n} <+\infty$, hence, $ \exists N \in \N $ such that $ \sum_{n=N+1}^{\infty}\lm{I_n} < \epsilon $. With this $ N $, we now see that:
\begin{equation*}
	\begin{split}
		\lm{E\Delta \bunion_{n=1}^{N} I_n} &= \lm{E\setminus \bunion_{n=1}^{N} I_n} + \lm{\bunion_{n=1}^{N}I_n \setminus E} \;\;\;\;\;\text{(both are disjoint.)}\\
		&\le \lm{E\setminus \bunion_{n=1}^{N} I_n} + \lm{\bunion_{n}I_n \setminus E}\\
		&= \lm{E\setminus \bunion_{n=1}^{N} I_n} + \lm{\bunion_{n}I_n \setminus E}\\
		&= \lm{E\intrs \left (\bunion_{n=1}^{N} I_n\right )^{\text{c}}} +  \lm{\bunion_{n}I_n \setminus E}\\
		&\le \lm{\bunion_{n=N+1}^{\infty} I_n} +\lm{\bunion_{n}I_n \setminus E}\;\;\;\;\text{$ \because \; E\intrs \left (\bunion_{n=1}^{N} I_n\right )^{\text{c}}\subseteq \bunion_{n=N+1}^{\infty} I_n$ }\\
		&\le \epsilon + \epsilon = 2\epsilon
	\end{split}
\end{equation*}
Hence, we get that for any finite Lebesgue measurable set $ E $, for all $ \epsilon > 0 $, $ \exists  $ a sequence of open intervals $ \{I_n\}_{n=1}^{N} $ such that their symmetric difference is a set with measure $ \le \epsilon $.
\end{proof}
\hrulefill
\begin{proposition}\label{P-30}
		Consider $ (\R,\msigm{\lambda^*}) $ to be the Lebesgue measurable space and $ A\in \msigm{\lambda^*} $. Let $ f : A\to [-\infty,+\infty] $ be a $ \lambda $-Measurable function. Then there exists a sequence of step functions $ \{\phi_k\} $ such that
		\[\phi_k \longrightarrow f\;\text{almost everywhere.}\]
\end{proposition}
\begin{proof}
		We will prove first that for any characteristic function, there exists a sequence of step functions converging to it. Let $ g = \chi_A $ be the characteristic function on $ A $. Continuing from Proposition \ref{P-29}, we see that if we write the \textbf{step-function }$ \phi $ as
		\[\psi = \sum_{k=1}^{N} \chi_{I_k}\]
		where $ \{I_k\} $ is the set of open intervals such that $ \lm{A \Delta \left (\bunion_{n=1}^{N} I_n\right )} < \epsilon$ for a given $ \epsilon > 0 $, from Proposition \ref{P-29}, then we get that the set $ \{x\where g(x) \neq \psi(x)\} $ has upper bound on it's measure given as follows:
		\begin{equation*}
			\begin{split}
				\left \{x\in A \union \left (\bunion_{k=1}^{N} I_k\right )\where g(x) = \psi(x)\right \} &\subseteq A \intrs \bunion_{k=1}^{N} I_k \;\;\;\;\text{$ \because $ $ g(x) = \psi(x) $ iff $ x\in \union_{k=1}^{N} I_k$ and $ 0 $ for other $ x \in A $}\\
				\left \{x\in A \union \left (\bunion_{k=1}^{N} I_k\right )\where g(x)\neq \psi(x)\right \}&\supseteq \left (A \intrs \bunion_{k=1}^{N} I_k\right )^{\text{c}}\supseteq A \Delta \bunion_{k=1}^{N} I_k.
			\end{split}
		\end{equation*}
	Similarly, it's easy to see that for any $ x $ such that $ g(x) \neq \psi(x) $, we have $ x \in A\Delta \bunion_{k=1}^{N} I_k $ so that we get,
	\[	\left \{x\in A \union \left (\bunion_{k=1}^{N} I_k\right )\where g(x) \neq \psi(x)\right \}\subseteq A \Delta \bunion_{k=1}^{N} I_k.\]
	Hence, 
	\[	\left \{x\in A \union \left (\bunion_{k=1}^{N} I_k\right )\where g(x) \neq \psi(x)\right \} = A\Delta \bunion_{k=1}^{N} I_k .\]
	Therefore, 
	\begin{equation*}
		\begin{split}
			\lm{	\left \{x\in A \union \left (\bunion_{k=1}^{N} I_k\right )\where g(x) \neq \psi(x)\right \}} < \epsilon
		\end{split}
	\end{equation*}
Therefore, for every $ n\ge 1 $, there exists a step-function $ \psi_n $ so that the set $ E_n = 	\left \{x\in A \union \left (\bunion_{k=1}^{N} I_k\right )\where g(x) \neq \psi_n(x)\right \} $ is such that
\[\lm{E_n} < \frac{1}{2^n}.\]
Now, \textbf{define} the following two sets:
\begin{equation*}
	\begin{split}
		F_n &= \bunion_{j=n+1}^\infty E_j\;\;\;\;\text{(a decreasing sequence)}\\
		F &= \bintrs_{k=1}^{\infty} F_k.
	\end{split}
\end{equation*}
For the set $ F_n $, observe that 
\begin{equation*}
	\begin{split}
		\lm{F_n} = \lm{\bunion_{j=n+1}^{\infty}} &\le \sum_{j=n+1}^{\infty} \lm{E_j}\\
		&< \sum_{j=n+1}^{\infty} \frac{1}{2^{j}}\\
		&= \frac{1}{2^n}
	\end{split}
\end{equation*}
and for set $ F $,
\begin{equation*}
	\begin{split}
		\lm{F} = \lm{\bintrs_{k=1}^{\infty} F_k} &= \lim_{k\to \infty} \lm{F_k}\;\;\;\;\text{$ \because $ Proposition \ref{P-7} \& $ \{F_k\} $ is measurable \& decreasing.} \\
		 &= 0.
	\end{split}
\end{equation*}
Note that $ \{F_k\} $ is measurable because any $ E_i $ is itself measurable because of Proposition \ref{P-18}, Statement 5.\\
Now, 
\[\psi_n(x) \longrightarrow g(x)\;\forall\;x\in \comp{F}\]
because $ \comp{F} = \comp{F_1} $ since $ \{F_k\} $ is a decreasing sequence, therefore $ \comp{F} $ is the set where $ g(x) $ satisfies with the limit step function.\\
Finally, $ \psi_n \not\to f\;\forall \;x\in F $, but since $ \lm{F} = 0 $, hence
\[\psi_n \longrightarrow g \;\text{almost everywhere}\]
Now, what we have proved so far is that for any characteristic function $ g = \chi_A $ on a measurable set, there exists a sequence of step functions converging to it point-wise almost everywhere. Since from Proposition \ref{P-27}, there exists a sequence of simple functions converging to $ f $, and since a simple function $ h = \sum_{i=1}^{M} \alpha_i \chi_{E_i}$ is a finite combination of characteristics functions over measurable sets, therefore there exists a sequence of step functions converging to $ f $ almost everywhere. \\
%In particular, note that $ \psi = \sum_{n} \alpha_n \psi_n $ is again a step function
In particular if $ \psi^{i}_{n} \longrightarrow \chi_{E_i} $ almost everywhere, then $ \sum_{i=1}^{M} \alpha_i \psi_n^{i} \longrightarrow \sum_{i=1}^{M}\alpha_i \chi_{E_i} = h$. Now by Proposition \ref{P-27}, there exists the sequence $ \{h_n\} $ of simple functions converging to $ f $. Since 
\[K_n = \left \{\sum_{i=1}^{M_n} \alpha_i \psi_n^i\right \} \longrightarrow h_n\;\text{almost everywhere,}\]
where note that $K_n = \sum_{i=1}^{M_n} \alpha_i \psi_n^i  $ is a step function because $ \psi_n^i $ is a step function and there are finitely many ($ M_n $) of them, and
\[\{h_n\}\longrightarrow f\]
therefore,
\[K_n \longrightarrow f\;\text{almost everywhere.}\]
Hence proved.
\end{proof}
\hrulefill
\subsection{Egorov's Theorem}
We now discuss a very important result in the theory of measurable functions named after Dmitri Fyodorovich Egorov, who published this result in 1911, thus establishing \textbf{a condition required for uniform convergence of a point-wise convergent sequence of Measurable functions.}
\begin{theorem} (\textbf{Egorov's Theorem})
	\label{T-3}
	Let $ (\R,\msigm{\lambda^*},\lambda) $ be the Lebesgue measure space on $ \R $. Suppose $ \{f_k\} $ is a sequence of real-valued, Lebesgue Measurable functions on $ E \in \msigm{\lambda^*} $ where $ \lm{E} < +\infty $. If 
	\[f_k \longrightarrow f\;\text{pointwise on $ E $,}\]\footnote{From Proposition \ref{P-20}, the limit of a sequence of Measurable functions is also Measurable, hence there's no point in writing extraneously the requirement for $ f $ to be also Measurable.}
	Then for each $ \epsilon > 0 $, there exists a \textbf{closed} set $ A_\epsilon \subset E$ such that
	\begin{enumerate}
		\item {$ \lm{E\setminus A_\epsilon} < \epsilon $, and}
		\item {$ f_k \longrightarrow f $ \textbf{uniformly} on $ A_\epsilon $.}
	\end{enumerate}
\end{theorem} 
\begin{proof} We break down the proof in the following 3 parts.\\\\
	\textbf{Act 1.} \emph{A Basic Construction.}\\
	 For each pair of integers $ n,k $, construct the following set:
	\[E^{n}_k = \left \{x\in E\;:\; \abs{f_j(x) - f(x)} < \frac{1}{n}\;,\;\;\forall\; j>k\right \}.\]
	Now, fix $ n $, so that we have the following observations:
	\begin{equation}
		\label{T-3.1}
		E_k^{n} \subseteq E_{k+1}^{n}
	\end{equation}
	and since $ f_k \longrightarrow  f$ point-wise, therefore 
\begin{equation}
	\label{T-3.2}
	\lim_{k\to \infty}\bunion_{i=1}^{k}E_i^{n} = E.
\end{equation}
Hence 
\[\lm{E\setminus E_k^{n}} \longrightarrow 0 \;\text{as } k\to \infty.\]
Note that the above result utilizes the fact that $ \lm{E} < +\infty $. Now by the above, we can say that $ \exists\; k_n $ such that 
\[\lm{E\setminus E_{k_n}^{n}} < \frac{1}{2^n}\]
which, by definition of $ E_k^{n} $ implies that 
\[\abs{f_j(x) - f(x)} < \frac{1}{n}\;\text{whenever }j> k_n\text{ and } \underline{x\in E^{n}_{k_n}}.\]
\textbf{Act 2.} \emph{Constructing $ A_\epsilon $.} \\
Now choose $ N \in \N $ such that
\[\sum_{n=N}^{\infty} \frac{1}{2^n} < \frac{\epsilon}{2}\]
and define
\begin{equation}
	\tilde{A_\epsilon} = \bintrs_{n=N}^{\infty} E_{k_n}^{n}
\end{equation}
We now observe that
\begin{equation*}
	\begin{split}
		\lm{E\setminus \tilde{A_{\epsilon}}} &= \lm{E \intrs \bunion_{n=N}^{\infty}\comp{\left (E_{k_n}^{n}\right )}}\\
		&= \lm{ \bunion_{n=N}^{\infty}E \intrs \comp{\left (E_{k_n}^{n}\right )}}\\
		&\le \sum_{n=N}^{\infty} \lm{E\intrs \comp{\left (E_{k_n}^{n}\right )}}\\
		&= \sum_{n=N}^{\infty} \lm{E\setminus E_{k_n}^{n}}\\
		&<\sum_{n=N}^{\infty} \frac{1}{2^n}\\
		&< \frac{\epsilon}{2}
	\end{split}
\end{equation*}
\textbf{Act 3.} \emph{Final\'e}.\\
We now claim and prove the following: 
\[\text{Claim : \textit{$ f_k \longrightarrow f $ uniformly on $ \tilde{A_\epsilon} $}.}\]
For this, let $ \delta > 0 $ and choose $ n^{\prime}\ge N $ such that $ \frac{1}{n^{\prime}} < \delta $. Then
\begin{equation}\label{T-3.4}
	\text{if } x\in \tilde{A_{\epsilon}} \implies x\in E_{k_{n^{\prime}}}^{n^{\prime}} \implies \abs{f_j(x) - f(x)} < \frac{1}{n^{\prime}} <\delta\;,\;\;\forall\;j > k_{n^{\prime}}.
\end{equation}
Note that this is just the definition of uniform convergence. \\
Finally, note that $ E_k^{n} $ is a Lebesgue measurable set due to Proposition \ref{P-18}, Statement 5. Hence, $ \tilde{A_\epsilon} $ is measurable. Now, by Proposition \ref{P-16}, Statement 2, there exists a closed set $ A_\epsilon \subset \tilde{A_\epsilon} $ such that \[\lm{\tilde{A_\epsilon} \setminus A_\epsilon} < \frac{\epsilon}{2}.\]
Now,
\begin{equation*}
	\begin{split}
	\epsilon &>	\lm{E\setminus \tilde{A_\epsilon} } + \lm{\tilde{A_\epsilon}\setminus A_\epsilon}\\
	&\ge \lm{E\setminus \tilde{A_\epsilon} \bunion \tilde{A_\epsilon}\setminus A_\epsilon}\\
	&= \lm{E\setminus A_\epsilon}.
	\end{split}
\end{equation*}
Now, by \eqref{T-3.4}, we see that $ f_k \longrightarrow f $ uniformly for all $ x\in A_\epsilon \subset \tilde{A_\epsilon} $ such that $ \lm{E\setminus A_\epsilon}< \epsilon $ and $ A_\epsilon $ is closed. Proof is now complete.
\end{proof}
\hrulefill
\subsection{Lusin's Theorem}
The following is the final important result on the basic theory of measurable functions, attributed to Nikolai Nikolaevich Luzin who penned this theorem around 1912.
\begin{theorem}
	(\textbf{Lusin's Theorem}) Consider the Lebesgue measure space $ (\R,\msigm{\lambda^*}, \lambda) $. Suppose $ f $ is a real-valued, Lebesgue Measurable function defined over a Lebesgue measurable set $ E $ with finite measure. Then for all $ \epsilon > 0 $, there exists a \textbf{closed} set $ F_{\epsilon} \subset E$ with 
	\begin{enumerate}
		\item {$ \lm{E\setminus F_{\epsilon}} < \epsilon $, and}
		\item {The restriction $ \restrict{f}{F_{\epsilon}} $ of $ f $ over $ F_{\epsilon} $ is \textbf{continuous}.}
	\end{enumerate}
\end{theorem}
\begin{proof}
	From the Proposition \ref{P-30}, we have a sequence $ \{f_n\} $ of step functions such that 
	\[f_n \longrightarrow f\;\text{almost everywhere.}\]
	Now, consider, for example the characteristic function over an interval $ \chi_{[a,b]} $. Then, we can define a function $ \phi(x) $ for any $ \delta > 0 $ as follows:
	\begin{equation*}
		\begin{split}
			\phi(x) &= \begin{cases}
				0 &\;\;x< a\\
				\frac{x-a}{\delta /2}&\;\; a\le x\le a+\frac{\delta}{2}\\
				1&\;\;a+\frac{\delta}{2} \le x \le b-\frac{\delta}{2}\\
				\frac{b-x}{\delta/2}& b-\frac{\delta}{2}\le x\le b\\
				0& \;\;x> b
			\end{cases}
		\end{split}
	\end{equation*}
which then satisfies 
\begin{equation*}
	\begin{split}
		\{x\in \R\where \phi(x) \neq \chi_{[a,b]}\} &= \left (a,a+\frac{\delta}{2}\right ) \bunion \left (b-\frac{\delta}{2},b\right )
	\end{split}
\end{equation*} 
which then implies that, 
\[\lm{\{x\in \R\where \phi(x) \neq \chi_{[a,b]}\}} = \lm{\left (a,a+\frac{\delta}{2}\right ) \bunion \left (b-\frac{\delta}{2},b\right )} = \delta.\]
Note that $ \phi(x) $ is also continuous over all $ \R $. Hence, for any step function (finite sum of $ \chi_{[a,b]} $-type functions) and $ \delta> 0 $, one can construct a continuous function which does not satisfies with the step function on a set with measure $ < \delta $.\\\\
Hence, for step-functions $ \{f_n\} $, corresponding to each $ f_n $, $ \exists $ a continuous function $ \phi_n $ and a set $ E_n $ such that
\[E_n = \{x\where \phi_n(x) \neq f_n(x)\}\;\text{and}\;\lm{E_n}<\frac{1}{2^{n}} .\]
Now, for all $ \epsilon > 0 $, there exists a $ N \in \N $ such that
\[\sum_{n\ge N} \frac{1}{2^{n}}< \frac{\epsilon}{3}.\]
With the above fact, construct the set $ F^{\prime} $ as follows:
\[F^{\prime} = \left (A_{\frac{\epsilon}{3}} \setminus \bunion_{n\ge N} E_n\right )\]
where $ A_{\frac{\epsilon}{3}} $ is the closed subset $ A_{\frac{\epsilon}{3}} \subset E$ such that \textbf{1.} $ \lm{E\setminus A_{\frac{\epsilon}{3}}} <\frac{\epsilon}{3}$ and \textbf{2.} $ f_n \longrightarrow f $ \textbf{uniformly} on $ A_{\frac{\epsilon}{3}} $. This is guaranteed by Theorem \ref{T-3} (Egorov's Theorem).\\\\
Note that $ \restrict{f_n}{F^{\prime}} $ is \textbf{continuous} $ \forall\;n\ge N $ because for any $ x\in F^{\prime} \implies \phi_n(x) = f_n(x)\;\forall\;n\ge N$ and since $ \phi_n $ are already continuous $ \forall n\in  \N $. \\\\
Furthermore, since $ F^{\prime} \subset E$ and $ f_n\longrightarrow f $ uniformly, then the restriction $ \restrict{f_n}{F^{\prime}} $ is continuous and converges uniformly to $ \restrict{f}{F^{\prime}} $, which \textbf{due to uniform convergence, is also continuous!}\\\\
Now, note that $ E_n $'s are measurable sets due to Proposition \ref{P-18}, Statement 5. Similarly, since $ A_{\frac{\epsilon}{3}} $ is closed, therefore it is also measurable. Hence, $ F^{\prime} $ is measurable.\\\\
Now by Proposition \ref{P-16}, there exists a closed set $ F_{\epsilon} \subset F^{\prime}$ such that $ \lm{F^{\prime}\setminus F_{\epsilon}} < \frac{\epsilon}{3} $. Note that because $ F_{\epsilon}\subset  F^{\prime} $ and $ \restrict{f}{F^{\prime}} $ is continuous, therefore the restriction $ \restrict{f}{F_{\epsilon}} $ is also continuous.\\\\
Finally, combining 
\begin{enumerate}
	\item {$ \sum_{n\ge N} \frac{1}{2^{n}} < \frac{\epsilon}{3}$,}
	\item {$ \lm{E\setminus A_{\frac{\epsilon}{3}}} < \frac{\epsilon}{3} $,}
	\item {$ \lm{F^{\prime} \setminus F_{\epsilon}} < \frac{\epsilon}{3} $}
\end{enumerate}
it can be easily seen that
\begin{equation*}
	\begin{split}
		\left (E\setminus A_{\frac{\epsilon}{3}}\right ) \bunion \left (F^{\prime} \setminus F_{\epsilon}\right ) = E\setminus F_{\epsilon}
	\end{split}
\end{equation*}
Hence,
 \begin{equation*}
	\begin{split}
		 \lm{E\setminus F_{\epsilon}}&= \lm{\left (E\setminus A_{\frac{\epsilon}{3}}\right ) \bunion \left (F^{\prime} \setminus F_{\epsilon}\right )} \\&\le \lm{\left (E\setminus A_{\frac{\epsilon}{3}}\right )} + \lm{ \left (F^{\prime} \setminus F_{\epsilon}\right )} \\
		 &< \frac{2\epsilon}{3}\\
		 &< \epsilon.
	\end{split}
\end{equation*}
which completes the proof.
\end{proof}
\hrulefill
\newpage
\section{Integration of Measurable Functions}
Let's first remind ourselves of the basic definition of a Riemann Integrable function. If we say that the function $ f : \R \to \R $ is Riemann Integrable, then the integral of $ f $ on $ [a,b] $, written as $ \int_a^{b} f $, is given by the following two constructions on a partition $ P $ of $ [a,b] $,
\begin{itemize}
	\item {Lower Sum :
	\[L(f,P) = \sum_i m_i (a_i - a_{i-1}) \;\text{where $ m_i = \inf_{x\in [a_{i-1},a_i] }f(x)  $}\]}
\item {Upper Sum:
\[U(f,P) = \sum_i M_i (a_i - a_{i-1}) \;\text{where $ M_i = \sup_{x\in [a_{i-1},a_i] }f(x)  $}\]
}
\end{itemize}
so that 
\[\int_a^{b} f = L(f,P) = U(f,P).\]
The chain of observations begins now. One can easily write the Lower and Upper Sum as the following simple functions (remember that the partition is finitely many) 
\begin{equation*}
	\begin{split}
		L(f,P) &= \sum_i m_i \lm{[a_{i-1},a_i]}\\
		U(f,P) &= \sum_i M_i \lm{[a_{i-1},a_i]}
	\end{split}
\end{equation*}
Or, equivalently, we can define a lower step function as follows:
\begin{equation*}
	\begin{split}
		\phi_P = \sum_i m_i \chi_{[a_{i-1},a_i]}
	\end{split}
\end{equation*}
so that the Riemann Integral is simply
\[\int_a^b f(x)dx = \sup_{P} \int_a^b \phi_P(x)dx\]
where supremum is defined over all partitions. But since, by definition, $ \phi_P(x) \le f(x) \;\forall\;x\in \R$, we can alternatively define Riemann Integral as
\begin{equation}\label{Eq 15}
	\begin{split}
		{\int_a^b f(x)dx = \sup_{\phi_P \le f} \int_a^b \phi_P(x)dx}
	\end{split}
\end{equation}
where the supremum is defined for all step functions on any partition $ P $.\\\\
 This definition presented in \eqref{Eq 15} provides the motivation for extending the definition of Integration from Riemann to Lebesgue. In particular, note the definition of $ \phi_P$, usual measure on the intervals is applied in Riemann's definition. But, since we know that Borel $ \sigma $-algebra is a proper subset of $ \msigm{\lambda^*} $, then it just makes sense to replace $ a_{i-1} - a_i $ by $ \lm{[a_{i-1},a_i]} $ in the motivation that it might generalize the notion of integration.\\\\
  Therefore, the following definitions of integration for non-negative functions are motivated:
 \newpage
 \subsection{Integration of Non-Negative Measurable Functions}
 \begin{definition}
 	(\textbf{Lebesgue Integral of a Simple Function}) Consider $ \phi: \R \to [0,+\infty) $ be a simple function as
 	\[\phi = \sum_{i=1}^{N} \alpha_i \chi_{E_i}\;\text{where $ \alpha_i \ge 0 $ and $ \lm{E_i} < +\infty $}\]
 	Then, the Lebesgue integral of $ \phi $ is defined as
 	\[\int \phi dx = \sum_{i=1}^{N}\alpha_i \lm{E_i}\]
 \end{definition}
\hrulefill

\begin{definition}
	(\textbf{Lebesgue Integral of a Measurable Function}) Suppose $ f : \R\to [0,+\infty) $ is a $ \lambda $-Measurable function, then the Lebesgue Integral of $ f $ is defined as
	\begin{equation*}
		\begin{split}
		\bigstar\;\;	\boxed{\int f dx = \sup_{\phi\le f} \int \phi dx\;\text{where $ \phi $ are the \textbf{simple} functions $ \le f $.}}\;\; \bigstar
		\end{split}
	\end{equation*}
\end{definition}
\hrulefill
\begin{definition}
	(\textbf{Lebesgue Integral over a measurable set}) Consider $ f : \R \to [0,+\infty) $ to be a $ \lambda $-Measurable function and $ E\subseteq \R $ is Lebesgue measurable. Then,
	\[\int_E f dx = \int f \cdot\chi_E dx\]
\end{definition}
\begin{remark}
	Therefore, the integral of a non-negative measurable function over a measurable set is given by the integral\footnote{From now on, any instance of \emph{integral} should be presupposed by Lebesgue integral, of-course, unless otherwise stated, in this text.} of restriction of $ f $ to it and zero otherwise.
\end{remark}
\hrulefill
\begin{proposition}\label{P-33}
	Consider the two $ \lambda $-Measurable functions $ f,g : \R \to [0,+\infty) $ and $ \phi: \R \to [0,+\infty) $ be a simple-function, then the Lebesgue Integral has the following properties:
	\begin{enumerate}
		\item {Consider two Lebesgue measurable subsets $ A $ and $ B $ of $ \R $ such that $ A\intrs B = \Phi $. Then,
	\[\int_{A\union B}\phi dx = \int_A \phi dx + \int_B \phi dx.\]	
	}
		\item {For any $ \alpha \in \R $,
	\[\int \alpha fdx = \alpha \int fdx.\]	
	}
\item {Integration for positive valued Measurable functions is therefore distributive:
\[\int (f+g)dx = \int fdx + \int gdx.\]
}
\item {If $ f(x) \le g(x) $ holds for all $ x\in \R $, then 
\[\int fdx \le \int gdx.\]
}
\item {Consider $ A $ and $ B $ be Lebesgue measurable subsets of $ \R $ such that $ A\subseteq B $. Then,
\[\int_A f dx \le \int_B fdx.\]
}
	\end{enumerate}
\end{proposition}
\begin{proof}
	\textbf{Part 1} : Since $ \phi $ is simple, therefore we can write
	\[\phi = \sum_{i=1}^N \alpha_i \chi_{E_i}.\]
	Now, by definition
	\begin{align*}
		\int_{A\union B} \phi dx &= \sum_{i=1}^{N} \alpha_i \lm{E_i \intrs (A\union B)}\\
		&= \sum_{i=1}^{N} \alpha_i \lm{(E_i \intrs A) \union (E_i \intrs B)}\\
		&= \sum_{i=1}^{N} \alpha_{i} \lm{E_i \intrs A} + \alpha_i \lm{E_i \intrs B} &&\text{$ \because E_i\intrs A$ and $ E_i \intrs B $ are disjoint.}\\
		&= \int_{A} \phi dx + \int_{B} \phi dx
	\end{align*}
\textbf{Part 2 \& 3} : Can be seen easily from Theorem \ref{T-5}.\\\\
\textbf{Part 4} : Note that we define
\[\int fdx = \sup_{\phi \le f} \int \phi dx\text{ where $ \phi $ are simple functions.}\]
Therefore, for any $ \phi \le f $, due to given condition $ f\le g $, we would have $ \phi \le g $. Hence,
\begin{align*}
	\int \phi dx \le \int g dx 
\end{align*}
Since this is true for all simple $ \phi \le f $, therefore $ \sup_{\phi \le f} \int \phi dx \le \int gdx $, proving the result.\\\\
\textbf{Part 5} : Consider the following:
\begin{align*}
	\int_{A} f dx &= \int f\chi_{A} dx\\
	&\le \int f\chi_{B} dx &&\text{$ \because \chi_{A} \le \chi_{B}$, then apply 4.}\\
	&= \int_{B} f dx
\end{align*}
Hence proved.
\end{proof}
\subsection{Monotone Convergence Theorem}
This is arguably one of the most important theorem in Integration theory,
\begin{theorem}\label{T-5}
	(\textbf{Monotone Convergence Theorem}) Consider a sequence $ \{f_n\} $ of $ \R\to [0,+\infty)  $ of $ \lambda $-Measurable functions which satisfies
	\[f_n(x) \le f_{n+1}(x)\;\forall\;x\in \R \text{ and }n\]
	 and suppose $ \lim_{n\to \infty} f_n $ exists. Then,
	\[\int \lim_{n\to \infty} f_n = \lim_{n\to \infty} \int f_n.\]
\end{theorem}
\begin{proof}
	Since $ f_n \le f_{n+1} $, therefore,
	\[\int f_n \le \int f_{n+1}\le \int \lim_{n\to \infty} f_n.\]
	Hence,
	\[\lim_{n\to \infty} \int f_n \le \int \lim_{n\to \infty} f_n. \]
	Therefore we have proved one inequality.\\\\
	 Now to prove the other inequality, consider any simple function $ \phi \le \lim_{n\to \infty} f $. If we can show that $ \int f_n \ge \int \phi $ for any $ n \in \N $, then we are done. To this goal, consider $ \alpha \in (0,1) $. Construct the set\footnote{After reading the proof, it should appear striking to the reader on actually how much the proof depends on this construction. Both the claims in the following page utilizes this construction $ E_n $ to full extent! Hence, it is advised (by Instructor) to purse such effective constructions in the problem sheets and your own proofs.}
	\[\boxed{E_n = \{x\where f_n(x) \ge \alpha \phi(x)\}}.\]
	Clearly,
	\[E_n \subseteq E_{n+1}\;\forall\;n\in \N.\]
	Now,
	\begin{equation}\label{T-5.1}
		\int f_n \ge \int_{E_n} f_n \ge \alpha \int_{E_n}\phi.
	\end{equation}
	Moreover, we can see that
	\[\text{Claim 1 : }\bunion_{n} E_n = \R.\]
	This is easy to see as follows:
	\begin{align*}
		\text{Take }x\in \bunion_n E_n&\implies x\in E_{i_0} \text{ for some }i_0 \in \N.\\
		&\implies x\in \R &&\because\text{ $ E_n $ are subsets of $ \R $.}\\
		\text{Take }x\in \R &\implies \text{Either }(1)\; x\in \{x\where f_n(x) - \alpha \phi(x) \ge  0\} \text{ or } (2) \;x\in \{x\where f_n(x) - \alpha \phi(x) < 0\}  &&\text{for any $ n\in \N $.}\\
		&\implies \text{If $ (1) $, then $ x\in E_n $, else if $ (2) $, then }\because \phi \le \lim_{n\to \infty}f_n, \;\exists \; n^{\prime} \;s.t.\; x\in E_{n^{\prime}}\\
		&\implies x\in \bunion_{n} E_n. && \text{Hence Claim 1.}
	\end{align*}
Next, we can also see that
\[\text{Claim 2 : }\int_{E_n}\phi \longrightarrow \int \phi\]
This can be seen by expanding the \emph{simplicity} of $ \phi $ as follows:
\begin{align*}
\lim_{n\to \infty}	\int_{E_n} \phi &= \sum_{i=1}^{N} a_i\lim_{n\to \infty} \lm{A_i\intrs E_n}\\
	&= \sum_{i=1}^{N} a_i \lm{\bunion_{n} A_i \intrs E_n} &&\because \text{ $ \{A_i\intrs E_n\}_n $ is increasing, use Proposition \ref{P-7}.}\\
	&= \sum_{i=1}^{N} a_i \lm{A_i \intrs \bunion_n E_n}\\
	&= \sum_{i=1}^{N}a_i \lm{A_i \intrs \R} &&\text{Claim 1.}\\
	&= \sum_{i=1}^{N} a_i \lm{A_i} = \int \phi &&\text{Hence Claim 2.}
\end{align*}
Finally, take limit in \eqref{T-5.1} to get:
\begin{align*}
	\lim_{n\to \infty} \int f_n \ge \lim_{n\to \infty} \int_{E_n}f_n &\ge \alpha\lim_{n\to \infty} \int_{E_n} \phi &&\\
	&=\alpha \int \phi&&\text{Claim 2.}\\
	\lim_{n\to \infty} \int f_n &\ge \int \phi &&\because\text{ $ 0<\alpha <1 $ is arbitrary.}
\end{align*}
Hence, \textbf{for any} simple function $ \phi  \le \lim_{n\to \infty}f_n$, we have concluded that $ \int \phi \le \lim_{n\to \infty} \int f_n  $, hence it must be true that
\begin{align*}
	\int \lim_{n\to \infty} f_n = \sup_{\phi \le \lim_{n} f_n} \int \phi &\le \lim_{n\to \infty}\int f_n.
\end{align*}
Combining the converse inequality at the beginning, we hence get the desired result.
\end{proof}
\hrulefill
\begin{proposition}\label{P-34}
	Consider a Lebesgue Measurable function $ f : \R\to [0,+\infty) $. Then,
	\[ \int f dx = 0 \mathbf{\iff} f \equiv 0 \;\text{almost everywhere.} \]
\end{proposition}
\begin{proof}
	\textbf{L $ \implies $ R} : Consider $ f $ is a non-negative real-valued function whose integral is zero. Construct the set,
	\[E_n = \left \{ x\where f(x) \ge \frac{1}{n}\right \}.\]
	In order to show that $ f \equiv 0$ almost everywhere, it is hence sufficient to show that $ \lm{E_n} = 0 \;\forall\; n\in\N$ because it equivalently proves that the measure of the set where $ f $ is greater than zero is zero.\\
	Now, consider the following function
	\[g_n = \frac{1}{n} \chi_{E_n}.\]
	Clearly, because $ g_n $ is a simple function and 
	\begin{equation*}
		\begin{split}
			\frac{1}{n} \chi_{E_n}(x) = \begin{cases}
				\frac{1}{n}&\text{ if } f(x) \ge \frac{1}{n}\\
				0 &\text{otherwise}
			\end{cases}
		\end{split}
	\end{equation*}
which clearly means that $ \frac{1}{n}\chi_{E_n} \le f $, therefore,
	\begin{align*}
		\int f = 0 &= \sup_{\phi \le f} \int \phi\\
		&\ge \int \frac{1}{n} \chi_{E_n}\\
		&= \frac{1}{n} \lm{E_n}\\
		\implies & \lm{E_n} = 0\; \forall \;n\in \N.
	\end{align*}
\textbf{R $ \implies  $ L} : If a non-negative real-valued Measurable function $f $ is $ 0 $ almost everywhere, then for any simple function $ \phi \le f $, $ \phi $ must also be $ 0 $ almost everywhere, so that 
\begin{equation*}
	\begin{split}
		\int \phi &= \sum_{i = 1}^{N} \alpha_{i} \lm{E_i}\\
		&= 0
	\end{split}
\end{equation*}
Since this is true for any simple $ \phi \le f$, therefore the supremum of all such $ \int\phi $ must also be zero, to make $ \int f = 0 $. 
\end{proof}
\hrulefill

\newpage
\subsection{Fatou's Lemma}

\begin{theorem}\label{T-6}
	If $ \{f_n\} $ is a sequence of Lebesgue Measurable functions from $ \R $ to $ [0,+\infty) $, then,
	\[\int \liminf_n f_n \le \liminf_n \int f_n.\]
\end{theorem}
\begin{proof}
	We will use Monotone Convergence Theorem to prove this result. Define
	\[g_k = \inf_{n\ge k} f_n\]
	Therefore $ g_k \le g_{k+1} $ with $ g_k \le f_n \forall\; n\ge k$. Then,
	\[\int g_k \le  \int f_n \forall\; n\ge k.\]
	This implies that 
	\[\int g_k \le \inf_{n\ge k} \int f_n.\]
	Now, by MCT, 
	\[\int \lim_k g_k = \lim_k \int g_k\]
	Therefore 
	\begin{align*}
		\lim_{k\to \infty} \inf_{n\ge k} \int f_n = \liminf_k \int f_k &\ge \lim_{k\to \infty} \int g_k\\
		&= \int \lim_{k\to \infty} g_k\\
		&= \int \lim_{k\to \infty} \inf_{n\ge k} f_n\\
		&= \int \liminf_k f_k
	\end{align*}
Hence Proved.
\end{proof}
\begin{remark}
	In fact,
	\[\text{\textbf{Fatou's Lemma} $ \iff $ \textbf{Monotone Convergence Theorem}.}\]
\end{remark}
\hrulefill
\newpage
\subsection{Integration of General Real-Valued Measurable Functions}
With the notion of integration of non-negative Measurable function in place, it's not difficult to see how can one extend the same notion to Measurable functions which takes value in the whole real line.
\begin{definition}
	(\textbf{Lebesgue Integral of a Real-Valued Measurable Function}) Consider $ f: \R \to \R $ to be a Measurable function such that 
	\begin{enumerate}
		\item {$ \int f^{+}dx < \infty $, and}
		\item {$ \int f^{-}dx < \infty $.}
	\end{enumerate}
If the above two conditions are satisfied, then $ f $ is called Lebesgue Integrable. Then, the Lebesgue Integral of $ f $ is defined as
\[\bigstar \;\boxed{\int fdx = \int f^{+} dx - \int f^{-} dx} \;\bigstar \]
\end{definition}
\begin{remark}
	It's important to note that the Integral $ \int fdx = \int f^{+} dx - \int f^{-} dx $ is easily defined for any measurable function, but $ f $ is called Lebesgue Integral only when it's value is finite!
\end{remark}
\hrulefill
\begin{definition}
	(\textbf{Lebesgue Integral over a Measurable Set}) Consider $ f : \R \to \R $ is Measurable, $ f\cdot \chi_{E} $ is an Lebesgue Integrable function and $ E \subseteq \R $ is also measurable. Then, 
	\[\int_{E} f dx = \int f\cdot \chi_{E} dx .\]
\end{definition}
\hrulefill
\subsubsection{Basic Properties of General Lebesgue Integral}
The following properties are direct extensions of Proposition \ref{P-33} to the bigger class of Lebesgue Integrable functions.
\begin{proposition}\label{P-35}
	Consider $ f,g : \R\to \R $ to be Lebesgue Integrable functions. Then,
	\begin{enumerate}
		\item {For any $ \alpha \in \R $, we have:
	\[\int \alpha fdx = \alpha\int fdx.\]	
	}
\item {$ f + g $ is also Lebesgue Integrable, with
\[\int (f+g)dx = \int fdx + \int gdx.\]
}
\item {If $ f \equiv 0 $ almost everywhere on $ \R$, then,
\[\int fdx = 0.\]
}
\item {If $ f\le g $ almost everywhere on $ \R $, then,
\[\int fdx \le \int gdx.\]
}
\item {If $ A $ and $ B $ are measurable sets such that $ A\intrs B = \Phi $, then,
\[\int_{A\union B} fdx = \int_{A} fdx + \int_{B} fdx.\]
}
	\end{enumerate}
\end{proposition}
\begin{proof}
	\textbf{S1} : Consider the case that $ \alpha \ge 0 $. Then,
	\begin{align*}
		(\alpha f)^{+} &= \max (\alpha f, 0) = \alpha \max(f,0) =\alpha f^{+}\\
		(\alpha f)^{-} &= -\min(\alpha f,0) = -\alpha \min (f,0 = \alpha f^{-} 
	\end{align*}
and since $ \int f^{+}dx <\infty$ and $ \int f^{-} dx < \infty$, therefore $ \alpha f $ is also Lebesgue Integrable, with the integral given as
\begin{align*}
	\int \alpha f &= \int \alpha f^{+} - \int \alpha f^{-}\\
	&= \alpha \left (\int f^{+} - \int f^{-}\right )\\
	&= \alpha \int f
\end{align*}
Now consider that $ \alpha  < 0 $, then
\begin{align*}
	(\alpha f)^{+} &= \max(\alpha f, 0) = -\abs{\alpha} \min(f,0) = \abs{\alpha} f^{-}\\
	(\alpha f)^{-} &= -\min(\alpha f, 0) = \abs{\alpha} \max (f,0) = \abs{\alpha} f^{+} 
\end{align*}
Hence, $ \alpha f $ is again Lebesgue Integrable, with the integral calculated as:
\begin{align*}
	\int \alpha f = \int (\alpha f)^{+} - \int (\alpha f)^{-} = \abs{\alpha} \left (\int f^{-} - \int f^{+}\right ) = -\abs{\alpha} \int f = \alpha \int f.
\end{align*}
\textbf{S2} : First,
\begin{align*}
	(f+g)^{+} &\le f^{+} + g^{+} \\
	(f+g)^{-} &\le f^{-} + g^{-}
\end{align*}
for all $ x\in \R $, so that $ f+g $ is Lebesgue Integrable. Now,
\begin{align*}
	f + g &= (f+g)^{+} - (f+g)^{-} \\
	&= f^{+} - f^{-} + g^{+} - g^{-}
\end{align*}
therefore,
\begin{align*}
	(f+g)^{+} - (f+g)^{-} &= f^{+} - f^{-} + g^{+} - g^{-}\\
	(f+g)^{+} + f^{-} + g^{-} &= (f+g)^{-} + f^{+} + g^{+}\\
	\int 	(f+g)^{+} + f^{-} + g^{-} &=  \int (f+g)^{-} + f^{+} + g^{+}\\
	\int	(f+g)^{+} + \int f^{-} + \int g^{-} &=\int (f+g)^{-} + \int f^{+} + \int g^{+}&&(\text{$ \because $ of Proposition \ref{P-33}, S3.})\\
\int	(f+g)^{+} -\int (f+g)^{-} &=  \int f^{+}  - \int f^{-}+ \int g^{+} - \int g^{-}\\
\int	(f+g)^{+} -\int (f+g)^{-} &=  \int f + \int g
\end{align*}
\textbf{S3} : Given to us is that $ f \equiv  0 $ almost everywhere. This means that 
\[\{x\in \R \where f(x) \neq 0\} \;\text{is of measure }0.\]
We can write it equivalently as the union of the following two disjoint sets
\begin{align*}
	\{x\in \R \where f(x) \neq 0\} &= \{x\where f(x) > 0\} \union \{x\in \R \where f(x) < 0\}.\\
	\lm{	\{x\in \R \where f(x) \neq 0\}}&=  \lm{\{x\where f^{+}(x) > 0\}} + \lm{\{x\where f^{-}(x) < 0\}}\\
	&= 0
\end{align*}
Since measure is positive valued by definition, therefore, these two have to be individually be zero. That is,
\[\lm{\{x\where f^{+}(x) > 0\}} = \lm{\{x\where f^{-}(x) < 0\}} = 0\]
Now, by Proposition \ref{P-34}, we get that
\[\int f^{+} = \int f^{-} = 0\]
which implies that
\[\int f = \int f^{+} - \int f^{-} = 0.\]
\textbf{S4} : 
\end{proof}
\hrulefill
\begin{proposition}
	If $ f : \R \to \R $ is a Lebesgue Integrable Function, then,
	\[\abs{\int fdx} \le \int \abs{f} dx\]
\end{proposition}
\begin{proof}
Simply note the following:
\begin{align*}
	\abs{\int fdx} &= \abs{\int \left ( f^{+} - f^{-}\right )dx}\\
	&= \abs{\int f^{+}dx - \int f^{-}dx}\\
	&\le \abs{\int f^{+}dx} + \abs{\int f^{-}dx}\\
	&= \int f^{+} dx + \int f^{-}dx\\
	&= \int \left (f^{+} + f^{-}\right )dx\\
	&= \int \abs{f}dx.
\end{align*}	
\end{proof}
\hrulefill
\newpage
\subsection{Dominated Convergence Theorem}
\begin{theorem}\label{T-7}
	Let $ \{f_n\} $ be a sequence of Measurable functions such that there exists a Lebesgue Integrable function $ g $ which satisfies
	\[\abs{f_n} \le g \;\forall\;n.\]
	Suppose that the limit $ \lim_{n\to \infty} f_n $ exists. Then $ \lim_{n\to \infty} f_n $ is Lebesgue Integrable and, 
	\[\lim_{n\to \infty} \int f_n dx = \int \lim_{n\to \infty} f_n dx.\]
\end{theorem}
\begin{proof}
	Since $ \{f_n\} $ is a sequence of Measurable functions, therefore, $ \lim_{n\to \infty} f_n = f $ is also Measurable and $ \abs{f} $ is bounded by $ g $. But since $ g $ is Lebesgue Integrable, and $ f_n $ and $ f $ are bounded by $ g $, then each $ f_n $ and also $ f$ are also Lebesgue Integrable (Trivial to see). \\
	Now, $ \{g + f_n\} $ is a sequence of Measurable functions. Moreover, since $ f_n \le g $ for all $ n $, therefore $ f_n + g \ge 0 $, so that $ \{f_n + g\} $ is a sequence of non-negative Measurable functions.\\
	 Now using Fatou's Lemma (Theorem \ref{T-6}), we get,
	\begin{align*}
		\int \liminf_{n} (g+ f_n) dx &\le \liminf_{n} \int (g + f_n) dx\\
		\int \left (g + \liminf_{n} f_n\right ) dx &\le  \int g dx+ \liminf_{n} \int f_n dx\\
		\int gdx + \int \liminf_{n} f_n dx&\le \int gdx + \liminf_{n} \int f_n dx\\
		\int \liminf_{n}f_n dx&\le \liminf_{n} \int f_n dx &&\text{$ \because g$ is L.I., so $ \int gdx < \infty $}\\
		\int f dx &\le \liminf_{n}\int f_n dx&&\text{$ \because \limsup_{n} x_n = \liminf_{n} x_n = \lim_{n} x_n $.}
	\end{align*}
Similarly, since $ \{g - f_n\} $ is also a sequence of non-negative Measurable functions, therefore we can use Fatou's Lemma to conclude:
\begin{align*}
	\int \liminf_{n} (g-f_n) dx &\le \liminf_{n} \int (g-f_n)dx\\
	\int \liminf_{n} \left (-f_n\right )dx&\le \liminf_{n} \left (-\int f_ndx \right )\\
	-\int \limsup_{n} f_n dx&\le - \limsup_{n} \int f_n dx &&\text{$ \because \liminf_{n} (-x_n) = - \limsup_{n} x_n$.}\\
	\int f dx & \ge \limsup_{n} \int f_n dx
\end{align*}
We hence have that 
\[\limsup_{n} \int f_n dx \le \int fdx \le \liminf_{n} \int f_n dx\]
But it is also true that 
\[  \liminf_{n} \int f_n dx \le \limsup_{n}\int f_n dx.\] 
Hence,
\[\liminf_{n} \int f_n dx = \limsup_{n} \int f_n dx = \lim_{n} \int f_n dx = \int fdx\]
Hence proved.
\end{proof}
\hrulefill
\newpage
\begin{proposition}
	Consider $ \{f_n\} $ to be a sequence of Lebesgue Integrable functions such that
	\[\sum_{n=1}^{\infty} \int \abs{f_n}dx < \infty.\]
	Then, 
	\begin{enumerate}
		\item {The series
		\[\sum_{n=1}^{\infty} f_n(x) \text{ converges almost everywhere on $ \R $.}\]}
	\item {The sum
\[f = \sum_{n=1}^{\infty}f_n\text{ is Lebesgue Integrable.}\]	
}
\item {The integral is
\[\int \sum_{n=1}^{\infty}f_ndx = \sum_{n=1}^{\infty} \int f_n dx.\]
}
	\end{enumerate}
\end{proposition}
\begin{proof}
	\textbf{S1}. Denote the following:
	\[\varphi = \sum_{n=1}^{\infty} \abs{f_n}\]
	Clearly, $ \varphi $ is a non-negative Measurable function. Since we know that Lebesgue Integral for non-negative functions is countably additive, therefore,
	\[\int \varphi dx = \sum_{n=1}^{\infty} \int \abs{f_n} dx < \infty.\]
	Now, if $ \int \varphi < \infty$, then $ \varphi $ is finite almost everywhere on $ \R $\footnote{For a non-negative Measurable function $ f $ with given that $ \int fdx < \infty $, the set $ E =\{x\in \R\where f(x) = \infty\} $ together with supposition that $ \lm{E} > 0 $ is such that; since $ \int fdx = \sup_{\phi \le f} \int \phi $, therefore, if we take $ \phi = n\chi_{E} $ for any $ n > 0$ then $ n\chi{E} < f$. Hence $ \int fdx > n\lm{E} $ for all $ n $, so that $ \int fdx = \infty $. But it's a contradiction to $ \int fdx <\infty $. Therefore $ \lm{E} = 0 $.}. Now, since $ \sum_{n=1}^{\infty}f_n $ is absolutely convergent almost everywhere (last line), hence it is convergent almost everywhere too on $ \R $.\\\\
	\textbf{S2}. Since $ \abs{\sum_{n=1}^{\infty} f_n} \le \sum_{n=1}^{\infty} \abs{f_n} = \varphi < \infty \;(\text{almost everywhere})$ and since we can modify the set where $ \varphi $ is not defined (infinite) arbitrarily to make a new function which would be Measurable and equal to $ \sum_{n=1}^{\infty} f_n $ almost everywhere, therefore $ \sum_{n=1}^{\infty}f_n $ would be Measurable.\\\\
	\textbf{S3}. Define
	\[\phi_{n} = \sum_{i=1}^{n} f_i.\]
	Clearly, $ \phi_{n} \le \abs{\sum_{i=1}^{n}f_i} \le \sum_{i=1}^{n} \abs{f_i} \le \sum_{i=1}^{\infty} \abs{f_i} = \varphi$. Therefore, $ \phi_n $ is a sequence of Measurable functions and $ \phi_n \le \varphi $ where $ \varphi $ is an Integrable function (given). Therefore, using Dominated Convergence Theorem (\ref{T-7}), we get, 
	\begin{align*}
		\lim_n \int \phi_n dx &= \int \lim_{n}\phi_n dx\\
		\lim_{n} \int \sum_{i=1}^{n} f_i dx&= \int \lim_{n} \sum_{i=1}^{n} f_i dx\\
		\lim_{n}\sum_{i=1}^{n}\int f_i dx &= \int \sum_{i=1}^{\infty} f_i dx&&\text{$ \because $ Proposition \ref{P-35}, S2 }\\
		\sum_{i=1}^{\infty} \int f_i dx &= \int \sum_{i=1}^{\infty} f_i dx.
	\end{align*}
\end{proof}
Hence proved.

\hrulefill
\newpage
\section{The $ L^{p}$ Spaces}
We now turn into some more abstract formulation for analysis of measurable functions, by analyzing their class formed under certain definitions.
\begin{definition}
	(\textbf{$ L^{p} $ Norm of a function}) Consider any function $ f $ and $ p>0 $. The $ L^{p} $ norm of $ f $, denoted $ \norm{f}_p $, is defined as:
	\[\norm{f}_p = \wnorm{p}{f}\]
\end{definition}
\hrulefill
\begin{definition}
	(\textbf{The $ L^{p}$ Space}) Consider $ (X,\alg{A},\mu) $ to be a measure space. Suppose $ p> 0 $. Then, the class of measurable functions defined as:
	\[\Lp{X,\alg{A},\mu} = \left \{f : X \to \R \where \norm{f}_p < \infty \right \}/(f\sim g \iff f=g\;\text{a.e.}).\]
	Moreover, two measurable functions $ f,g \in \Lp{X,\alg{A},\mu} $ are said to be equal if and only if:
	\[f = g\;\text{almost everywhere on $ \R $.}\]
\end{definition}
\begin{remark}
	Note that $ \Lp{X,\alg{A},\mu} $ is just the class of Integrable functions when $ p=1 $.
\end{remark}
\begin{remark}
	Note carefully the use of word \emph{class} rather than set. It is because that an element of $ \Lp{X,\alg{A},\mu} $ is not a function, but a class of functions identified by the relation $ f \sim g $ if and only if $ f = g $ almost everywhere. But for out purposes, one can get away by writing $ f \in  \Lp{X,\alg{A},\mu} $ to mean that $ f $ is Measurable and $ \norm{f}_p < \infty $ so that $ \abs{f}^{p}$ is Integrable.
\end{remark}
%\footnote{A class, naively speaking, is supposed to be that collection which cannot be a set, otherwise it would lead to contradictions. The set of such functions would be "too big"/"will cause paradoxes" to be a set, hence it is called a class. Perhaps a more concrete notion would be developed in a set theory course, not now.}
\hrulefill
\subsection{Algebraic Properties of $ L^{p} $ Space}
We will now see some of the properties of $ L^{p} $ Spaces which reflects it's algebraic nature. In particular, we would prove that $ L^{p} $ is a vector space for any $ p> 0 $. But proving that $ \norm{\cdot }_{p} $ is actually the norm for functions in $ L^{p} $ ($ p\ge 1 $) would require a lot of construction.
\subsubsection{$ L^{p} $ is a Vector Space}
\begin{proposition}\label{P-38}
	Consider $ (X,\alg{A},\mu) $ to be a measure space. Then, the $ L^{p} $ space
	\[\Lp{X,\alg{A},\mu} \text{ is a \textbf{Vector Space}}.\]
\end{proposition}
\begin{proof}
	First, let's deal with the scalar multiplication. Note that the ground field here is $ \R $. For any $ a,b\in \R $ and $ f,g \in \Lp{X,\alg{A},\mu} $, we trivially have:
	\begin{align*}
		(ab)f &= a(bf)\\
		1f &= f\\
		a(f + g) &= af + ag\\
		(a+b)f &= af+ bf
	\end{align*}
Now, to show that $ L^{p}(X,\alg{A},\mu) $ is an abelian group under addition, the associativity, commutativity, identity ($ f $ such that $ f = 0 $ a.e.) and inverse (for $ f $, $ -f $ is the inverse) follows trivially. What remains to be shown is that for $f,g \in \Lp{X,\alg{A},\mu}  $, $ f + g \in \Lp{X,\alg{A},\mu} $ too. To see this, note that we know already, that $ f +g $ is Measurable, what we need to then show is that 
\[\text{To Show : }\norm{f+ g}_p < \infty\]
for any $ p > 0 $. All we need to show is therefore,
\[\int \abs{f + g}^{p} < \infty\]
To see this, note:
\begin{align*}
	\abs{f + g}^{p} &\le \left (\abs{f} + \abs{g}\right )^{p}\\&\le 2^{p}\max\left (\abs{f}^{p},\abs{g}^{p}\right )\\
	&\le 2^{p}\left (\abs{f}^{p} + \abs{g}^{p}\right )
\end{align*}
By Proposition \ref{P-35} S4,
\[\int \abs{f+ g}^{p} \le 2^{p}\left ( \int \abs{f}^{p} + \int \abs{g}^{p}\right) < \infty\]
Therefore, $ \Lp{X,\alg{A},\mu} $ is a Vector Space.
\end{proof}
\hrulefill

\newpage
\subsubsection{Norm on $ L^{p} $ Vector Space}
We first see that the norm defined at the beginning is actually not a norm in the case when $ p< 1 $. Therefore, $ L^{p} $ Vector Space with norm $ \norm{\cdot }_{p} $ would make sense only when $ p \ge 1 $.
\begin{definition}
	(\textbf{Norm on a Vector Space}) Consider a Vector Space $ (V,\R) $. A norm $ \norm{\cdot} $ on $ V $ is a function 
	\[\norm{\cdot } : (V,\R) \to [0,\infty)\]
	satisfying following three conditions:
	\begin{enumerate}
		\item {For any $ x\in (V,\R) $,
	\[\norm{x} = 0\;\iff\;x = 0_{V}\]	
	}
\item {For any $ x\in (V,R) $ and $ \alpha \in \R $,
\[\norm{\alpha x} = \abs{\alpha}\norm{x}\]
}
\item {For any $ x,y \in (V,\R) $
\[\norm{x + y}\le \norm{x} + \norm{y}\]
}
	\end{enumerate}
\end{definition}
\hrulefill
\\
Now suppose $ 0<p<1 $, then, it is simple to see $ \norm{\cdot}_p $ does not follow Triangle Inequality on $ \Lp{X,\alg{A},\mu} $. To see this, note that for any $ a,b > 0 $ and $ p\in (0,1) $, we have:
\begin{equation}\label{E-17}
		a^{p} + b^{p}  > (a+b)^{p}
\end{equation}
This comes naturally from the relation:
\[t^{p-1} > (a+t)^{1-p}\]
and then it's integration.\\
Using \eqref{E-17}, we can see that for any two sets $ E,F \in \alg{A} $ such that $ E\intrs F = \Phi $, if we write 
\begin{align*}
	a&= \m{E}^{1/p} = \wnorm{p}{\chi_{E}} = \norm{\chi_{E}}_{p}\\
	b&= \m{F}^{1/p} = \wnorm{p}{\chi_{F}} = \norm{\chi_{F}}_{p}
\end{align*}
then, 
\begin{align*}
	\norm{\chi_{E} + \chi_{F}}_{p}&= \wnorm{p}{\chi_{E} + \chi_{F}}\\
	&= \left (\int \left (\chi_{E} + \chi_{F}\right )^{p}\right )^{\frac{1}{p}}\\
	&= \left (\int \left (\chi_{E}^{p} + \chi_{F}^{p} \right) \right )^{\frac{1}{p}}&&\text{$ \because \chi_{E} \cdot \chi_{F} = \chi_{E\intrs F} = \chi_{\Phi} = 0$ }\\
	&= \left (\int \chi_{E}^{p} + \int \chi_{F}^{p}\right )^{\frac{1}{p}}\\
	&=\left (a^{p} + b^{p}\right )^{\frac{1}{p}}  \\
	&> a+b &&\text{Take power $ \frac{1}{p}$ both sides of Eq. \eqref{E-17}}\\
	&= a+b\\
	&= \norm{\chi_{E}}_p + \norm{\chi_{F}}_p
\end{align*}
Hence, there exists functions in Vector Space $ \Lp{X,\alg{A},\mu} $ for $ p\in (0,1) $ such that $ \norm{\cdot}_p $ does not satisfies the $ \Delta $-Inequality, hence $ \norm{\cdot}_p $ is not a norm on the vector space $ \Lp{X,\alg{A},\mu} $ for $ p\in (0,1) $.\\\\
$ \bigstar $ \textbf{But what about $ p \ge 1 $?} It turns out we need more revelations, in terms of results, to prove that for $ p\ge 1 $, $ \norm{\cdot}_p $ is a norm on the vector space $ \Lp{X,\alg{A},\mu} $. We now discuss those \emph{revelations}.
\newpage
\begin{lemma*}
	Consider $ a\ge 0  $, $ b\ge 0 $ and $ 0<\lambda < 1 $, then 
	\[a^{\lambda}b^{1-\lambda} \le \lambda a + (1-\lambda) b.\]
\end{lemma*}
\begin{proof}
	Consider the convex function $e^{x}  $. Since it is convex, therefore,
	\begin{equation*}
		\begin{split}
			a^{\lambda} b^{1-\lambda} &= e^{\lambda  \ln a + (1-\lambda) \ln b}\\ &\le \lambda e^{\ln a} + (1-\lambda) e^{\ln b}\\
			&= \lambda a + (1-\lambda ) b
		\end{split}
	\end{equation*}
\end{proof}
\hrulefill
\subsubsection{H\"older's Inequality}
One of the important \& frequently used inequalities which would be a stepping stone to show that $ \norm{\cdot}_p $ is a norm on $ \Lp{X,\alg{A},\mu} $ for $ p\ge 1 $.
\begin{theorem}\label{T-8}
	(\textbf{H\"older's Inequality}) Consider $ 1<p,q<\infty $ such that 
	\[\frac{1}{p} + \frac{1}{q} = 1.\]
	Then, for any $ f \in \LS{p}{X,\alg{A},\mu}$ and $ g\in \LS{q}{X,\alg{A},\mu} $
	\begin{enumerate}
		\item { \[\begin{aligned}[t]
				fg \in \LS{1}{X,\alg{A},\mu}
			\end{aligned}\]}
		\item {\[\begin{aligned}[t]
				\int \abs{fg}\le \wnorm{p}{f} \cdot \wnorm{q}{g}
			\end{aligned}\].\\
		OR\\
\[\begin{aligned}[t]
	\norm{fg}_1\le \norm{f}_p\cdot \norm{g}_q 
\end{aligned}\]
}
	\end{enumerate}
\end{theorem}
\begin{proof}
	From the above Lemma, we have that for any $ a > 0 $ and $ b> 0	 $, the following holds:
	\begin{align*}
		a^{\frac{1}{p}} b^{\frac{1}{q}} \le \frac{a}{p} + \frac{b}{q}
	\end{align*}
Now, if we set 
\begin{align*}
	a &= \frac{\abs{f}^{p}}{{\left (\norm{f}_p\right )}^{p}}\\
	b &= \frac{\abs{g}^{q}}{{\left (\norm{g}_q\right )}^{q}}
\end{align*}
and then use the inequality in above lemma, we get:
\begin{align*}
	\frac{\abs{f} \abs{g}}{\norm{f}_{p} \norm{g}_{q}} \le \frac{1}{p}\cdot \frac{\abs{f}^{p}}{{\left (\norm{f}_p\right )}^{p}} + \frac{1}{q} \cdot \frac{\abs{g}^{q}}{{\left (\norm{g}_q\right )}^{q}}.
\end{align*}
Now, because $ \abs{f}\abs{g} = \abs{fg} $, therefore from above inequality, we see that
\[\int \abs{fg} < \infty\]
hence $ fg \in \LS{1}{X,\alg{A},\mu} $. Furthermore, since we know that inequality is preserved in Integration, therefore integrating the above inequality leads to the following:
\begin{align*}
	\int \frac{\abs{f} \abs{g}}{\norm{f}_{p} \norm{g}_{q}} &\le \int \frac{1}{p}\cdot \frac{\abs{f}^{p}}{{\left (\norm{f}_p\right )}^{p}} + \int \frac{1}{q} \cdot \frac{\abs{g}^{q}}{{\left (\norm{g}_q\right )}^{q}}\\
	\frac{1}{\norm{f}_p\norm{g}_q}\int \abs{fg} &\le \frac{1}{p \left (\norm{f}_p\right )^{p}}\int  \abs{f}^{p} + \frac{1}{q \left (\norm{g}_q\right )^{q}}\int  \abs{g}^{q}\\
	\frac{\norm{fg}^{1}_1}{\norm{f}_p\norm{g}_q} &\le \frac{1}{p} + \frac{1}{q} =1\\
	\norm{fg}_1&\le \norm{f}_p \norm{g}_q
\end{align*}
Hence proved.
\end{proof}
\begin{remark}
	With H\"older's Inequality, we are one step closer to proving that $ \norm{f+g}_p \le\norm{f}_p + \norm{g}_p  $ for any $ f,g\in \LS{p}{X, \alg{A}, \mu} $ where $ 1\le p < \infty $, to formally make $ \norm{\cdot}_p $ a norm on the vector space $ \Lp{X,\alg{A},\mu} $. \textbf{This is finally proved by Minkowski's Inequality} which we prove now: 
\end{remark}
\hrulefill
\subsubsection{Minkowski's Inequality}
\begin{theorem}\label{T-9}
	(\textbf{Minkowski's Inequality}) : Consider any $ f,g \in \Lp{X,\alg{A},\mu} $ and $ 1\le p < \infty $. Then
	\[\wnorm{p}{f +g} \le \wnorm{p}{f} + \wnorm{p}{g}\]
	OR, 
	\[\norm{f +g}_p \le \norm{f}_p + \norm{g}_p.\]
\end{theorem}
\begin{proof}
	Since $ \abs{f+g} \le \abs{f} + \abs{g} $, therefore if $ p = 1 $, then the result follows immediately. \\
	Now consider $ p > 1 $. Moreover, suppose that $ q> 1 $ is such that
	\[\frac{1}{p} + \frac{1}{q} = 1.\]
	Note that this also leads to following equations
	\begin{align*}
		(p-1)q &= p\\
		p\left (1-\frac{1}{q}\right ) &=1
	\end{align*}
	Now, with this, we can bound $ \norm{f+ g}_p^{p} $ as follows:
	\begin{align*}
		\norm{f+g}_{p}^{p} &= \int \abs{f+g}^{p}\\
		&= \int \abs{f+g} \cdot \abs{f+g}^{p-1}\\
		&\le \int \left (\abs{f} + \abs{g}\right )\cdot \abs{f+g}^{p-1}\\
		&= \int \abs{f}\cdot \abs{f+g}^{p-1} + \int \abs{g}\cdot \abs{f+g}^{p-1}\\
		&= \int \abs{\abs{f} \cdot\abs{f+g}^{p-1}} + \int \abs{\abs{g}\cdot \abs{f+g}^{p-1} }\\
		&= \int \abs{f\cdot (f+g)^{p-1}} + \int \abs{g \cdot (f+g)^{p-1}}\\
		&= \norm{f\cdot (f+g)^{p-1}}_1 + \norm{g\cdot (f+g)^{p-1}}_1\\
		&\le \norm{f}_p \cdot\norm{(f+g)^{p-1}}_q + \norm{g}_p\cdot \norm{(f+g)^{p-1}}_q&&\text{H\"older's Inequality (\ref{T-8})}\\
		&= \left (\norm{f}_p + \norm{g}_p\right )\cdot \norm{(f+g)^{p-1}}_q\\
		&= \left (\norm{f}_p + \norm{g}_p\right ) \cdot \wnorm{q}{(f+g)^{p-1}}\\
		&= \left (\norm{f}_p + \norm{g}_p\right ) \cdot \left (\int (f+g)^{(p-1)q}\right )^{\frac{1}{q}}\\
		&= \left (\norm{f}_p + \norm{g}_p\right ) \cdot \left (\int (f+g)^{p}\right )^{\frac{1}{q}}\\
		&= \left (\norm{f}_p + \norm{g}_p\right )\cdot \left (\int (f+g)^{p}\right )^{\frac{1}{p}\cdot \frac{p}{q}}\\
	 &= \left (\norm{f}_p + \norm{g}_p\right )\cdot \norm{f+g}_p^{\frac{p}{q}}\\
		\frac{\norm{f+g}_p^{p}}{\norm{f+g}_p^{\frac{p}{q}}} &\le  \left (\norm{f}_p + \norm{g}_p\right )\\
		\norm{f+g}_p^{p\left (1-\frac{1}{q}\right )} &\le  \left (\norm{f}_p + \norm{g}_p\right )\\
		\norm{f+g}_p &\le \norm{f}_p + \norm{g}_p
	\end{align*}
Hence proved.
\end{proof}
\begin{remark}
 $ \bigstar $ Hence, in continuation of our effort to prove that \textbf{$ \norm{\cdot}_p $ is a norm on the vector space $ \Lp{X,\alg{A},\mu} $ for $ 1\le p<\infty $}, we can now satisfactorily say that it is indeed such, especially by Minkowski's Inequality just proved. One also calls a vector space with norm a norm space.
\end{remark}

\hrulefill
\newpage
\subsection{Completeness of Norm Space $ \Lp{X,\alg{A},\mu} $}
We now see that the norm space $ \Lp{X,\alg{A},\mu} $ is actually a complete metric space on the metric induced by the norm! But before stating the result, let us revisit the definitions of \emph{series}, \emph{Cauchy Sequences} \& \emph{completeness} for any arbitrary norm space $ (V,\R,\norm{\cdot}) $.    
\subsubsection{General Definitions and Results in Normed Spaces}
\begin{definition}
	(\textbf{Convergent Sequence}) Let $ (V,\R,\norm{\cdot}) $ be a norm space and $ \{x_n\} $ be a sequence in it. Then $ \{x_n\} $ is said to converge to $ x\in (V,\R,\norm{\cdot}) $ if 
	\[\norm{x_n - x} \longrightarrow 0\;\text{ as }n\to \infty.\]
\end{definition}
\hrulefill

\begin{definition}
	(\textbf{Cauchy Sequence}) Let $ (V,\R,\norm{\cdot}) $ be a norm space and $ \{x_n\} $ be a sequence in it. Then $ \{x_n\} $ is said to be a Cauchy Sequence in $ (V,\R,\norm{\cdot}) $ if
	\[\forall \epsilon > 0, \;\exists \; N\in \N \;\text{such that}\; \norm{x_n - x_m} < \epsilon\;\forall \;n,m \ge N.\]
\end{definition}
\hrulefill
\begin{definition}
	(\textbf{Complete Norm Space} or \textbf{Banach Space}) A norm space $ (V,\R,\norm{\cdot}) $ is called a Complete Metric Space or a Banach Space if 
	\[\text{Every Cauchy Sequence in $ (V,\R,\norm{\cdot}) $ is convergent in $ (V,\R,\norm{\cdot}) $.}\]
\end{definition}
\hrulefill
\begin{definition}
	(\textbf{Series in a Norm Space}) A series in a norm space $ (V,\R,\norm{\cdot}) $ is defined as 
	\[\sum_{n=1}^{\infty} x_n\;\text{where }x_n \in (V,\R,\norm{\cdot}).\]
\end{definition}
\hrulefill
\begin{definition}
	(\textbf{Convergent Series in a Norm Space}) A series $ \sum_{n=1}^{\infty} x_n$ in a norm space $ (V,\R,\norm{\cdot}) $ is said to be convergent if the sequence 
	\[\{S_n\} \text{ where }S_n = \sum_{i=1}^{n} x_i\text{ is convergent in $ (V,\R,\norm{\cdot}) $.}\]
\end{definition}
\hrulefill
\begin{definition}
	(\textbf{Absolutely Convergent Series}) Consider a series $ \sum_{n=1}^{\infty}x_n $ in a norm space $ (V,\R,\norm{\cdot}) $. Then it is called absolutely convergent if and only if
	\[\sum_{n=1}^{\infty} \norm{x_n} < \infty.\]
\end{definition}
\hrulefill

We now see the equivalent condition needed for a norm space to become a complete norm space:
\begin{theorem}\label{T-10}
	(\textbf{\emph{Equivalent condition for a Banach Space}}) Suppose that $ (V,\R,\norm{\cdot}) $ is a norm space. Then,
	\[\text{ $ (V,\R,\norm{\cdot}) $ is a Complete Norm Space (or Banach Space) $ \bm{\iff} $ Every Absolutely Convergent Series is also Convergent in $ (V,\R,\norm{\cdot}) $.}\]
\end{theorem}
\begin{proof}
	\textbf{L $ \implies  $ R} : Suppose $ (V,\R,\norm{\cdot}) $ is a Banach Space. Hence any Cauchy sequence in it converges at a point within it. Now, take any Absolutely Convergent series, say,
	\[\sum_{n=1}^{\infty} x_n\]
	in $ (V,\R, \norm{\cdot}) $. This means that
	\[\sum_{n=1}^{\infty}\norm{x_n} < \infty.\]
	Now this also means that if we write $ S_n = \sum_{i=1}^{n} x_i $, then 
	\begin{align*}
		\norm{S_n - S_m} &= \norm{\sum_{i=1}^{n} x_i - \sum_{i=1}^{m} x_i}\\
		&= \norm{\sum_{i=n}^{m} x_i }\\
		&\le \sum_{i=n}^{m} \norm{x_i}
	\end{align*}
Now since $ \sum_{n=1}^{\infty} x_n$ is absolutely convergent, therefore, 
\[\sum_{i=n}^{m} \norm{x_i} \le \sum_{i=1}^{\infty} \norm{x_i} < \infty\;\forall\;n\le m\in\N \]
also with the fact that $ \sum_{i=n}^{m} \norm{x_i} \longrightarrow 0$ as $ n,m \to \infty $.
Hence,
\[\norm{S_n-S_m} < \infty \;\text{and}\;\norm{S_n-S_m}\to 0 \;\text{as }n,m\to \infty.\]
Hence, $ \{S_n\} $ is a Cauchy Sequence in $ (V,\R,\norm{\cdot}) $ and thus is convergent. Therefore, $ \sum_{i=1}^{\infty}x_i $ is also convergent.\\\\
\textbf{R $ \implies  $ L} : Suppose that $ (V,\R,\norm{\cdot}) $ is a norm space with given that every absolutely convergent series converges. Since we need to show that $ (V,\R,\norm{\cdot}) $ is then a Banach space, hence we now consider any arbitrary Cauchy sequence, say, $ \{x_n\} $.\\
Now, construct a new sequence from the taken Cauchy sequence $ \{x_n\} $ as $ \{y_n\} $ defined by the following:
\begin{align*}
	y_1 &= x_{N_1} \;\text{where } N_1 \text{ is such that } \norm{x_n - x_m} < \frac{1}{2^{1}}  \;\forall\; n,m\ge N_1\\
		y_2 &= x_{N_2} - x_{N_1} \;\text{where } N_2 \text{ is such that } \norm{x_n - x_m} < \frac{1}{2^{2}}  \;\forall\; n,m\ge N_2 > N_1\\
		\vdots &= \vdots\\
			y_k &= x_{N_k} - x_{N_{k-1}} \;\text{where } N_k \text{ is such that } \norm{x_n - x_m} < \frac{1}{2^{k}}  \;\forall\; n,m\ge N_k > N_{k-1}.
\end{align*}
Now, with $ \{y_n\} $ in hand, we see some peculiar properties of it, such as:
\[\sum_{j=1}^{k} y_j= x_{N_k}.\]
and especially, we see that $ \sum y_n $ is absolutely convergent(!) as follows:
\begin{align*}
	\sum_{j=1}^{\infty} \norm{y_j} &\le \norm{y_1} + \sum_{j=1}^{\infty} \norm{y_j}\\
	&\le \norm{x_{N_1}} + \sum_{j=1}^{\infty} \frac{1}{2^{j}}\\
	&= \norm{x_{N_1}} + 1 < \infty&&\text{$ \because \{x_n\}$ is Cauchy, so $ \norm{x_i} < \infty\forall i$. }
\end{align*} 
Now, since we are given that every absolutely convergent series in $ (V,\R,\norm{\cdot}) $ converges, therefore $ \sum y_n $ also converges in $ (V,\R,\norm{\cdot}) $. But convergence of a series means convergence of it's sequence of partial sums $ S_n = \sum_{i=1}^{n} y_i $ and $ S_n = x_{N_{n}} $ as shown above. Therefore, we have
\[\{x_{N_{n}}\} \text{ converges in $ (V,\R,\norm{\cdot}) $.} \] 
Since $ \{x_{N_n}\} $ converges, therefore, if we suppose $ x_{N_n}\to x $, then:
\begin{align*}
	\norm{x_n - x} &=\norm{x_n - x_{N_n} + x_{N_n} - x}\\
	&\le \norm{x_n - x_{N_n}} + \norm{x_{N_n} - x}&&\text{$ 2^{nd} $ term $ \to 0 $.}\\
	&= \norm{x_n -x_{N_n}} 
\end{align*}
Now, we know that $ n < N_n $, therefore, $ \exists p\in \N$ such that $ N_n > n \ge N_{n-p} $. Hence
\[\norm{x_n - x_{N_n}} < \frac{1}{2^{n-p}}\]
and as $ n\to \infty $, $ \norm{x_n - x_{N_n}} \to 0 $ too.
Therefore,
\[\norm{x_n - x} \to 0.\]
Hence, $ \{x_n\} $ is a convergent sequence, apart from being Cauchy. Since the choice of $ \{x_n\} $ was arbitrary, therefore all Cauchy sequences are convergent. Hence $ (V,\R,\norm{\cdot}) $ is a Complete Norm Space or Banach Space.
\end{proof}
\hrulefill
\newpage
\subsubsection{$ \Lp{X,\alg{A},\mu} $ is a Banach Space!}
We now see that $ \Lp{X,\alg{A},\mu} $ is a Complete Norm Space.
\begin{theorem}
	The normed vector space $ \Lp{X,\alg{A},\mu} $ for $ 1\le p < \infty $ is a Banach Space.
\end{theorem}
\begin{proof}
	From the Theorem \ref{T-10}, we just need to equivalently show that any absolutely convergent series is convergent.\\
	 Now consider $ \{f_k\} $ in $ \Lp{X,\alg{A},\mu} $ to be absolutely convergent, so that
	\[\sum_{k=1}^{\infty}\norm{f_k}_p = B < \infty.\]
	Also consider the following sequence:
	\[G_n = \sum_{k=1}^{n} \abs{f_k}\;\text{and}\; G = \sum_{k=1}^{\infty} \abs{f_k}.\]
	Clearly, for all $ n\in \N $ we have
	\begin{align*}
		\norm{G_n}_p &= \norm{\sum_{k=1}^{n} \abs{f_k}}\\
		&\le \sum_{k=1}^{n}\norm{f_k}_p \le B < \infty.
	\end{align*}
	Also note that $ \{G_n\} $ is an increasing sequence of positive-valued Measurable functions.\\
	Since $ \lim_{n} G_n $ exists, therefore, by the Monotone Convergence Theorem (Theorem \ref{T-5}), we have:
	\begin{align*}
		\int \lim_{n} G_n^{p} &= \int G^{p} \\
		&= \lim_{n}\int G_n^{p} \\
		&\le B^{p}
	\end{align*}
Therefore, we have that $ \int G^{p} $ is finite almost everywhere on $ \R $ (just consider the above result that $ \int \left (G^{p} - \chi_{X}\right ) = 0 $ where $ \m{X} = B^{p}$.)\\
Since $ G^{p} $ is finite almost everywhere, therefore $ G $ is finite almost everywhere. Hence, we get 
\begin{align*}
	\sum_{k=1}^{\infty}{f_k} &\le \sum_{k=1}^{\infty} \abs{f_k}\\
	&= G < \infty\;\;\text{almost everywhere.}
\end{align*}
Now write 
\[F = \sum_{k=1}^{\infty}f_k.\]
Clearly, we have
\begin{align*}
	\abs{F} &= \abs{\sum_{k=1}^{\infty} f_k}\\
	&\le \sum_{k=1}^{\infty} \abs{f_k}\\
	&= G < \infty
\end{align*}
and since $ f_k $ are members of the vector space $ \Lp{X,\alg{A},\mu} $, we also have that $ F \in \Lp{X,\alg{A},\mu} $.\\
Now, we see that
\begin{align*}
	\abs{F - \sum_{k=1}^{n}f_k}^{p} &\le \abs{F} + \abs{\sum_{k=1}^{n}f_k}\\
	&\le G + G = 2G\\
	&\le (2G)^{p} \;\;\text{$ \because $ $ 1\le p < \infty $.}\\
	&< \infty
\end{align*}
Now since $ \abs{F - \sum_{k=1}^{n} f_k}^{p} < \infty$, hence it is in $ \LS{1}{X,\alg{A},\mu} $.\\
With the above inequality, we see that $ \abs{F - \sum_{k=1}^{n}} $ is finite and is absolutely bounded by another Measurable function for each $ n $, hence, we can now use the Dominated Convergence Theorem (Theorem \ref{T-7}) to write 
\begin{align*}
	\left (\lim_{n} \int \abs{F - \sum_{k=1}^{n}f_k}^{p}\right )^{\frac{1}{p}} &= \left (\int \lim_{n} \abs{F - \sum_{k=1}^{n}f_k}^{p}\right )^{\frac{1}{p}}\\
\lim_{n}\norm{F- \sum_{k=1}^{n}f_k}_p	&= 0&&\text{Note that $ F = \sum_{k=1}^{\infty} f_k \in  \Lp{X,\alg{A},\mu}$.}
\end{align*}
Hence, we have
\[\sum_{k=1}^{\infty} f_k = F \in \Lp{X,\alg{A},\mu}\]
that is, the absolutely convergent series $ \sum_{k=1}^{f_k} $ is also convergent in the same space! Therefore, $ \Lp{X,\alg{A},\mu} $ is a Banach Space. 
 \end{proof}
\hrulefill
\newpage
\section{Product Measure}
We now turn to product measure spaces. This concept would help us to formalize the notion of double (or higher) integration over the so defined \emph{product measure spaces}. In fact, this concept actually shows the generality of the concept of measure spaces, which we might discuss afterwards.\\\\
To introduce formal notion of product measure space, we need a definition based framework to work in, which we learn now:
\begin{definition}
	(\textbf{Premeasure}) Consider an \textbf{algebra}\footnote{Note that this just an algebra, not a $ \sigma $-algebra.} $ \alg{A}$ over a set $ X $. The map 
	\[\mu_{0} : \alg{A} \longrightarrow [0,+\infty]\]
	is called a premeasure if it satisfies:
	\begin{enumerate}
		\item {$ \prem{\Phi} = 0$, and}
		\item {For $ A_1,A_2,\dots $ a sequence of disjoint sets from $ \alg{A} $,
	\[\prem{\bunion_{i=1}^{\infty} A_i} = \sum_{i=1}^{\infty} \prem{A_i}\]	
	}
	\end{enumerate}
\end{definition}
\hrulefill
\begin{definition}\label{D-37}
	(\textbf{Outer Measure by Premeasure}) Consider an algebra $ \alg{A} $ defined on set $ X $. Suppose $ \mu_{0} : \alg{A} \to [0,+\infty] $ is a premeasure on it. We then define $ \mu_{*} $ as the following:
	\[\mu_{*} : \alg{A} \longrightarrow [0,+\infty]\]
	defined by, for $ A \subseteq X $:
	\[\mu_{*}(A) = \inf \left \{\sum_{n=1}^{\infty} \prem{E_n} \where A \subseteq \bunion_{n=1}^{\infty} E_n\text{ where } \{E_n\}\text{ is a sequence in $ \alg{A} $}\right \}\]
\end{definition}
\hrulefill
\begin{proposition}\label{P-39}
	For an algebra $ \alg{A} $ on $ X $, $ \mu_{*} $ satisfies the following:
	\begin{enumerate}
		\item {$ \mu_{*} $ is an Outer Measure.}
		\item {The collection of $ \mu_{*} $ measurable sets, $ \msigm{\mu_{*}} $, is a $ \sigma $-algebra.}
		\item {The $ \sigma $-algebra generated by algebra $ \alg{A} $, $ \alg{B} $, is a proper subset of $ \msigm{\mu_{*}} $. That is,
	\[\alg{B} \subsetneq \msigm{\mu_{*}}\]	
}
	\end{enumerate}
\end{proposition}
\begin{proof}
	Clearly, $ \opm{\Phi} = 0$ as, the $ \infty $ sequence of $ \Phi $, $ \{A_n\} $ where $ A_i = \Phi \;\forall\;i$, is such that
	\[\Phi \subseteq \bunion_{i}A_i\]
	and 
	\[\sum_{i} \prem{A_i} = 0.\]
	The next parts has proof similar to one done for Lebesgue Outer Measure.
\end{proof}
\hrulefill
\newpage
\subsection{Some Set Theoretic Concepts}
We would need this concepts for later discussions.
\begin{definition}
	(\textbf{Elementary Class/Family}) A collection of sets denoted by $ \alg{E} $ is called an elementary class if:
	\begin{enumerate}
		\item {$ \Phi \in \alg{E} $,}
		\item {For any $ E,F \in \alg{E} $, then 
	\[E\intrs F \in \alg{E}\]	
	}
\item {If $ E\in \alg{E} $, then
\[\exists\; \{F_n\}_{n=1}^{N} \;\text{where $ F_n $'s are disjoint and in $ \alg{E} $}\;\text{such that } \comp{E} = \bunion_{n=1}^{N} F_n\]
}
	\end{enumerate}
\end{definition}
\hrulefill
\begin{proposition}\label{P-40}
	If $ \alg{E} $ is an Elementary Class, then the collection $ \alg{A} $ defined as:
	\[\text{For any $ A \in \alg{A} $, }\exists \;\{E_n\}_{n=1}^{N} \;\text{where $ E_n $'s are disjoint and in $\alg{E}  $ such that } A = \bunion_{n=1}^{N} E_n\]
	is an Algebra.
\end{proposition}
\begin{proof}
	We already have $ \Phi \in \alg{A} $ as $ \Phi $ is in $ \alg{E} $. Moreover, we have for any $ A\in \alg{A} $ a sequence of disjoint sets, $ \{E_n\}_{n=1}^{N} $, from $ \alg{E} $, such that,
	\[A = \bunion_{n=1}^{N} E_n.\]
	Now, note that for $ \comp{A} $, we have:
	\begin{align*}
		\comp{A} &= \bintrs_{n=1}^{N} \comp{E_n}\\
		&= \bintrs_{n=1}^{N} \bunion_{i=1}^{M_n} F_i^{n} 
	\end{align*} 
where last line follows because $ E_n \in \alg{E} $ where $ \alg{E} $ is an elementary class, so for each $ \comp{E_n} $, there exists a finite sequence of disjoint sets $ \{F^{n}_i\}_{i=1}^{M_n} $ in $ \alg{E} $  whose union is $ \comp{E_n} $. \textbf{\emph{Complete the proof later, in diary.}}
\end{proof}
\hrulefill
\begin{definition}
	(\textbf{Monotone Class}) A collection of subsets of a set $ X $ denoted $ \alg{C} \subseteq \pow{X}$ is called a monotone class if:
	\begin{enumerate}
		\item {For if $ \{E_n \} $ is a sequence of monotonically increasing sets from $ \alg{C} $, that is,
	\[E_1\subseteq E_2 \subseteq \dots \]	
	then,
\[\bunion_{n=1}^{\infty} \in \alg{C}.\]
 }
\item {For if $ \{E_n \} $ is a sequence of monotonically decreasing sets from $ \alg{C} $, that is,
	\[E_1\supseteq E_2 \supseteq \dots \]	
	then,
	\[\bintrs_{n=1}^{\infty} \in \alg{C}.\]}
	\end{enumerate}
\end{definition}
\hrulefill
\newpage
\begin{proposition}
	Consider a family of monotone class given as $ \{\alg{C}_n\} $. Then,
	\[\bintrs_{n} \alg{C}_n \text{ is a monotone class.}\]
\end{proposition}
\begin{proof}
	Take any sequence of sets $ \{I_n\} $ from $ \bintrs_{n=1}^{\infty}  \alg{C}_n $ such that they are monotonically increasing,
	\[I_1\subseteq I_2 \subseteq \dots.\]
	Now, because each $ I_i \in \alg{C}_n \forall \; n$ and it is a monotonically increasing sequence, therefore $ \bunion_{i=1}^{\infty} I_i \in \alg{C}_n\;\forall\;n$. Hence,
	\[\bunion_{i=1}^{\infty} I_i \in \bintrs_{n}\alg{C}_n.\]
	Similarly, suppose $ \{J_n\} $ is a monotonically decreasing sequence of sets from $ \bintrs_{n}\alg{C}_n $,
	\[J_1\supseteq J_2 \supseteq \dots\]
	Hence, each $ J_i \in \alg{C}_n \;\forall\;n$. Since each $ \alg{C}_n $ is a monotone class, therefore, $ \bintrs_{i=1}^{\infty} J_i \in \alg{C}_n \;\forall\;n$. Hence,
	\[\bintrs_{i=1}^{\infty} J_i \in \bintrs_{n}\alg{C}_n.\]
	Hence proved.
\end{proof}
\hrulefill
\begin{definition}
	(\textbf{Generated Monotone Class}) Consider any $ \alg{S} \subset \pow{X} $. Then, $ \alg{C}(\alg{S}) $ is called the monotone class generated by $ \alg{S} $ if $ \alg{C}(\alg{S}) $ is the smallest monotone class containing $ \alg{S} $.
\end{definition}
\hrulefill
\begin{proposition}
	Let $ \alg{A} $ be an Algebra. Suppose 
	\begin{itemize}
		\item {$ \alg{C}(\alg{A}) $ is the Monotone Class generated by $ \alg{A} $, and}
		\item {$ \alg{M} $ is the $ \sigma $-Algebra generated by $ \alg{A} $.} 
	\end{itemize}
Then,
\[\alg{M} =  \alg{C}(\alg{A}).\]
\end{proposition}
\begin{proof}
	\emph{Typeset the proof later. Written in Diary at $ 4^{th} $ August, 2018.}
\end{proof}
\hrulefill
\newpage
\subsection{Product Measure Space}
\begin{definition}
	(\textbf{Measurable Rectangle}) Suppose $ (X,\alg{A},\mu) $ and $ (Y,\alg{B},\nu) $ are two measure spaces. Suppose $ X\times Y $ is the Cartesian Product of the sets $ X $ and $ Y $. Then, $ A\times B \subseteq X\times Y$ is called a Measurable Rectangle if
	\[A\in \alg{A}\text{ and } B\in \alg{B}\]
\end{definition}
\hrulefill
\begin{definition}
	(\textbf{Elementary Rectangles}) Suppose $ (X,\alg{A},\mu) $ and $ (Y,\alg{B},\nu) $ are measure spaces. Denote by $ \alg{K} $ the collection of all Measurable Rectangles. Then, we define Elementary Rectangles, $ \alg{E} $, as the collection :
	\[\text{For any $ A \in \alg{E} $, }\exists \;\{E_n\}_{n=1}^{N} \;\text{where $ E_n $'s are disjoint Measurable Rectangles in $\alg{K}  $ such that } A = \bunion_{n=1}^{N} E_n.\]
\end{definition}
\begin{remark}
$ \bigstar $ It is important to note that\textbf{ Elementary Rectangles $ \alg{E} $ is an Algebra}, due to Proposition \ref{P-40}.
\end{remark}
\hrulefill
\begin{definition}
	(\textbf{Product of Measurable Spaces}) Denote $ \alg{A} \times \alg{B}$ to be the $ \sigma $\textbf{-Algebra generated by} $ \alg{E} $. Then, 
	\[(X\times Y, \alg{A}\times\alg{B})\]
	is the product of Measurable Spaces $ (X,\alg{A}) $ and $ (Y,\alg{B}) $.
\end{definition}
\hrulefill
\begin{definition}
	(\textbf{Product Measure Space}) Suppose $ (X,\alg{A},\mu) $ and $ (Y,\alg{B},\nu) $ are two Measure Spaces. The product of these two Measure Spaces is defined as the following triple:
	\[\left (X\times Y, \alg{A}\times \alg{B},\mu \times \nu\right )\]
	where
	\begin{enumerate}
		\item {$ X\times Y $ is the Cartesian Product of $ X $ and $ Y $.}
		\item {$ \alg{A}\times\alg{B} $ is the $ \sigma $-Algebra generated by Elementary Rectangles of the product $ X\times Y $ under respective measure spaces.}
		\item {$ \mu \times \nu $ is defined as:
	\[\left (\mu \times \nu\right )(A\times B) = \mu(A) \cdot \nu(B)\]	
	where $ A\times B $ is a Measurable Rectangle.
	}
	\end{enumerate}
\end{definition}
\begin{remark}
	$ \bigstar $ Note the following:
	\begin{itemize}
		\item {$ \mu \times \nu $ defines a premeasure on Elementary Rectangles, $ \alg{E} $, which is an Algebra. }
		\item {With the premeasure $ \mu \times \nu $ on $ \alg{E} $, we then construct the outer measure $ \mu_{*}\times \nu_{*} $ by premeasure as done in Definition \ref{D-37}.}
		\item {As Proposition \ref{P-39} shows, the collection of $ \mu_{*} \times \nu_{*}$ measurable sets from $ \alg{E} $ forms a $ \sigma $-Algebra, that is, the $ \sigma $-Algebra generated from all Elementary Rectangles. This is exactly what we did now.}
	\end{itemize}
\end{remark}
\hrulefill
\newpage
\subsubsection{Properties of Product Measure Space}
\begin{definition}
	(\textbf{$ x $ \& $  y $ Sections}) Suppose $ E\subseteq X\times Y $. Then we define
	\begin{enumerate}
		\item {$ x $-Section as all $ y $ available in $ E $ if $ x $ is fixed:
	\[E_x = \{y\in Y \where (x,y) \in E\}\]	
	}
		\item {$ y $-Section as all $ x $ available in $ E $ if $ y $ is fixed:
	\[E^y = \{x\in X \where (x,y) \in E\}\]	
}
	\end{enumerate}
\end{definition}
\hrulefill
\begin{definition}
	(\textbf{$ x $ \& $ y $ Sections of a Function}) Suppose $ f $ is a function on $ X\times Y $. Then,
	\begin{enumerate}
		\item {$ x $-Section of $ f $ given $ x\in X $ is just $ f_x(y) = f(x,y) $.	
	}
\item {$ y $-Section of $ f $ given $ y \in Y$ is just $ f^{y}(x) = f(x,y) $}
	\end{enumerate}
\end{definition}
\hrulefill
\begin{proposition}\label{P-43}

		{Suppose $ (X,\alg{A}) $ and $ (Y,\alg{B}) $ are two measurable spaces and $ E\subseteq \alg{A}\times \alg{B} $. Then,
			\begin{enumerate}
				\item {$ E_x \in \alg{B} $ $ \forall x\in X $ , and}
				\item {$ E^{y} \in \alg{A} $ $ \forall y\in Y $.}
			\end{enumerate}
			That is, each section of a subset of product of measurable spaces, $ \alg{A} \times \alg{B}$, is itself measurable.}

\end{proposition}
\begin{proof}
	 \emph{Trivial, but done in Diary at $ 13^{th} $ August, 2018.}
\end{proof}
\hrulefill
\begin{proposition}
	Suppose $ f : \alg{A}\times \alg{B} \longrightarrow \R$ is a $ \alg{A} \times \alg{B}$-Measurable Function. Then,
	\begin{enumerate}
		\item {$ f_x : \alg{B}\longrightarrow \R $ is a $ \alg{B} $-Measurable Function $ \forall x\in X $.}
		\item {$ f^y : \alg{A}\longrightarrow \R $ is a $ \alg{A} $-Measurable Function $ \forall y\in Y $.}
	\end{enumerate}
That is, each section of a Measurable Function on product Measurable space is itself a Measurable function.

\end{proposition}

\begin{proof}
	\emph{Trivial, same as Proposition \ref{P-43}.}
\end{proof}
\hrulefill
\newpage
\subsection{The Fubini-Tonelli Theorem}
This is perhaps the most important result of this course, whose proof can be found in any course book, available on webpage.
\begin{theorem}
	 (\textbf{Tonelli's Theorem}) Suppose $ (X,\alg{A},\mu) $ and $ (Y,\alg{B},\nu) $ are two $ \sigma $-finite\footnote{This means that there are finite $ \{A_n\} $ sets in $ \alg{A} $ with finite measure such that $ \bunion_{n} A_n = X $. Similarly for $ (Y,\alg{B},\nu) $.} measure spaces. Consider an $ \alg{A}\times \alg{B} $-Measurable function 
	 \[f : X\times Y \longrightarrow [0,+\infty].\]
	 Then,
	 \begin{enumerate}
	 	\item {The function:
	\begin{itemize}
		\item {$ g : X\to [0,+\infty] $ given by:
	\[g(x) = \int_{Y} f_xd\nu\]	
	is $ \alg{A} $-Measurable.
	}
\item {$ h : Y\to [0,+\infty] $ given by:
	\[h(y) = \int_{X} f_yd\mu\]	
	is $ \alg{B} $-Measurable.}
	\end{itemize} 	
 	}
 \item {$ f $ satisfies:
\begin{align*}
	\int_{X\times Y} f  d(\mu\times\nu) &= \int_{X}\left (\int_{Y} f_x d\nu\right ) d\mu\\
	&= \int_{Y} \left (\int_{X} f_y d\mu \right ) d\nu
\end{align*} 
}
	 \end{enumerate}
\end{theorem}
\hrulefill
\begin{theorem}
	(\textbf{Fubini's Theorem}) Suppose $ (X,\alg{A},\mu) $ and $ (Y,\alg{B},\nu) $ be $ \sigma $-finite measurable spaces. Consider an $ \alg{A}\times \alg{B} $-Measurable function which is also $ \mu\times\nu $-Integrable given as:
	\[f : X\times Y \longrightarrow [-\infty,+\infty]\footnote{\textbf{Note the target set here!}}.\]
	Then,
	\begin{enumerate}
		\item {We have that
	\begin{itemize}
		\item {$ f_x $ is $ \nu $-Integrable \textbf{almost everywhere} on $ Y $.}
		\item {$ f_y $ is $ \mu $-Integrable \textbf{almost everywhere} on $ X $.}
	\end{itemize}	
	}
\item {The following relation holds:
\begin{align*}
	\int_{X\times Y} fd(\mu\times \nu) &= \int_{X}\left (\int_{Y} f_x d\nu\right ) d\mu\\
	&= \int_{Y} \left (\int_{X} f_y d\mu \right ) d\nu
\end{align*}
}
	\end{enumerate}
\end{theorem}
\hrulefill

%\section{Integration of Non-Negative Functions}
%We can now write any measurable function as 
%\[f = \sum_{}\]
%\newpage	
\newpage
\section{Differentiation}
We now study some of the interconnections between integration and differentiation and related notions.
\subsection{Differentiability}
\begin{definition}
	(\textbf{Upper/Lower Left \& Upper/Lower Right Derivatives}) Suppose $ f : \R \longrightarrow [-\infty,+\infty]$ is a function such that for all $ x \in \R $, $ f $ is defined on some open interval around $ x $, then we define the following quantities:
	\begin{itemize}
		\item {\textbf{\emph{Upper Right Derivative}} : 
	\[D^{+}f(x) = \limsup_{h\to 0^{+}} \frac{f(x+h) - f(x)}{h}\]	
	}
\item {\textbf{\emph{Lower Right Derivative}} : 
\[D_{+}f(x) = \liminf_{h\to 0^{+}} \frac{f(x+h) - f(x)}{h}\]
}
\item {\textbf{\emph{Upper Left Derivative}} : 
\[D^{-}f(x) = \limsup_{h\to 0^{-}}\frac{f(x+h) - f(x)}{h}\]
}
\item {\textbf{\emph{Lower Left Derivative}} : 
\[D_{-}f(x) = \liminf_{h\to 0^{-}} \frac{f(x+h) - f(x)}{h}\]
}
	\end{itemize}
\end{definition}
\hrulefill
\begin{definition}
	(\textbf{Differentiable function}) The function $ f $ is said to be differentiable at $ x $ if and only if:
	\[D^{+}f(x) = D_{+}f(x) = D^{-}f(x) = D_{-}f(x).\]
	Hence, a function is said to be differentiable if it is differentiable at all points of it's domain. 
\end{definition}
\hrulefill
\subsection{Functions of Bounded Variation}
We now study those functions which do not change too erratically over an interval. We already have the notion of differentiability for the same, so we would see connections between such type of functions and there differential character.
\begin{definition}
	(\textbf{Variations of a Function}) Suppose we are given a function on an interval
	\[f : [a,b] \longrightarrow \R\]
	and any partition $ \mathcal{P}_{[a,b]} = \{a =x_0,x_1,x_2,\dots,x_{k-1},x_k = b\}$ where $ x_i < x_{i+1} $. Now, define the following the following three quantities:
	\begin{align*}
		p_\mathcal{P} &= \sum_{i=1}^{k} \left (f(x_i) - f(x_{i-1})\right )^{+} \\
		n_\mathcal{P} &= \sum_{i=1}^{k}\left (f(x_i) - f(x_{i-1})\right )^{-}\\
		t_\mathcal{P} &= p_{\mathcal{{P}}}+n_{\mathcal{P}} = \sum_{i=1}^{k} \abs{f(x_i) - f(x_{i-1})} 
	\end{align*} 
where $ _{\mathcal{P}} $ denotes the partition over which the sum is defined and it's simple to observe that $ p_\mathcal{P} - n_{\mathcal{P}} = f(b) - f(a)$. Also, $ x^{+} = \max(x,0) $ and $ x^{-} = \max(-x,0) $.\\
 Then, we finally define the following three quantities:
\begin{itemize}
	\item {\textbf{\emph{Positive Variation of $ f $}} : 
	\[P_f = \sup_{\mathcal{P}} p_{\mathcal{P}}.\]	
}
\item {\textbf{\emph{Negative Variation of $ f $}} :
\[N_f = \sup_{\mathcal{P}} n_{\mathcal{P}}.\]
}
\item {\textbf{\emph{Total Variation of $ f $}} : 
\[T_f = \sup_{\mathcal{P}} t_\mathcal{P}.\]
}
\end{itemize}
\end{definition}
\hrulefill
\begin{definition}
	(\textbf{Function of Bounded Variation}) Suppose $ f : \R \to \R $ is a given function. Then $ f $ is said to be of bounded variation over interval $ [a,b] $ if 
	\[T_f[a,b] = T_f < \infty.\]
	The class of functions on a given interval $ [a,b] $ which are of bounded variation is denoted by:
	\[\bv{[a,b]}.\]
	So that for any $ f\in \bv{[a,b]} $, $ T_f < \infty $.
\end{definition}
\begin{remark}
	A function $ f $ is said to belong to $ \bv{(-\infty,\infty)} $ if $ f $ belongs to each $ \bv{[a,b]} $ for each interval $ [a,b] $. Clearly, in this case $ T_f(-\infty,\infty)  = \sup_{[a,b]} T_f[a,b]$.
\end{remark}
\hrulefill
\begin{proposition}
	Suppose $ f\in \bv{[a,b]} $. Then,
	\begin{enumerate}
		\item {$ f(b) - f(a) = P_f-N_f$.}
		\item {$ T_f = P_f + N_f $.}
	\end{enumerate}
\end{proposition}
\begin{proof}
	Take any $ f\in \bv{[a,b]} $. Then we have $ T_f[a,b] < \infty$. Now, we know that for any partition $ \mathcal{P} $ of $ [a,b] $, $ f(b) - f(a) = p_\mathcal{P} - n_\mathcal{P} $. Now, take supremum over all partitions of $ [a,b] $, both sides of the above, to write:
	\begin{align*}
		\sup_{\mathcal{P}} \left (f(b) - f(a)\right ) &= \sup_{\mathcal{P}} \left (p_\mathcal{P} - n_\mathcal{P}\right )\\
		f(b) - f(a)&= \sup_{\mathcal{P}} p_\mathcal{P} - \sup_{\mathcal{P}} n_\mathcal{P} &&\text{Known result : $ \sup_n \left (x_n - y_n \right ) = \sup_n x_n - \sup_n y_n $.}\\
		&= P_f - N_f
   	\end{align*}
   For the $ 2^{nd} $ part, we have
   \begin{align*}
   	T_f &= \sup_{\mathcal{P}} t_\mathcal{P}\\
   	&= \sup_{\mathcal{P}} (p_\mathcal{P} + n_{\mathcal{P}})\\
   	&= \sup_{\mathcal{P}} p_{\mathcal{P}} + n_{\mathcal{P}} &&\text{Known result : $ \sup_n \left (x_n + y_n \right ) = \sup_n x_n + \sup_n y_n $.}\\
   	&= P_f + N_f
    \end{align*}
Hence proved.
\end{proof}
\hrulefill

The following theorem is important as it characterizes the functions in $ \bv{[a,b]} $.
\begin{proposition} \label{P-46}
	The following result holds:
	\begin{align*}
		f\in \bv{[a,b]} \;\;\bm{\iff}\;\; \exists \;g,\;h\text{ which are \textbf{monotonically increasing} and \textbf{finite} on $ [a,b] $, such that } f = g-h.   
	\end{align*}
\end{proposition}
\begin{proof}
	\textbf{ L $ \implies $ R} : Consider the functions 
	\begin{align*}
		g(x) &= P_f[a,x] + f(a)\\
		h(x) &= N_f[a,x]
 	\end{align*}
 For any $ a\le x_0  \le x_1 \le b $, we observe that:
 \begin{align*}
 	g(x_0) &= P_f[a,x_0] + f(a) \le P_f[a,x_1] + f(a) = g(x_1)\\
 	h(x_0) &= N_f[a,x_0] \le N_f[a,x_1] = h(x_1)
 \end{align*}
because we are adding another partitioning point. Hence, $ g,h $ are monotonically increasing functions on $ [a,b] $. Hence, $ g(b) = P_f[a,b] + f(a) < \infty$ as $ f $ is of bounded variation, so $ g $ is finite. Similarly $ h $ is finite on $ [a,b] $. Finally, we note that:
\begin{align*}
	g(x) - h(x) &= P_f[a,x] + f(a) - N_f[a,x]\\
	&= P_f[a,x] - N_f[a,x] + f(a)\\
	&= f(x) - f(a) + f(a)\\
	&= f(x) 
\end{align*}
Hence proved that if $ f \in \bv{[a,b]}$, then there exists two monotonically increasing, finite functions on $ [a,b] $ such that $ f $ is their difference.\\\\
\textbf{R $ \implies $ L} : Take any partition $ \mathcal{P}[a,b] = a = x_0 < x_1 < x_2 < \dots <  x_k = b $. Now we see that
\begin{align*}
	t^{f}_\mathcal{P} &= \sum_{i=1}^{k} \abs{f(x_i) - f(x_{i-1})} \\
	&= \sum_{i=1}^{k} \abs{g(x_i) - h(x_i) - g(x_{i-1}) + h(x_{i-1})} \\
	&\le \sum_{i=1}^{k} \abs{g(x_i) - g(x_{i-1})} + \sum_{i=1}^{k}\abs{h(x_{i}) - h(x_{i-1})}\\
	&= t^{g}_{\mathcal{P}} + t^{h}_{\mathcal{P}}\\
	&< \infty
\end{align*}
Hence $ T_f = \sup_{\mathcal{P}} t^{f}_{\mathcal{P}} < \infty$. So $ f\in \bv{[a,b]} $.
\end{proof}
\hrulefill
\subsection{Differentiability of Monotone Functions \& Lebesgue's Theorem}
\begin{definition}
	(\textbf{Vitali Covering}) A collection $ \mathcal{C} $ of closed, bounded, nondegenerate\footnote{An interval $ [a,b] $ is said to be nondegenerate if $ a< b $.} intervals is said to cover a given set $ E $ in the sense of Vitali if:
	\[\text{For any $ x\in E $ and any $ \epsilon > 0 $, $ \exists \; I \in \mathcal{C}$ such that $x\in I \;\&\; \lm{I} < \epsilon $.}\]
\end{definition}
\hrulefill
\begin{theorem}\label{T-14}
	(\textbf{The Vitali Covering Lemma}) Suppose $ E \subset \R$ is of finite outer measure, that is $ \lom{E} < \infty $. Also consider a collection $ \mathcal{C} $ of closed, bounded intervals which covers $ E $ in the sense of Vitali. Then,
	\[\forall \;\epsilon > 0,\; \exists \text{ disjoint \& finite subcollection } \{I_k\}_{k=1}^{n} \text{ of } \mathcal{C} \text{ such that }\]
	\[\lom{E\setminus \bunion_{k=1}^{n} I_k } < \epsilon.\]
\end{theorem}
\begin{proof}
	\emph{In Progress \HalfFilledHut.}
\end{proof}
\newpage
The following is a generalization of Mean Value Theorem from Calculus.
\begin{proposition}
	Let $ f $ be an increasing function on a closed, bounded interval $ [a,b] $. Then, $ \forall \; \alpha > 0$, we have
	\[\lom{\{x\in(a,b)\where D^{+}f(x) = D^{-}f(x) \ge \alpha\}} \le\frac{1}{\alpha}\cdot \left ( f(b) - f(a) \right )\]
\end{proposition}
\begin{proof}
	We first begin be denoting $ E_\alpha = \{x\in (a,b)\where D^{+}f(x) = D^{-}f(x) \ge\alpha\} $. Now, construct the following collection $ \mathcal{F} $ of closed and bounded intervals $ [c,d] $ for which,
	\[f(d)-f(c) \ge \alpha^{\prime}(d-c)\]
	where $ 0 < \alpha^{\prime} \le \alpha $. Now take any $ x\in E_\alpha $. We hence see that $ D^{+}f(x) \ge \alpha $. Now for any $ \epsilon > 0 $, we can construct a closed bounded interval $ I = \left [x-\frac{\epsilon}{2}, x+ \frac{\epsilon}{2}\right ] $ for which $ \lom{I} = \epsilon $ with $ x\in I $. But moreover, we have that 
	\begin{align*}
		Df(x) = \frac{f(x+\epsilon/2) - f(x-\epsilon/2)}{\epsilon} &\ge \alpha \\
	f(x+ \epsilon/2) - f(x-\epsilon /2) &\ge \epsilon\alpha \ge \epsilon \alpha^{\prime}
	\end{align*}
Hence the interval $ \left [x-\frac{\epsilon}{2}, x+\frac{\epsilon}{2}\right ] \in \mathcal{F}$. Therefore, \textbf{\emph{$ \mathcal{F} $ covers $ E_\alpha $ in the sense of Vitali}}(!)\\\\
Now, by Vitali Covering Lemma (Theorem \ref{T-14}), we get that 
\[\forall\;\epsilon>0,\;\exists\;\text{finite disjoint }\{I_k\}_{k=1}^{n} \text{ from $ \mathcal{F} $ such that }\lom{E_\alpha \setminus \bunion_{k=1}^{n} I_k} < \epsilon\]
Now, observe that
\[E_\alpha \subseteq \bunion_{k=1}^{n}I_k \union \left (E_\alpha \setminus \bunion_{k=1}^{n} I_k\right )\]
Hence, by finite sub-additivity of outer measures, we get the following:
\begin{align*}
	\lom{E_\alpha} &\le \lom{E_\alpha \setminus \bunion_{k=1}^{n}} + \lom{\bunion_{k=1}^{n} I_k}\\
	&< \epsilon + \sum_{k=1}^{n} \lom{I_k}\\
	&\le \epsilon + \sum_{k=1}^{n} \frac{f(d_k) - f(c_k)}{\alpha^{\prime}}&&\text{Suppose $ I_k = [c_k,d_k] $.}
\end{align*}
Now since $ f $ is an increasing function and $ I_k \subset [a,b] \;\forall \;k$, therefore we have:
\begin{align*}
	\sum_{k=1}^{n} f(d_k) - f(c_k) \le f(b) - f(a)
\end{align*} 
That is,
\begin{align*}
	\lom{E_\alpha} < \epsilon + \frac{1}{\alpha^{\prime}} \cdot \left (f(b) - f(a)\right )
\end{align*}
But since $ \epsilon > 0 $ and $ \alpha^{\prime}\in (0,\alpha] $ are arbitrary, therefore,
\begin{align*}
	\lom{E_\alpha} \le \frac{1}{\alpha} \cdot \left (f(b) - f(a)\right )
\end{align*}
Hence proved.
\end{proof}
\newpage
\subsubsection{Lebesgue's Differentiation Theorem}
This is also one of the most important theorems of this course. This theorem portrays that monotonicity of a function is much better attribute of \emph{niceness} of a function than the usual belief of continuity, because we know example of continuous functions which is not differentiable, that is the \textbf{Weierstrass function}. But with this theorem, if we are given a monotone function on an open interval, then it ought to be differentiable almost everywhere on that interval. The same is obviously not true for just continuous functions.
\begin{theorem}\label{T-15}
	(\textbf{Lebesgue's Differentiation Theorem}) Suppose $ f $ is a \textbf{monotone} function on open interval $ (a,b) $ to $ \R $. Then, 
	\[f \text{ is differentiable on $ (a,b) $ almost everywhere (!)}\]
\end{theorem}
\begin{proof}
	\emph{Written in Diary at $ 1^{st} $ September, $ 2018 $. Typeset it here when time allows.}
\end{proof}
\begin{corollary}\label{C-2}
	A function $ f $ of bounded variation over an interval $ [a,b] $ is differentiable almost everywhere in $ (a,b) $.
\end{corollary}
\begin{proof}
	Lebesgue's Differentiation Theorem (\ref{T-15}) and the fact that any function of bounded variation is a difference of two increasing functions (Proposition \ref{P-46}). 
\end{proof}
\hrulefill
\newpage
\subsection{Integration \& Differentiation in Context}
We now learn some relationships between differentiation and integration. But let us begin with the following basic proposition.
\begin{proposition}\label{P-48}
	Let $ f : X \to [0,+\infty) $ be a Measurable function which is Lebesgue Integrable on a set $ E \subseteq X $. Then,
	\[\forall \;\epsilon > 0, \;\exists\; \delta > 0 \;\text{ such that } \forall \; A\subset E\;\text{ with }\;\lm{A} < \delta \;,\;\;\int_{A} f < \epsilon.\]
\end{proposition}
\begin{proof}
	Consider the following sequence $ \{f_n\} $ of functions:
	\begin{align*}
		f_n(x) &= \begin{cases}
			f(x) &\text{ if } f(x)\le n\\
			n &\text{ if }f(x)\ge n
		\end{cases}
	\end{align*}
Now, we see that $ f_n(x) \le f_{n+1}(x) $ because if $ f_n(x) = f(x) $ then $ f_n(x) = f(x) \le n < n+1 $ so that $ f_{n+1}(x) = f(x) $. Hence $ \{f_n\} $ is an increasing sequence. Therefore 
\[f_n \longrightarrow f \;\text{almost everywhere}\]
Hence, by Monotone Convergence Theorem (\ref{T-5}), we get that 
\[\int \lim_{n}f_n = \lim_{n} \int f_n.\]
Now, observe the following:
\begin{align*}
	\int_{E} f- \lim_{n} \int f_n &= 0\\
	\lim_{n}\int_E f - \lim_{n}\int f_n &= 0\\
	\lim_{n} \int_E \left ( f -f_n\right ) &= 0
\end{align*}
where we see that
\begin{align*}
	(f-f_n)(x)= \begin{cases}
		0&\text{ if } f(x) - n\le 0\\
		f(x) -n&\text{ if }f(x) - n\ge 0.
	\end{cases}
\end{align*}
From this, we can construct the following sequence of sets:
\begin{align*}
	E_n &= \{x\in E\where f(x) - n \ge 0\}.
\end{align*}
Again, we see that for any $ x\in E_n $, we would have $ f(x) \ge n > n-1$, so that $ x\in E_{n-1} $. Hence $ \{E_n\} $ is a decreasing sequence of subsets of $ E $.\\
We now observe that 
\[\int_{E_n} f \ge \int_{E_n} n = n\lm{E_n}.\]
Hence, we get that, for any $ n\in \N $,
\begin{align*}
	\lm{E_n} \le \frac{1}{n}\int_{E_n} f
\end{align*}
So that, we can choose $ n $ corresponding to any $ \delta = \epsilon/n $ such that 
\[\lm{E_n} \le \frac{1}{n}\int_{E_n} f < \delta  = \epsilon/n\] 
and
\[\int_{E_n} f < n\delta = \epsilon.\]
Hence proved.
\end{proof}
\hrulefill
\newpage
\subsubsection{Indefinite Integral}
The Indefinite Integral of a Lebesgue Integrable function forms a sort of \emph{bridge} between Integration and Differentiation.

\hrulefill
\begin{definition}
	(\textbf{Indefinite Integral}) Suppose $ f : [a,b] \longrightarrow \R $ is a Lebesgue Integrable function. We then define the indefinite integral of $ f $ as the following function:
	\begin{align*}
		F(x) = \int_{a}^{x} f(t) dt
	\end{align*}
\end{definition}
\hrulefill
\begin{proposition}
	Suppose $ f : [a,b]\longrightarrow \R$ is a Lebesgue Integrable function. Then,
	\begin{enumerate}
		\item {$ F(x)  $ is a \textbf{continuous} function on $ [a,b] $.}
		\item {$ F(x) $ is of \textbf{bounded variation} on $ [a,b] $.}
	\end{enumerate}
\end{proposition}
\begin{proof}
	\textbf{1.} Take any $ \epsilon > 0 $. By Proposition \ref{P-48}, we get that 
	\[F(x) = \int_{a}^{x} f(t)dt < \epsilon \;\implies \; \exists\;\delta > 0\;\&\; \exists A \subset [a,x] \text{ such that } \lom{A} < \delta.\]
In more precise words, $ \forall \epsilon > 0 $, $ \exists \delta > 0 $ such that whenever
\[\abs{a-x_0} = \lom{[a,x_0]} < \delta\]
then we would have
\begin{align*}
	\abs{F(x_0) - F(a) } &< \epsilon\\
	\abs{\int_{a}^{x_0} f(t)dt - \int_a^a f(t)dt} &< \epsilon\\
	\abs{\int_{a}^{x_0} f(t) dt} &< \epsilon 
\end{align*}
which is just the definition of continuity.\\\\
\textbf{2.} Take any partition of $ [a,b] $, say, $ \mathcal{P}([a,b]) = a= x_0 < x_1 < x_2 < \dots x_{k-1} < x_k = b$. Now, we see that,
\begin{align*}
	t_{\mathcal{P}} &= \sum_{i=1}^{k} \abs{F(x_i) - F(x_{i-1})} \\
	&= \sum_{i=1}^{k} \abs{\int_{a}^{x_i} f(t) dt - \int_{a}^{x_{i-1}} f(t)dt}\\
	&= \sum_{i=1}^{k} \abs{\int_{x_{i-1}}^{x_{i}} f(t)dt}\\
	&< \sum_{i=1}^{k} \int_{x_{i-1}} ^{x_i} \abs{f(t)} dt\\
	&< \infty 
\end{align*}
where last line follows because $ f $ is Lebesgue Integrable. Since our choice of partition $ \mathcal{P} $ was arbitrary, therefore $ \sup_{\mathcal{P}}t_{\mathcal{P}} < \infty $ hence $ F(x) $ is of bounded variation.
\end{proof}
\begin{corollary}
	The Indefinite Integral of a Lebesgue Integrable function is Differentiable almost everywhere.
\end{corollary}
\begin{proof}
	By-product of Lebesgue's Differentiation Theorem (\ref{T-15}), or more succinctly, Corollary \ref{C-2}. 
\end{proof}
\hrulefill
\newpage
\begin{proposition}
Suppose $ f : [a,b]  \longrightarrow \R$ is a Lebesgue Integrable function and 
\[F(x) = \int_{a}^{x_0} f(t)dt = 0 \;\forall\;x\in (a,b).\]
Then, $ f = 0 $ almost everywhere on $ (a,b) $.
\end{proposition}
\begin{proof}
		Suppose to the contrary that $ \exists\; E \subset [a,b] $ such that $ f(x) \neq 0 \;\forall\;x\in E $ with $ \lom{E} > 0 $. By Proposition \ref{P-16}, we get that $ \exists\;G\subset E $ which is closed such that $ \lom{G} > 0 $ and $ \lom{E\setminus G} = 0 $. Hence $ (a,b)\setminus G $ is open. Now consider the integral:
		\begin{align*}
			\int_{G} f &= \int_{(a,b)} f - \int_{(a,b)\setminus G} f
		\end{align*}
	We know that $ \int_{(a,b)} f = 0 $ as $ F(x) = 0\;\forall \;x\in(a,b) $. In a similar tone, we have $ \left. f\right \vert_{(a,b)\setminus G} = 0 $ almost everywhere because $ \lom{E\setminus G} = 0 $ and $ f = 0$ on $ (a,b) \setminus E $ anyways. Therefore, we have $ \int_G  f = 0$. But $ \left .f\right \vert_{G} \neq 0 $ by definition of $ G \subset E $. Hence we have a contradiction. Therefore such a set $ E $ cannot exist. Thus $ f = 0 $ almost everywhere on $ (a,b) $ if $ F = 0 \;\forall \;x\in (a,b) $.
\end{proof}
\hrulefill
\begin{theorem}
	Let $ [a,b] $ be a finite interval and let $ f : [a,b] \longrightarrow \R $ be a Lebesgue Integrable function over it. Then,
	\[F^{\prime} = f\text{ almost everywhere in }[a,b].\]
\end{theorem}
\begin{proof}
	\emph{Yet to be completed.}
\end{proof}
\hrulefill
\subsubsection{Absolutely Continuous Functions}
This is a more general form of continuity, and since it has connections with indefinite integral, we then learn them here.
\begin{definition}
	(\textbf{Absolutely Continuous Function}) A function $ f : [a,b] \longrightarrow\R $ is said to be Absolutely Continuous if
	\[\forall \;\epsilon > 0\;,\;\;\exists\;\delta > 0\text{ such that }\forall\text{ \textbf{finite} \& \textbf{disjoint} collection of open intervals }\{(a_k,b_k)\}_{k=1}^{n} \text{ each subset of }(a,b) \text{ which satisfies }\]
	\[\sum_{k=1}^{n}(b_k - a_k) < \delta,\]
	\[\text{also satisfies}\]
	\[\sum_{k=1}^{n} \abs{f(b_k) - f(a_k)} < \epsilon.\]
\end{definition}
\begin{remark} Some straightforward results are:
	\begin{itemize}
		\item {Any Absolutely Continuous function is Continuous in usual sense. This follows trivially from their definitions.}
		\item {Any Absolutely Continuous function is Uniformly Continuous. This also follows from the definition.}
	\end{itemize}
\end{remark}
\hrulefill
\begin{proposition}
	Suppose $ f : [a,b] \longrightarrow \R $ is an Absolutely Continuous function. Then, $ f $ is of Bounded Variation over $ [a,b] $.
\end{proposition}
\begin{proof}
	 Since $ f $ is an absolutely continuous function, therefore, for any fixed $ \epsilon > 0 $, we can construct the partition of $ [a,b] $, say $ \mathcal{P}^{\prime} = a = x_0 < x_1 <\dots< x_k = b$ such that each $ (x_{i-1},x_i) $ is of length $ < \delta $. Clearly, we would then have that
	 \[\sum_{i=1}^{k}\abs{f(x_i) - f(x_{i-1})} < k\epsilon.\]
	 Now, consider any arbitrary partition say $ \mathcal{P} \equiv a = y_0 < y_1 < \dots< y_N = b $ of $ [a,b] $. Collect the partition points of $ \mathcal{P} $ as the open disjoint intervals $ \left \{(y_{i-1}, y_i)\right \}_{i=1}^{N}$. Then, for each $ i^{th} $ interval in this partition, we can further partition it into $ k_i $ open disjoint intervals such that each has length $ < \delta $. In particular, we would have the following partition of $ [y_{i-1},y_i] $:
	 \[\left \{(z_{j-1}^{i}, z_{j}^{i})\right \}_{j=1}^{k_i}\text{ where }z_0^{i} = y_{i-1}\;,\;\;z^{i}_{k_i} = y_i.\]
	 Now, note that the variation of $ f $ over the $ [y_{i-1},y_i] $ would then be:
	 \begin{align*}
	 	\abs{f(y_i) - f(y_{i-1})} &= \abs{\sum_{j=1}^{k_i} f(z_{j}^{i}) - f(z_{j-1}^{i})}\\
	 	&\le \sum_{j=1}^{k_i} \abs{f(z_{j}^{i}) - f(z_{j-1}^{i})}\\
	 	&< \sum_{j=1}^{k_i} \epsilon = k_i \epsilon&&\text{$ \because $ $ z_j^{i} - z_{j-1}^{i} < \delta$ by construction}
	 \end{align*}
 Now, the variation over whole of $ \mathcal{P} $ would then be:
 \begin{align*}
 	t_{\mathcal{P}} &= \sum_{i=1}^{N} \abs{f(y_i) - f(y_{i-1})} \\
 	&< \sum_{i=1}^{N} k_i \epsilon\\
 	&< \infty
 \end{align*}
as $ k_i $ is finite for all $ i $. Hence proved.
\end{proof}
\hrulefill

This theorem relates Indefinite Integral of a Lebesgue Integral and Absolute Continuity.
\begin{theorem}
	Suppose $ f : [a,b] \longrightarrow \R$ is a Lebesgue Integrable function and it's Indefinite Integral is denoted by the function $ F(x) $. Then, 
	\[\text{$ F $ is an Indefinite Integral }\bm{\iff}\text{ $ F $ is Absolutlely Continuous.}\] 
\end{theorem}
\begin{proof}
	\emph{Yet to be completed.}
\end{proof}
\hrulefill
\newpage
\section{Signed Measures and Derivatives}
The concept of signed measures is the next generalization that we seek to understand. So far we have only encountered measures on a space which maps subsets to $ [0,\infty] $. But what happens when we \emph{increase} the co-domain to the whole $ [-\infty,\infty] $? First of all, we can see clearly that a measure shall not map some subsets to $ +\infty $ and some other subset to $ -\infty $ as we would then have the problem of $ \infty - \infty $, and since we are not doing set theory here, hence we would refrain ourselves only to such \emph{signed} measures which either maps to $(-\infty, \infty]  $ or $ [-\infty,\infty) $ but not both. 

Later we would see that having such a notion of \emph{signed} measure actually leads to some very striking results!
\begin{definition}
	(\textbf{Signed Measure}) Suppose $ (X,\alg{M}) $ is a Measurable Space. A function 
	\[\nu : \alg{M} \longrightarrow [-\infty,+\infty) \;\text{ \textbf{OR} }\nu : \alg{M} \longrightarrow (-\infty,+\infty]\]
	is called a Signed Measure if it satisfies:
	\begin{enumerate}
		\item {$ \nu $ atmost maps sets either to $ +\infty $ or $ -\infty $, but not both\footnote{Hence the two possible choices for the $ \nu $ above.}.}
		\item {$ \nu $ maps null-set to $ 0 $:
	\[\sm{\Phi} = 0\]	
	}
\item {$ \nu $ follows countable additivity:
\[\sm{\bunion_{i=1}^{\infty} A_i} = \sum_{i=1}^{\infty} \sm{A_i}\]}
where $ \{A_i\} $ is any sequence of disjoint sets from $ \alg{M} $.
	\end{enumerate}
\end{definition} 
\hrulefill
\begin{definition}
	(\textbf{Positive Set}) Suppose $ (X,\alg{M}) $ is a measurable space and $ \nu $ is a signed measure on it. Then a set $ A \in \alg{M} $ is said to be a positive set w.r.t. $ \nu $ if:
	\[\forall \;S\subseteq A\text{ such that $ S\in \alg{M} $}\;,\;\;\sm{S} \ge 0.\]
\end{definition}
\hrulefill
\begin{definition}
	(\textbf{Negative Set}) Suppose $ (X,\alg{M}) $ is a measurable space and $ \nu $ is a signed measure on it. Then a set $ B \in \alg{M} $ is said to be a negative set w.r.t. $ \nu $ if:
	\[\forall \;S\subseteq B\text{ such that $ S\in \alg{M} $}\;,\;\;\sm{S} \le 0.\]
\end{definition}
\begin{remark}
	One could alternatively say that a set  is a negative set if it is positive w.r.t. $ -\nu $.
\end{remark}
\hrulefill
\begin{definition}
	(\textbf{Null Set}) Suppose $ (X,\alg{M}) $ is a measurable space and $ \nu $ is a signed measure on it. Then a set $ N \in \alg{M} $ is said to be a null set w.r.t. $ \nu $ if 
	\[\text{$ N $ is both a Positive and Negative set w.r.t. $ \nu $}\]
\end{definition}
\hrulefill
\begin{proposition}
	Suppose $ (X,\alg{M}) $ is a measurable space and $ \nu  $ is a signed measure on it. Let $ \{A_i\} $ be a sequence of positive sets w.r.t. $ \nu $. Then,
	\[A= \bunion_{i}A_i \text{ is a Positive Set w.r.t. $ \nu $.}\]	
\end{proposition}
\begin{proof}
%	\emph{Written in Diary at $ 24^{th} $ September, $ 2018 $. Typeset it here when time allows.}
	We know that we can write the sequence $ \{A_i\} $ in the following form:
	\[\{B_i\}\text{ where $ B_i $'s are disjoint \& $ B_i \subseteq A_i $.}\]
	This can be easily be seen by $ B_1 = A_1$ and $ B_i = A_i \setminus B_{i-1} $. Hence $ \{B_i\} $ is a sequence of disjoint positive sets. Moreover, we can see that
	\[A = \bunion_{i} A_i = \bunion_{i} B_i.\]
	Now, take any subset $ E\subseteq  A $, which we can simply write as:
	\begin{align*}
		\sm{E} &= \sm{E\intrs A} \\
		&= \sm{\bunion_{i} E \intrs B_i}\\
		&= \sum_{i} \sm{E\intrs B_i}\\
		&>0
	\end{align*}
because $ E\intrs A_i\subseteq B_i $ and $ B_i $ is a positive set. Hence proved.
\end{proof}
\begin{remark}
	This is clearly also true for negative sets and null sets. That is, countable union of negative (null) sets is also a negative (null) set.
\end{remark}
\hrulefill
\begin{proposition}
	Suppose $ (X,\alg{M}) $ is a measurable space and $ \nu $ is a signed measure on it. If $ E\in \alg{M} $ is such that $ \sm{E} \ge 0 $, then
	\[\text{$ \exists \;A \subseteq E$ such that $ A $ is a Positive Set w.r.t. $ \nu $, $ A\in \alg{M} $ \& $ \sm{A} > 0 $. }\]
\end{proposition}
\begin{proof}
		\emph{Written in Diary at $ 26^{th} $ September, $ 2018 $. Typeset it here when time allows.}
\end{proof}
\hrulefill
\subsection{The Hahn Decomposition Theorem}
\begin{theorem}(\textbf{Hahn Decomposition Theorem})\label{T-18}
	Suppose $ (X,\alg{M}) $ is a measurable space and $ \nu $ is a signed measure on it. Then, 
	\[\exists \text{ Positive Set $ A \in \alg{M}$ and Negative Set $ B\in \alg{M} $} \text{ such that } \boxed{A\union B = X \;\&\;A\intrs B = \Phi}\]
	Moreover, any two such pairs $ (A,B) $ and $ (A^{\prime},B^{\prime}) $ are unique upto the fact that
	\[A\Delta A^{\prime}\;\&\;B\Delta B^{\prime} \text{ are $ \nu $-Null Sets}\]
	
\end{theorem}
\begin{proof}
		\emph{Written in Diary at $ 1^{st} $ October, $ 2018 $. Typeset it here when time allows.}
\end{proof}

\newpage

\subsection{The Jordan Decomposition of a Signed Measure}
We now, in a sense, generalize the Hahn Decomposition Theorem (\ref{T-18}), but to the signed measure $ \nu $ itself. As usual, let's first familiarize ourselves with some definitions.
\begin{definition}
	(\textbf{Mutual Singularity of Signed Measures}) Let $ \nu_1 $ and $ \nu_2 $ be two measures (NOT signed!) over measurable space $ (X,\alg{M}) $. Then $ \nu_1 $ and $ \nu_2 $ are called mutually singular if 
	\[\exists \; A\in \alg{M} \text{ such that }\nu_1(A) = \nu_2(\comp{A}) = 0\]
	and is then denoted by:
	\[\nu_1 \ms \nu_2.\]
\end{definition}
\hrulefill
\begin{theorem}(\textbf{Jordan Decomposition Theorem})\label{T-19}
	Suppose $ (X,\alg{M}) $ is a measurable space and $ \nu $ is a signed measure on it. Then, 
	\[\exists \text{ Measures }\nu^{+}\;\&\;\nu^{-}\text{ on $ (X,\alg{M}) $ such that } \boxed{\nu= \nu^{+} - \nu^{-} \;\&\; \nu^{+}\ms \nu^{-}}\]
	and such a decomposition of $ \nu $ is unique.
\end{theorem}
\begin{proof}
	\emph{Written in Diary at $ 5^{th} $ October, $ 2018 $. Typeset it here when time allows.}	
\end{proof}
\newpage
\subsection{The Radon-Nikodym Theorem}
This is one of the final and most important theorems of this course. As we will see, this theorem gives us a notion of the derivative of a signed measure. However, we would not go more deeper into that fact. \\
As usual, we first introduce some definitions.
\begin{definition}
	(\textbf{Total Variation of a Signed Measure}) The total variation of a signed measure $ \nu $ over some measurable space is defined by
	\[\abs{\nu} = \nu^{+} + \nu^{-}\]
	where $ \nu = \nu^{+} - \nu^{-}$ is the Jordan Decomposition (Theorem \ref{T-19}) of $ \nu $. 
\end{definition}
\begin{remark}
	Since $ \nu^{+} $ and $ \nu^{+} $ are the usual measures on the measurable space, therefore $ \abs{\nu} $ is also a usual measure on the same measurable space.
\end{remark}
\hrulefill
\begin{definition}
	(\textbf{$ \sigma $-Finite Signed Measure}) Suppose $ \nu $ is a signed measure on measurable space $ (X,\alg{M}) $. Then $ \nu $ is called $ \sigma $-Finite if 
	\[\exists\;\{X_n\}_{n=1}^{\infty}\text{ where $ X_i \in \alg{M} $ and $ \abs{\nu(X_i)} < \infty $ such that }\bunion_{n=1}^{\infty} X_n = X \]
\end{definition}
\begin{remark}
	$ \nu $ is $ \sigma $-Finite $ \bm{\iff} $ $ \abs{\nu} $ is $ \sigma $-Finite.
\end{remark}
\hrulefill
\begin{definition}
	(\textbf{Absolute Continuity of Usual Measures}) Suppose $ \lambda $ and $ \gamma$ are usual measures over a measurable space $ (X,\alg{M}) $. If,
	\begin{align*}
\lambda(E) = 0 \text{ for some }  E\in \alg{M} \implies \gamma(E) = 0
	\end{align*}
	always, then $ \gamma $ is said to be absolutely continuous w.r.t. $ \lambda $. This is denoted by $ \gamma \ll \lambda $.
\end{definition}
\hrulefill
\begin{definition}
	(\textbf{Absolute Continuity of Signed Measures}) Suppose $ \mu $ and $ \nu $ are signed measures over a measurable space $ (X,\alg{M}) $. If,
	\[\abs{\mu}(E) = 0 \text{ for some } E\in \alg{M} \implies \nu(E) = 0\]
	always, then $ \nu $ is called absolutely continuous w.r.t. $ \mu $. This is denoted by $ \nu \ll \mu $.
\end{definition}
\hrulefill
\begin{theorem}\label{T-20}
	(\textbf{Radon-Nikodym Theorem}) Suppose $ (X,\alg{M}) $ is a measurable space and $ \lambda $ \& $ \gamma $ are two $ \sigma $-finite measures on it such that $ \gamma \ll \lambda $. Then,
	\begin{align*}
		\exists \;\text{Measurable }&\text{Function w.r.t. $ \lambda $ }  f : X \longrightarrow [0,\infty)  \text{ such that}\\
		&\boxed{\gamma(E) = \int_{E}fd\lambda \;\forall\;E\in \alg{M}}
	\end{align*}
Moreover, $ f $ is unique up-to almost everywhere equality, that is, if $ \gamma(E) =\int_{E}gd\lambda \;\forall\;E\in\alg{M}$, then,
\[f = g\text{ almost everywhere on $ X $ w.r.t. $ \lambda $.}\]
\end{theorem}
\begin{proof}
	\textbf{\emph{Yet to be completed.}}
\end{proof}
\newpage
\section{The Dual of $ \Lp{\R^{n}} $ : Riesz Representation Theorem}
\begin{definition}
	(\textbf{Linear Functional}) Suppose $ (V,\R,\norm{\cdot})$ is a Banach Space. A \textbf{linear}\footnote{$ f(\alpha v_1 + \beta v_2) =  \alpha f(v_1) + \beta f(v_2) \;\forall \;v_1,v_2 \in V$ and $ \alpha,\beta \in \mathbb{R} $. Or more simply, a morphism in the Category of Vector Spaces \textbf{Vect} :):} map $f : V \longrightarrow \R  $ is called a linear functional.
\end{definition}
\hrulefill
\begin{definition}
	(\textbf{Bounded Linear Functional}) A linear functional $ \phi : V\longrightarrow\R $ where $ (V,\R,\norm{\cdot}) $ is a Banach space is called bounded if
	\[\exists\; c \ge 0 \text{ such that }\abs{\phi(x)} \le c\norm{x}\;\forall\;x\in V.\]
	The space of all such bounded linear functionals is denoted by
	\[\alg{B}(V).\]
	That is, any $ \phi \in \alg{B}(V) $ is a bounded linear functional.
\end{definition}
\hrulefill
\begin{proposition}
	Suppose $ (V,\R,\norm{\cdot}_V) $ is a Banach Space and $ \sblf{V} $ is the space of bounded linear functionals over $ V $. Then,
	\[\sblf{V}\text{ forms a Vector Space}\]
	and the map
	\[\norm{\cdot} : \sblf{V} \longrightarrow [0,\infty)\]
	defined by
	\begin{align*}
		\norm{\phi} &= \sup\left \{\frac{\abs{\phi(x)}}{\norm{x}_V} \;:\: x\in V\right \}\\
		&= \inf \left \{c\;:\: \abs{\phi(x)} \le c\norm{x}_V\;,\;\;x\in V\right \}
	\end{align*}
for any $ \phi \in \sblf{V} $ forms a norm on the Vector Space $ \sblf{V} $.
\end{proposition}
\begin{proof}
	Take any two $ \phi_1,\phi_2 \in \sblf{V} $ and $ \alpha ,\beta \in \R $. By the very nature of their existence, $ \phi_1\;\&\; \phi_2 $ have to be bounded linear functionals. Suppose 
	\begin{align*}
		\abs{\phi_1(x)} &\le c_1\norm{x}_V\\
		\abs{\phi_2(x)} &\le c_2 \norm{x}_V
	\end{align*}
$ \forall\; x\in V $. Then $ \alpha \phi_1  $ is a function such that:
\begin{align*}
	\abs{\alpha\phi_1(x)} \le \alpha c_1 \norm{x}_V
\end{align*}
Hence $ \alpha \phi_1 \in \sblf{V}$. Similarly, $ \beta\phi_2 \in \sblf{V} $. Now, since we have that
\begin{align*}
	\abs{\phi_1 + \phi_2 } \le \abs{\phi_1} + \abs{\phi_2} 
\end{align*}
Therefore, $ \phi_1 + \phi_2 \in \sblf{V} $. Hence, $ \sblf{V} $ is a Vector Space.\\\\
To see that $ \norm{\cdot } $ is a norm over $ \sblf{V} $, we see that for any $ \alpha\in \R $ and $ \phi \in \sblf{V} $, we trivially get that
\begin{align*}
	\norm{\alpha\phi} &= \abs{\alpha} \norm{\phi}
\end{align*}
and, for $ f_1,f_2 \in \sblf{V} $, we also note that
\begin{align*}
	\norm{f_1 +f_2} &\le \norm{f_1} + \norm{f_2}
 \end{align*}
Hence, $ \norm{\cdot} $ is a norm on Vector Space $ \sblf{V} $.
\end{proof}
\hrulefill
\newpage
\subsection{$ \sblf{V} $ is a Banach Space!}
\begin{proposition}
	Suppose $ V $ is a Banach Space. Then $ \sblf{V} $ is a Banach Space.
\end{proposition}
\begin{proof}
	Take any Cauchy sequence $ \{\phi_n\} $ in $ \sblf{V} $. Now, since $ \phi_n $'s are bounded linear functionals, therefore,
	\[\exists\; c_n\ge 0 \text{ such that }\abs{\phi_n(x)} \le c_n \norm{x}_V\;\forall\;x\in V \]
	Now take any $ x\in V $. Since $ \phi_n(x) \in V$, we therefore have a sequence $ \{\phi_n(x)\} $ in $ \R $. We now note that
	\begin{align*}
		\abs{\phi_n(x) - \phi_m(x)} &\le \norm{x}_V\times \sup\left \{\frac{\abs{\phi_n - \phi_m}}{\norm{x}_V}\;:\;x\in V\right \}\\
		&= \norm{\phi_n - \phi_m} \norm{x}_V.
	\end{align*}
	Hence, $ \{\phi_n(x)\} $ is a Cauchy Sequence in $ \R $. Now write 
	\[\phi(x) = \lim_{n} \phi_n(x).\]
	Since $ \R $ is complete, therefore $ \phi(x) \in \R$. But our choice of $ x $ was arbitrary, hence $ \phi(x) = \lim_{n}\phi_n(x) < \infty\;\forall\;x\in V$. Hence $ \phi \in \sblf{V} $. \\\\
	Moreover,
	\begin{align*}
		\lim_n\norm{\phi_n} &=  \lim_n\sup\left \{\frac{\abs{\phi_n(x)}}{\norm{x}_V}\;:\; x\in V\right \}\\
		&= \sup\left \{\lim_n\frac{\abs{\phi_n(x)}}{\norm{x}_V}\;:\; x\in V\right \}\\
		&= \sup\left \{\frac{\abs{\lim_n \phi_n(x)}}{\norm{x}_V}\;:\; x\in V\right \}\\
		&= \sup\left \{\frac{\abs{\phi(x)}}{\norm{x}_V}\;:\; x\in V\right \}\\
		&= \norm{\phi}
	\end{align*}
Hence proved.
 \end{proof}
\hrulefill
\newpage
\begin{thebibliography}{9}
	\bibitem{Solovay70} 
	Solovay, R. (1970). A Model of Set-Theory in Which Every Set of Reals is Lebesgue Measurable. Annals of Mathematics, 92(1), second series, 1-56.
\end{thebibliography}
\end{document}
